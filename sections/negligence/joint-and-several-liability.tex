\subsection{Joint and Several Liability}

\subsubsection{Comparative Indemnity: \emph{American Motorcycle Association v. 
Superior Court}}

\begin{enumerate}
    % todo: add facts
    \item Common law rule: defendants cannot bring in other parties to 
    lawsuits, because (1) plaintiffs should be able to control their own 
    cases, and (2) we should not take court time to shift the loss from one 
    wrongdoer to another.
    \item 40s onward (1957 in CA): \textbf{contribution states} (``Uniform 
    Contribution Act'') prevent defendants from bringing in other defendants, 
    but if there were already two defendants in the suit, and plaintiff 
    collected all damages from one defendant, a defendant could collect pro 
    rated damages from another (e.g., if there were two defendants, they'd 
    each pay 50\%; three defendants would pay 33\%; etc.).
    \item Appellate court here eliminated the doctrine of joint and several 
    liability---thus the Supreme Court's care to reaffirm that the doctrine is 
    still good law.
    \item Rules from \emph{American Motorcycle}:
    \begin{enumerate}
        \item % todo see pp. 470-471
    \end{enumerate}
    \item One of the rationales for not allowing defendants to bring in other 
    defendants is that it would greatly complicate a plaintiff's case. After 
    \emph{American Motorcycle}, \item Plaintiffs can settle with secondary 
    defendants that the primary defendant brings in. But if a third defendant 
    settles with the plaintiff for a disproportionately small amount, he can 
    still be liable to the main defendant for remaining damages 
    (\emph{Tech-bilt v. Woodward-Clyde}). So, the defendants have a quick 
    hearing to make sure the settlement is within a reasonable range.
    \item Plaintiffs will settle with one of multiple defendants because (1) 
    first settlement can pay for the rest of the case, and (2) early in the 
    case you can play defendants off each other (e.g., make settlement offers 
    to both, and offer to accept the first).
    \item \emph{American Motorcycle} dramatically changed day-to-day 
    lawyering. Now, around 20\% of cases involve multiple defendants. Before 
    \emph{American Motorcyle}, a case could not have had multiple defendants.
    \item The set-off problem (\emph{Jess v. Herrmann}):
    \begin{enumerate}
        \item What if A is 75\% at fault and B is 25\% at fault, and both 
        suffer \$100,000 injuries? Should (1) A receive \$25,000 in insurance 
        money and B receive \$75,000, or (2) B recover \$50,000 and A nothing 
        (the set-off problem)?
        \item (Levy: under contract law, there would not be set-off unless 
        insurance companies had expressly stipulated it.)
    \end{enumerate}
\end{enumerate}

\subsubsection{Proposition 51: Fair Responsibility Act of 1986}

\begin{enumerate}
    \item Non-economic damages: things without a price tag, e.g., pain and 
    suffering.
    \item Rules:
    \begin{enumerate}
        \item The full economic damages can be recovered from any defendant 
        individually---i.e., defendants are joint and severally liable for 
        their collective share of the blame. For instance, if the plaintiff is 
        10\% at fault, B is 30\% at fault, and C is 60\% at fault, then B and 
        C are jointly and severally liable for 90\% of the damages.
        \item Non-economic damages can only be recovered proportionally to 
        each defendant's degree of blame. For instance, if total non-economic 
        damages are \$200,000, and defendant B is 30\% liable, plaintiff can 
        recover \$60,000 in non-economic damages from plaintiff B.
    \end{enumerate}
    \item Example: hotel fails to put a proper lock on a door; someone breaks 
    in and rapes a tenant. The economic losses can be small, but the 
    non-economic consequences can lead to disastrous consequences for the 
    hotel owner.
    \item \end{enumerate}


% todo:
% History of the law when multiple tortfeasors are involved
% ---------------------------------------------------------
% 
% [see Levy PPT.]
% 
% Hypo: A and B both cause injury to C.
% 
% Common law: Plaintiff can sue both. If plaintiff only sued A, A could not 
% implead B.
% 
% Contribution statute: plaintiff sues both. Collects all from A. A can 
% recover the pro rata contribution from B, so they each end up paying half.
% --> if plaintiff only sued A, A could *not* implead B under the contribution 
%  statute.
% 
% After _American Motorcycle_: defendant can implead another tortfeasor. 
% (Argument against this model: plaintiff should be able to control his own 
% suit. E.g., a simple auto accident could become a complicated products 
% liability case. To avoid this scenario, the plaintiff could settle with the 
% new defendant (or the old)--and it must be a good faith settlement--see 
% _tech-bilt v. woodward-clyde_ which fleshes out the good faith settlement 
% question. there must be a short good faith hearing. after settling, 
% defendant is no longer liable for damages or indemnity. notice of the 
% settlement must be given to the other defendants.)

% once a defendant settles, the defendant is no longer liable for damages or 
% indemnity.

