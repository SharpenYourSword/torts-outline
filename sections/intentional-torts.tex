\section{Intentional Torts}

\subsection{Intent}

\begin{enumerate}
    \item Intent requires \textbf{desire} or \textbf{substantial certainty}.
    \item Reckless behavior can sometimes suffice for intent (see below). % TODO
    \item ``...the law of torts is not criminal law and does not condemn, but only shifts the burdens of economic loss.''\footnote{\emph{Understanding Torts}, p. 6.}
    \item Can a child meet the requirements for intent? \textbf{\emph{Garratt v. Dailey}}: A five year old moved a chair from the place where the plaintiff was about to sit. The plaintiff fell and fractured her hip. The plaintiff's battery claim requires proof that the defendant had intent to cause contact that was not consensual or otherwise privileged. The Second Restatement indicates that intent exists if the actor is \textbf{substantially certain} that the harmful contact \textbf{will} (not might) occur. Court finds that it's unclear whether the defendant had substantial certainty. Remanded to trial court for clarification.
        \begin{enumerate}
            \item When should infancy make a difference in intent?
        \end{enumerate}
    \item Can an insane person meet the requirements for intent? \textbf{\emph{Williams v. Kearbey}}: Minor shot up a school and claimed insanity. Court held that defendant intended to commit the action (even if his motivation was irrational) and is therefore liable.
        \item Insanity is never a defense to intentional torts. It can sometimes be used as a defense in criminal law.\footnote{See Dressler, \emph{Criminal Law}, p. 613.}
    \item Other notes on torts and intent:
    \begin{enumerate}
        \item Torts are generally excepted from workers' comp immunity.
        \item In most jurisdictions, you can't insure against intentional torts.
        \item The constitution's Supremacy Clause leads to three kinds of preemption of federal laws over state laws:
        \begin{enumerate}
            \item \emph{Express preemption}: Explicit or implicit overriding of a state statute.
            \item \emph{Conflict preemption}: In case of direct conflict, federal law preempts state law.
            \item \emph{Field preemption}: Congress legislates for an entire field of regulation, leaving no room for states to regulate.
        \end{enumerate}
        \item The Second Restatement on Torts blends purpose and knowledge (i.e., substantial certainty) into one rule. The Third Restatement proposes separating them into two distinct rules, since limiting liability to ``purpose'' can have consequences in areas like workplace litigation.
        \item Inadvertent results of actions are not intentional. (But, mistakes usually constitute intent---see below.)     
        \item The substantial certainty test significantly expands the use of intentional torts in workplace and environmental litigation.
        \item A few jurisdictions have rejected the substantial certainty rule.\footnote{Casebook p. 6.}
    \end{enumerate}
    \item Can the different approaches towards insanity and infancy and torts and criminal law be justified?
\end{enumerate}

\subsection{Battery}

\begin{enumerate}
    \item Battery requires \textbf{intent to cause harmful or offensive contact} and that harmful or offensive contact directly or indirectly results.
    \begin{enumerate}
        \item ``Unpermitted'' touching can be enough---see \emph{White v. University of Idaho}, where a piano teacher touched a student's back and caused significant injury.
        \item Any touching in anger can also be enough.
    \end{enumerate}
    \item Is battery actionable for very small harms? \textbf{\emph{Leichtman v. WLW Jacor Communications, Inc.}}: A cigar smoker blew smoke in the face of an anti-smoking advocate. The court finds that ``No matter how trivial the incident, a battery is actionable...'' But it rejects the ``smoker's battery'' (which imposes liability if there is substantial certainty that second-hand smoke will touch a nonsmoker).
    \item Does compliance with safety standards affect liability for intentional torts? Can radiation constitute contact? \textbf{\emph{Bohrmann v. Maine Yankee Atomic Power Co.}}: University of Southern Maine students took a tour of a nuclear power plant. Plaintiffs allege the company knew a flushing procedure would release radioactive gases during the tour, and that tour guides knowingly took students through plumes of unfiltered radioactive gases. Plaintiffs also allege the company falsely told them they had not been exposed to ``bad'' radiation. The court holds that compliance with federal safety standards does not affect the defendant's liability for intentional acts.
    \item The proposed Third Restatement would limit intent liability based on substantial certainty to small, localized groups of people. So, for instance, tobacco companies would not be liable for second-hand smoke damages.
    % TODO: environmental battery based on substantial certainty
\end{enumerate}

\subsection{Assault}

\begin{enumerate}
    \item The threat or use of force on another that causes that person to have a \textbf{reasonable apprehension of imminent harmful or offensive contact}.
    \item The Second Restatement does not require apprehension to be ``reasonable,'' but most courts do.
    \item Assault in torts is different than assault in criminal law. The criminal law definition requires an attempt to inflict harmful or offensive contact, but it does not require perception.\footnote{Casebook p. 21.}
    \item Can damages be awarded if physical harm did not occur? \textbf{\emph{I de S et Ux v. W de S}}: Defendant tries to buy wine from the plaintiff. He beats on the door with a hatchet. When the plaintiff's wife asks him to stop, he tries to hit her with the hatchet (but did not hit her). The court ruled that the plaintiff was entitled to damages even though no physical harm was done.
    \item Can forward looking verbal threats suggest imminent harm? \textbf{\emph{Castro v. Local 1199, National Health \& Human Services Employees Union}}: Plaintiff has asthma, which prevented her from working in extremely hot or cold situations. After a disability leave, she attended a meeting where she didn't receive her usual assignment. She asked what was going on, and Frankel (another employee) replied, ``If I was you, I would take whatever they give me, because you could lose more than your job.'' When asked he was threatening her life, Frankel said, ``Take it any way you want.'' The court held that \textbf{verbal threats, without ``circumstances inducing a reasonable apprehension of bodily harm,'' do not constitute an assault}. Here, the threat was ``forward-looking'' and did not suggest imminent harm. The court granted the defendant's motion for summary judgment.
    \item More questions on assault:
    \begin{enumerate}
        \item Is it possible to rationalize the difference between the criminal and tort definitions of assault?
        % TODO Plaintiff must be contemporaneously aware
        % threat must be unconditional
        \item Why prevent assault?
        \item Why must assault be imminent?
        \item Can words alone ever be enough to constitute assault? See \emph{Campbell v. Kansas State University}, where a university head said ``he felt like hitting his assisstant on the buttocks, after he had already slapped her on the buttocks,'' which the court held to be assault.\footnote{Casebook p. 20.}
    \end{enumerate}
\end{enumerate}

\subsection{Transferred Intent}

\begin{enumerate}
    \item Historically, transferred intent means that intent to commit any of the five traditional torts (battery, assault, false imprisonment, trespass to land, trespass to chattel---because these are torts where you would file a writ of trespass in old English contexts) can constitute the necessary intent to commit any of the other five. 
    \item Transferred intent is a legal fiction.
    \item The Second Restatement only incorporates transferred intent for assault and battery.
    \item Generally, \textbf{intent towards anyone for anything is intent towards anyone else for anything else}.
    \item Does the intended target matter? \textbf{\emph{Alteiri v. Colasso}}:  The defendant threw a rock that hit the plaintiff in the eye, but he intended to scare somebody else. He did not intend to hit anyone, and he did not throw the rock recklessly. The court ruled that there was no error in the jury's verdict for willful battery.
    \item Is it appropriate to transfer intent from a property tort to a personal tort?
\end{enumerate}

% TODO Add subsection on mistake doctrine
% Why are mistakes deemed intentional and not negligence or recklessness? -- plaintiff may sometimes prefer to make a negligence claim (e.g., in insurance contexts) 

\subsection{False Imprisonment}

\begin{enumerate}
    \item Intentional, unlawful, and unconsented restraint.
    \item Can the implied threat of physical restraint be enough to constitute false imprisonment? \textbf{\emph{Dupler v. Seubert}}: Dupler was fired from her job. Her superiors (including Deubert) kept her against her will in a 1.5-hour meeting. Dupler argued that Deubert and the other defendant screamed and shouted. The trial jury found false imprisonment and awarded damages of \$7,500. The trial judge offered a remittitur of \$500, and Deubert appealed. The Supreme Court of Wisconsin affirmed the order, holding that false imprisonment occurred when Dupler was held against her will after her hours of employment had ended at 5 PM (in contrast to Weiler v. Herzfeld-Phillipson, where the imprisonment occurred during work hours).
    \item According to the Second Restatement, confinement may be caused by:
    \begin{enumerate}
        \item Physical barriers.
        \item Force or threat of immediate force.
        \item Omissions where there is a duty to act.
        \item False arrest.
    \end{enumerate}
    \item Victim must be confined in a bounded area (e.g., if movement is allowed in any direction, even if it's not the desired direction, false imprisonment did not occur).
    \item The victim must usually be conscious of confinement, but not always (e.g., infant abduction).
    \item False imprisonment usually does not recognize highly coercive but non-physical threats (e.g., economic retaliation).
    \item Lawful restraint does not constitute false imprisonment
    \item \emph{Additur} and \emph{remittitur}: after jury delivers damages, judge adjusts them up or down.
    \begin{enumerate}
        \item Should appellate courts be allowed to issue remittances? Levy says no, because appellate courts get a thin version of the case (only transcripts, etc.) and more is needed to make an accurate determination about damages.
        \item There are generally no limits on the damages a jury can award (with a few exceptions).
    \end{enumerate}
    \item \emph{Shopkeeper's privilege}: Shopkeepers can detain suspected shoplifters.
    \item ``[A] form of false imprisonment whereby the improper assertion of legal authority can unlawfully restrain a victim.''\footnote{Casebook p. 38 n. 1.}
\end{enumerate}

\subsection{Malicious Prosecution}

\begin{enumerate}
    \item Second Restatement requires initiation of proceedings without probable cause and for a purpose other than bringing the offender to justice. It also requires that the proceedings have terminated in favor of the accused---so, a defendant who is sued and loses can't claim malicious prosecution (i.e., the defendant must have been exonerated to have a cause of action for malicious prosecution).\footnote{Understanding Torts p. 48.} The defendant can't bring a malicious prosecution claim until the initial suit is resolved, and if charges are dropped, there are no grounds for malicious prosecution. % TODO: is this true? check with Levy ####### purpose: prevent frivolous lawsuits for malicious prosecution
    \item Anti-SLAPP statutes also help prevent against frivolous litigation.
    \item Some jurisdictions recognize malicious prosecution only in criminal contexts, with the parallel civil tort ``wrongful institution of civil proceedings.''\footnote{Casebook p. 39 n. 2.}
    \item The ``American Rule'' dictates that the loser in a suit doesn't have to pay the winner's legal fees (in contrast to the ``British Rule'').
    \item Can a legal process be used for a purpose other than that for which it was intended? \textbf{\emph{Maniaci v. Marquette University}}: Saralee Maniaci became dissatisfied with Marquette University and, with her father's permission, decided to leave. School administrators tried to persuade her not to leave. When they were unsuccessful, they requested that the Milwaukee police bring papers for temporary emergency detention in a mental hospital for people considered ``irresponsible and dangerous.'' The school physician, the Dean of Women, and a registered nurse signed the application for temporary custody.

She was held for a night until her father demanded her release. Maniaci and her father filed suit on multiple charges, all of which were dismissed except false imprisonment. The jury assessed compensatory and punitive damages, which the court reduced on motions after the verdict.

The defendants appeal, arguing that the only legitimate cause of action was \textbf{malicious prosecution}, and moreover that the evidence was insufficient to prove malicious prosecution and that the damages were excessive. Court agrees that there is no cause of action for false imprisonment because the restraint was ``lawful.''

The court does not find that malicious prosecution applies. It finds no malice because the defendants ``had a genuine concern for the plaintiff's welfare.''

The court believes there is support---``skeletally at least''---for a cause of action on the basis of abuse of process. The defendants did not have serious concerns about Maniaci's mental health. Rather, their purpose was to restrain her until her parents had been notified of her decision to leave school, and had either given their permission or directed Saralee to stay in school.

Judgment is reversed and remanded, and the plaintiffs are required to amend their claim to allege cause of action for abuse of process.
    \item You can file multiple (and in California, even contradictory) causes of action.
\end{enumerate}

\subsection{Abuse of Process}

\begin{enumerate}
    \item Misuse of legal process for an ulterior purpose.
    \item Does not require termination of the legal process in favor of the one bringing the complaint (or even termination at all).
    \item Can a legal process be used for a purpose other than that for which it was intended? \textbf{\emph{Maniaci v. Marquette University}}: Saralee Maniaci became dissatisfied with Marquette University and, with her father's permission, decided to leave. School administrators tried to persuade her not to leave. When they were unsuccessful, they requested that the Milwaukee police bring papers for temporary emergency detention in a mental hospital for people considered ``irresponsible and dangerous.'' The school physician, the Dean of Women, and a registered nurse signed the application for temporary custody.

She was held for a night until her father demanded her release. Maniaci and her father filed suit on multiple charges, all of which were dismissed except false imprisonment. The jury assessed compensatory and punitive damages, which the court reduced on motions after the verdict.

The defendants appeal, arguing that the only legitimate cause of action was \textbf{malicious prosecution}, and moreover that the evidence was insufficient to prove malicious prosecution and that the damages were excessive. Court agrees that there is no cause of action for false imprisonment because the restraint was ``lawful.''

The court does not find that malicious prosecution applies. It finds no malice because the defendants ``had a genuine concern for the plaintiff's welfare.''

The court believes there is support---``skeletally at least''---for a cause of action on the basis of abuse of process. The defendants did not have serious concerns about Maniaci's mental health. Rather, their purpose was to restrain her until her parents had been notified of her decision to leave school, and had either given their permission or directed Saralee to stay in school.

Judgment is reversed and remanded, and the plaintiffs are required to amend their claim to allege cause of action for abuse of process.

% TODO add Anti-SLAPP: CCP 425.16 -- and see others from Levy PPTs

\end{enumerate}

\subsection{Intentional Infliction of Emotional Distress}

\begin{enumerate}
    \item ``Intentional infliction of emotional distress occurs when the defendant, through extreme and outrageous conduct, intentionally or recklessly causes the victim severe emotional distress.''\footnote{Casebook p. 44 n. 1.}
    \item Can overlap with other torts---e.g., wrongful termination, sexual/racial harassment.
    \item Not a historic tort, but a product of the 20th century. The torts (above) all have rigid factors. Courts invented intentional infliction to get around these restrictions.
    \item No need to prove physical injury. Most states require a showing of outrageous behavior beyond all reasonable bounds of decency.
    \item The relationship between the plaintiff and defendant can impact the court's characterization of the conduct as extreme or outrageous.
    \item Should employers be liable for wrongful termination as well as intentional infliction of emotional distress?
    \item What constitutes gross recklessness or intent to cause severe distress? \textbf{\emph{Slocum v. Food Fair Stores of Florida, Inc.}}: A shopper in a store asked the price of an item. An employee replied, ``if you want to know the price, you'll have to find out the best way you can...you stink to me.'' She had a heart attack and sued for intentional infliction of emotional distress. The court denied the claim, reasoning that the language did not constitute ``gross recklessness,'' nor was it intended to cause ``severe emotional distress.''
    \begin{enumerate}
        \item Would racial identities have affected the court's holding?
        \item Levy thinks \emph{Slocum} is wrong---the jury's verdict should have been taken into account.
    \end{enumerate}
    \item \textbf{\emph{Rulon-Miller v. International Business Machines Corporation}}: The plaintiff, a longtime IBM employee carried on a relationship with an employee at a rival office products firm, QYX. Her managers at first indicated they did not think the relationship constituted a conflict of interest---``I don't have any problem with that.'' But then her manager told her to end the relationship or lose her job, giving her ``a couple of days to a week'' to think about it. The next day, he said ``he had made up her mind for her'' and dismissed her. The court held that the manager ``intended to emphasize that she was powerless to to do anything to assert her rights,'' affirming the judgment for intentional infliction of emotional distress.
    \item Does sexual harassment constitute intentional infliction of emotional distress? \textbf{\emph{Jones v. Clinton}}: Paula Jones claimed Bill Clinton's ``actual exposure of an intimate private body part'' constituted extreme and outrageous conduct. The court found no evidence that the incident caused any significant lasting emotional distress and rejected the claim in a summary judgment.
    \item Spectrum of intent:
    \begin{enumerate}
        \item Desire.
        \item Substantial certainty.
        \item Negligence.
        \item Gross negligence.
        \item Recklessness.
    \end{enumerate}
    \item The plaintiff's sensitivity usually isn't enough---e.g., \emph{Nickerson v. Hodges}, where a woman believed her dead relatives had buried a pot of gold in her backyard. The defendants buried a pot of dirt, which she opened at the bank, expecting gold. The court found that the joke caused her extreme distress.
    % TODO: is sexual harassment distinct from intentional infliction? -- probably there are statutory restrictions -- e.g., FEHA
\end{enumerate}

\subsection{Defenses to Intentional Torts}

% TODO: add subsection for consent
% in CA and most states, plaintiff must prove lack of consent as part of the prima facie case.

\subsubsection{Self Defense}

% TODO: defendant must prove it (right? see book)
% TODO: Consent: states split about whether it's a defense, or whether absence of consent is part of the plaintiff's prima facie case.

% TODO: old cases: the aggressor can *never* have a cause of action. not so much today. 


\begin{enumerate}
    \item Force intended to inflict death or serious injury must be necessary and is only reasonable in response to the \textbf{immediate threat of serious bodily injury or death}.
% TODO    \item Valid even if used mistakenly. It's so instinctual that we need a different rule. (Compare to mistake in plaintiff's prima facie case, where there is liability). % shooting dogs, etc. -- mistake doctrine -- p. 24 ranson v kitner
    \item The Restatement of Torts also requires retreat if safely possible (except from the victim's own dwelling) before the victim can respond with force intended to inflict serious bodily injury or death. Most courts disagree.\footnote{Casebook p. 64.}
    \item If the threat is not immediate, self-defense is not valid. There is dispute about spousal abuse cases, however---should the smaller spouse be required to wait until the physical threat is immediate before asserting the right to self-defense?
    \item There is a limited right to self defense against excessive police force.
    \item Reasonable mistakes in perceiving threats can be valid bases for self defense.
    \item Should good samaritans be encouraged to intervene? The Second Restatement allows bystanders to assert self defense if they reasonably believe that the third party has a privilege of self defense and that intervention is necessary to protect him. The traditional rule, however, only allows intervention when the third party is privileged. The Second Restatement would allow reasonable mistakes.
    \item \textbf{\emph{Drabek v. Sabley}}: Ten-year-old Drabek and friends were throwing snowballs at passing cars. One driver, Sabley, stopped, caught Drabek, took him by the arm to his car, and drove him back to the village of Williams Bay. He turned Drabek over to the police. Drabek was with Sabley for a total of 15-20 minutes. The court held that Sabley was justified in preventing the commission of a crime, and so it was reasonable to admonish Drabek and march him home. But it was not reasonable to detain him and take him to the police station, so Sabley is liable for false imprisonment and nominal battery. Remanded to determine compensatory (but not punitive) damages.
\end{enumerate}

% TODO: defense of others -- you can't do any better than stepping into the shoes of the person threatened. Reasonable mistakes are not a defense [??]. Self defense includes defending members of family and workplace (not defense of others). Little authority on mistakes in defense of others. -- p. 65

\subsubsection{Defense of Property}

\begin{enumerate}
    \item Reasonable force can be used to protect property. Force intended to cause death or serious injury (i.e., wounding force) to protect property is \textbf{never reasonable}. % TODO civil code 50 'any necessary force' -- oddly would allow use of wounding force, if read literally -- but courts have yet to interpret it.
    \item The Second Restatement holds that reasonable force can be used when intrusion on property is not privileged, when the actor believes the intrusion can only be prevented by force, and when the owner first makes a request to desist (or believes a request will be useless).
    \item The person using force must give notice if feasible.
    \item \emph{Katko v. Briney}: spring guns protecting property are not reasonable unless the owner would have been privileged to use the same force if present.
\end{enumerate}

\subsubsection{Private Necessity}

% TODO prevent disproportionately greater damage to property -- and see hypos challenging this

% TODO can you do it on behalf of someone else?

``Private necessity is a privilege which allows the defendant to interfere with the property interests of an innocent party in an effort to avoid a greater injury. The privilege is incomplete since the actor must still compensate the victim for the property.''\footnote{Casebook pp. 69--70.}

\begin{enumerate}
    \item \textbf{\emph{Vincent v. Lake Erie Transp. Co}}: Defendant was moored at the plaintiff's dock to unload goods when a severe storm struck. Defendant kept his boat secured (and repeatedly replaced damaged or broken lines) to the dock throughout the storm, causing \$500 in damages to the dock. The court held that an actor is justified in using another's property in extreme circumstances, but will be held responsible for any damages incurred.
    \item The dissent analyzed this as a contracts problem, not a torts problem.
\end{enumerate}

\subsubsection{Public Necessity}

Public necessity allows appropriation of property in order to prevent a greater public harm. Compensation to the property owner is not required. In \emph{Surocco v. Geary}, the city San Francisco ordered the destruction of a building to create a gap to prevent the spread of a citywide fire. The owner unsuccessfully claimed he should have been allowed to remove his wine cellar before the building was destroyed. Not all courts, however, hold that public necessity can insulate municipalities from damages in all cases.

\subsection{Intentional Interference with Contractual and Economic Relations}

% TODO: interference with contract vs. prospective advantage; california treats them as separate torts
% "intentional and unjustified third-party interference with valid contractual relations or business expectations."
% best CA definition: "one who without a privilege to do so induces [] or causes a third party to ... (a), (b), (c)"
% another def (for intf with contacts):  (1)
% plaintiff has burden of proof. intf may be justified. burden of proof for justification is also on plaintiff; plaintiff must prove act was not justified -- justifications include: fair competition, protecting financial interest, and cases with fre speech ramifications.

\begin{enumerate}
    \item Economic torts: we want competition, but not too much.
    \item Contractual interference torts are rooted in anti-labor motivations.
    \item According to the Second Restatement, the elements of these two torts are:
    \begin{enumerate}
        \item A valid contract or economic expectancy.
        \item Defendant's knowledge of the contract or economic expectancy.
        \item Defendant's intent to interfere.
        \item Interference.
        \item Damage to the plaintiff.\footnote{Casebook p. 82 n. 1.}
    \end{enumerate}
    \item Many courts recognize various justifications:
    \begin{enumerate}
        \item Fair competition or proper protection of one's own financial interest (as long as the contract is freely terminable at will).
        \item Protecting the welfare of another for whom one is responsible.
        \item Providing truthful or honest information if requested.
        \item Assertion of a bona fide property right (e.g., preventing a thief from selling your car).
        \item Interfering with an agreement that is illegal or against public policy.
    \end{enumerate}
    \item Not all courts treat these as distinct torts, though California does.
    \item For non-legal reasons (e.g., public relations), these torts are rarely brought (do you really want to sue several people for leaving your firm?).
    \item \textbf{\emph{Calbom v. Knudtzon}}: Mr. Henderson died and left Mrs. Henderson to execute his estate. Harry Calbom, a lawyer, had been hired to help sort out the legal issues. Mrs. Henderson's accountant, Mr. Knudtzon, told Mrs. Henderson that Calbom was unsatisfactory and provided a list of other attorneys. Mrs. Henderson found another attorney, and Calbom sued for intentional interference with his employment contract. The court held that an attorney-client relationship existed, which Calbom had every right to expect would continue. It found that the ``defendants' interference was malicious, intentional, and without justification,'' affirming the judgment for Calbom.
    \begin{enumerate}
        \item Levy: this case is wrong. Knudtzon gave multiple suggestions for other attorneys, and there is no evidence of favoritism or kickbacks.
    \end{enumerate}

    \item \textbf{\emph{Lowell v. Mother's Cake \& Cookie Co. }}: TODO (85--90) % see paper notes
    \item \textbf{\emph{Texaco, Inc. v. Penzoil, Co.}}: TODO (90-97)
    \item \textbf{\emph{Environmental Planning and Information Council of Western El Dorado County, Inc., Superior Court}}: TODO (98--104)
\end{enumerate}

\subsection{Wrongful Termination of Employee}

\begin{enumerate}
    \item An employer can be liable for wrongful termination if the termination contradicts significant public policy.
    % TODO unless otherwise indicated explicitly or implicitly, the employment contract is at-will.
    % TODO breach will lead to contract damages (< tort damages) -- leading plaintiff's lawyers to develop the tort (aka wrongufl discharge). employer can be liable for emo distress, and punitive damages (in addition to contractual breach damages)
    \item Can an employee be fired for protecting the interests of his employer? \textbf{\emph{Foley v. Interactive Data Corp.}}: A well-regarded employee, Foley, became concerned when he learned that the person hired to be a new Vice President was under FBI investigation for embezzlement from Bank of America, his previous employer. Foley was fired within a few days. The court found that there was no wrongful termination because Foley's disclosure benefited only the private interests of his employer, not the public. % TODO ie, termination has to be in contravention of public policy. _Tameny_ was a breach of a statute. _Foley_: need not decide if breach of a non-statutory source (common law public policy) can be sufficient. tort is available only for breach of a constitutionally or statutorily public policy -- gant v sentry 1 Cal 4th 1083.
\end{enumerate}

\subsection{Tortious Breach of the Covenant of Good Faith and Fair Dealing}

\begin{enumerate}
    \item Every contract imposes a duty of good faith and fair dealing. Some courts hold that a breach of this covenant constitutes a tort, allowing tort damages (e.g., punitive damages and compensation for mental distress) as well as contract remedies.
% TODO    \item CA was the first state to develop the tort, adn developed in the context of bad faith insruacne breach. then in siemens, allowed to extend to others. but after 1986 political switch in court, then freeman & mills v belcher oil, ct says bad faith is only lalowed in insrnc cases. other states: allowed for unqeual bargaining power or fiduciary duties.
    \item What duties do insurance companies have to policyholders? \textbf{\emph{Egan v. Mutual of Omaha Insurance Co.}}: The plaintiff purchased a disability insurance policy from the defendant. When the plaintiff became disabled, the insurance company wrongly and maliciously withheld payments, calling the plaintiff a ``fraud.'' The court found that the insurer ``cannot reasonably and in good faith deny payments to its insured without thoroughly investigating the foundation for its denial.'' The court found for the plaintiff (but deemed the punitive damages of \$5 million to be excessive).
    \item Two types of insurance: first party (which bets on whether an event will happen---disability, life, etc.) and third party (which addresses incidents involving third parties--liability, homeowners, auto, etc.).
    \item Many courts have limited this tort to insurance contexts. However, as many as 16 of the 36 states that recognize the tort have applied it to non-insurance contexts.\footnote{Casebook p. 115.}
\end{enumerate}

\subsection{Intentional Misrepresentation}

\begin{enumerate}
    \item Second Restatement definition: ``One who fraudulently makes a [material] misrepresentation of fact, opinion, intention or law for the purpose of inducing another to act or to refrain from action in reliance upon it, is subject to liability to the other in deceit for pecuniary loss caused to him by his justifiable reliance upon the misrepresentation.''\footnote{Casebook p. 121.}
    \item Misrepresentation must be intentional or reckless. It usually has to be a statement of fact, not opinion, except in the case of a fiduciary. 
    % TODO tort of concealment or nondisclosure: one form of the tort of deceit. cal civ code 17 10 sub 3: suppression of a fact by one who is boudn to disclose it, or who gives misleading facts for want of comm'n of that fact, are actionable. // nondisclosure of a fact is actionable if the undisclosed fact is material and the defendant has a duty to disclose it.
    \item \textbf{\emph{Nader v. Allegheny Airlines, Inc.}}: Allegheny Airlines bumped Nader from a flight, causing him to miss a speaking engagement. The airline intentionally overbooked the flight, but told all passengers that they had ``confirmed reservations.'' Allegheny argued that ``confirmed'' was reasonable language because the probability of being bumped was very low. The court held, however, that the airline's nondisclosure was misleading. It awarded \$10 in compensatory damages to Nader and \$15,000 in punitive damages. The court of appeal reversed the decision, arguing that Nader's reliance was not justifiable because he had been bumped many times before and knew about the airline's policy.
    \item To succesfully recover, a plaintiff must have authentically relied on the misrepresentation.
    \item Courts have traditionally not included broken promises within this tort (though they may constitute breach of contract). However, some courts and the Second Restatement have begun to distinguish between promises that are lies (which are tortious) and sincere promises.\footnote{Casebook p. 124.}
    \item Failure to disclose can constitute concealment.
\end{enumerate}


% TODO civil code 32.4 -- when punitivie damages can be assessed
