\subsection{Loss of Consortium, Wrongful Birth, Wrongful Life}

\begin{enumerate}
    \item ``\textbf{Loss of consortium} actions compensate the plaintiff for 
    the loss of society and companionship suffered when another person is 
    injured or the relationship is otherwise tortiously 
    disrupted.''\footnote{Casebook p. 364.}
    \begin{enumerate}
        \item Generally, only spouses (and in California, domestic partners) 
        can recover.
        \item Recoverable harms include loss of sexual function, 
        companionship, household services, etc. Courts have upheld recovery 
        for misdiagnosis of syphilis which led a spouse to believe the other 
        was having an affair.\footnote{Casebook p. 365.}
        \item Some states allow for recovery for loss of consortium after 
        death, not just disability. % TODO: p. 366. What about CA?
    \end{enumerate}
    \item \textbf{Wrongful life} allows a child to recover for having been 
    born under certain conditions (but not for the condition itself). 
    California limits recovery to expenses related to the disability (see 
    \emph{Turpin} below). Most states, however, do not allow wrongful life 
    actions.\footnote{Casebook p. 373--74.} Courts usually allow recovery for 
    children who were negligently injured while fetuses.
    \item \textbf{Wrongful conception} allows parents to recover for negligent 
    conception of an unwanted but healthy child. Most jurisdictions allow it.
    \item \textbf{Wrongful birth} allows parents to recover for negligently 
    causing the birth of a child with a health disability (but not for the 
    disability itself). Many courts allow recovery for the costs of the 
    pregnancy and of raising a disabled child.
\end{enumerate}

\subsubsection{Loss of Consortium: \emph{Borer v. American 
Airlines, Inc.}}

Only spouses (and in California, domestic partners) can recover for loss of 
consortium. Children cannot recover for the loss of consortium of parents.

\begin{enumerate}
    \item The plaintiffs' mother was disabled when a roof collapsed in an 
    airline terminal.
    \item In \emph{Rodriguez}, the California Supreme Court held that loss of 
    consortium covered ``loss of love, companionship, society, sexual 
    relationships, and household services'' between spouses.\footnote{Casebook 
    p. 360.} The question before the court in this case was whether children 
    can recover for loss of consortium from their parents.
    \item The court held that loss of consortium should not extend to the 
    parent-child relationship because (1) monetary compensation is 
    ``essentially unrelated'' to theloss of ``maternal guidance'' and (2) loss 
    of parental consortium is ``very difficult to measure.''\footnote{Casebook 
    p. 361--62.}
    \item In parent-child relationships, loss of consortium is distinct from 
    wrongful death in two ways: (1) without a wrongful death cause of action, 
    it was rational for the defendant to kill the victim rather than inflict 
    injury, and (2) the consequences of a parent's disability can be relieved 
    through the parent's own cause of action.
\end{enumerate}

\subsubsection{Wrongful Life: \emph{Turpin v. Sortini}}

A child born with a hereditary condition resulting from a doctor's negligence 
cannot recover general damages, but she can recover special damages for 
expenses relating to her condition.

\begin{enumerate}
    \item The Turpins brought their first child to have Dr. Sortini examine 
    her for a possible hearing defect. Sortini told them her hearing was 
    within normal limits when, in fact, she was completely deaf. Relying on 
    that diagnosis, the Turpins had another child before learning that their 
    first child was completely deaf and that the deafness was a hereditary 
    condition. Their second child was also born completely deaf.
    \item The Turpins sued on behalf of their second daughter to recover (1) 
    general damages for the right to be born without deafness and (2) special 
    damages for medical expenses, teaching, etc.
    \item The question before the California Supreme Court was whether a child 
    born with a hereditary condition can recover from a doctor who negligently 
    failed to advise the parents of the possibility of the condition.
    \item The court held that the daughter could not recover for general 
    damages, but she ``may recover special damages for extraordinary expenses 
    necessary to treat the hereditary ailment.''\footnote{Casebook p. 373.}
\end{enumerate}
