\section{Workers' Compensation}

\begin{enumerate}
    \item Workers' compensation allows employees to recover damages for 
    injuries without needing to prove negligence. The full economic loss is 
    generally not recovered, and compensatory damages for intangibles (e.g., 
    pain and suffering) and punitive damages are generally 
    excluded.\footnote{Casebook p. 661--62.}
    \item The injury must have occured within the scope of employment.
    \item \emph{Fellow servant rule}: under common law, contributory 
    negligence of a co-employee is attributed to the injured employee, making 
    it more difficult for the injured employee to sue his employer under 
    negligence.\footnote{Casebook p. 661.}
\end{enumerate}

\subsection{Immobility Requirement: \emph{Bletter v. Harcourt, Brace \& World, 
Inc.}}

You're allowed to dance on the job.

\begin{enumerate}
    \item A high school textbooks editor was feeling good. He ``attempted to 
    do a dance step but fell and fractured his thigh.''\footnote{Casebook p.  
    657.}
    \item The workers' compensation board ``finds that claimant's casual 
    indulgence in a little dance step on the employer's premises and while in 
    a swiftly moving elevator, was not an unreasonable activity in view of his 
    feeling of well-being created by his liking for both the job and his 
    co-workers, so as to be deemed a deviation from the employment.''
    \item The court agreed with the board that employees are ``not required to 
    remain immobile.''
    \item Many jurisdictions have an exception for intentional torts. Courts 
    are divided on whether the threshold is desire or substantial 
    certainty.\footnote{Casebook pp. 674--75.}
\end{enumerate}

\subsection{Off Duty Employees: \emph{Ralphs Grocery v. Workers' Comp. Appeals Bd.}}

Employees are not within the scope of employment when they are off duty.

\begin{enumerate}
    \item Moeller was on disability leave for a finger injury. He also had 
    congenital heart disease. His employer, Ralphs Grocery, laid him off while 
    he was on disability leave, but he was scheduled to return several months 
    later. The night before he wsa scheduled to return, Ralphs called to tell 
    him that his position had been reduced to part time without benefits. 
    Moeller immediately suffered a fatal heart attack.
    \item The Workers' Compensation Judge found that the call caused Moeller's death 
    and that the call arose in the course of employment, awarding damages to 
    Moeller's widow.
    \item The appellate court held that employer-employee relationship is 
    severed while the employee is off duty. Moeller was off duty when he 
    received the call from Ralphs, and therefore he was outside the scope of 
    employment. Reversed.
\end{enumerate}

\subsection{Special Risk Exception: \emph{Johnson v. Stratlaw, Inc.}}

A plaintiff cannot bring a negligence cause of action if the employee was 
within the scope of employment when the harm occurred. The special risk 
exception to the going and coming rule puts an employee within the scope of 
employment.

\begin{enumerate}
    % TODO: replace AM/PM with a.m./p.m. throughout.
    \item Daryl worked at Stratlw's Straw Hat Pizza Parlor. One evening, Straw 
    Hat required Daryl to work from 5 p.m. to 2 a.m. Daryl died of a car 
    accident while driving home.
    \item Daryl's family sued for wrongful death and negligent infliction of 
    emotional distress. Stratlaw demurred that workers' compensation barred 
    negligence actions and that the NIED claim was invalid because the 
    plaintiffs had not witnessed the accident.
    \item The trial court sustained the NIED demurrer but overruled on all 
    other grounds. The trial later granted summary judgment on all counts for 
    the defendants.
    \item On appeal, plaintiffs argued that Daryl was not within the scope of 
    employment when the accident occurred.
    \item The appellate court held that the going and coming rule generally 
    puts the employee outside the scope of employment while commuting. 
    However, the \textbf{special risk exception} applies ``if (1) `but for' 
    the employment the employee would not have been at the location where the 
    injury occurred and (2) if `the risk is distinctive in nature and 
    qualitatively greater than risks common to the 
    public.'''\footnote{Casebook p. 666.}
    \item The court held that the special risk exception applied here. 
    Affirmed.
\end{enumerate}

\subsection{Intentional Torts: \emph{Fermino v. Fedco, Inc.}}

In California, workers' compensation does not bar employees' claims against 
employers for intentional torts.

\begin{enumerate}
    \item Fermino worked at Fedco's department store. Fedco accused her of 
    stealing and interrogated her in a room for more than an hour. Fermino 
    sued for false imprisonment and negligent and intentional infliction of 
    emotional distress.
    \item Fedco demurred that workers' compensation barred the claims. The 
    trial court sustained and the appellate court affirmed.
    \item The California Supreme Court held that ``the basis for the 
    exclusivity rule in workers' compensation law is the `presumed 
    compensation bargain, pursuant to which the employer assumes liability for 
    industrial and personal injury or death without regard to fault in 
    exchange for limitations on the amount of that 
    liability.''\footnote{Casebook p. 670.} It is possible for an employer to 
    step out of its proper role.
    \item Fedco's actions constituted false imprisonment. The question was 
    whether such behavior goes beyond the ``compensation bargain'' (which 
    allows for ``reasonable interrogation and detention''\footnote{Casebook 
    p. 672}). Tne court found that it did.
    \item Reversed.
\end{enumerate}
