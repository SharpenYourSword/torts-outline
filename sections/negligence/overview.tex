\subsection{Overview}

\begin{enumerate}
    \item Negligence is ``the failure to exercise the standard of care that a 
    \textbf{reasonably prudent [careful] person} would have exercised in a 
    similar situation.''\footnote{Black's Law.}
    \item Negligence requires proof that the defendant acted unreasonably.
    \item The standard of care is objective.
    \item Negligence has \textbf{five factors}:
    \begin{enumerate}
        \item Duty.
        \item Breach of duty.
        \begin{enumerate}
            \item Breach of the standard of care.
            \item Failure to act as a reasonably careful person would under 
            the circumstances.
        \end{enumerate}
        \item Cause-in-fact.
        \item Proximate cause.
        \item Damages.
    \end{enumerate}
    \item Tort law generally doesn't lower the standard of care for people who 
    are unable to meet it.
    \item The \textbf{Hand Formula} is Judge Learned Hand's test for determining 
    negligence. It works best in scenarios where the actor takes a calculated 
    risk (e.g., business decisions). It is less useful in cases where the 
    actor was simply not paying attention.
    \begin{enumerate}
        \item B = Burden of precautions necessary to prevent an accident.
        \item P = Probability that an accident will occur.
        \item L = Magnitude of the loss if the accident occurs.
        \item \textbf{Negligence exists if B\textless PL}---i.e., if the burden of 
        precautions is less than the harm multiplied by the probability of 
        occurrence.
    \end{enumerate}
    \item Prosser: compare the utility of the risk with the gravity of the 
    loss.
\end{enumerate}

\subsubsection{Foreseeable and Unreasonable: \emph{Pitre v. Employers 
Liability Assurance Corporation}}

\textbf{Negligence occurs only if the danger is both foreseeable and 
unreasonable.} 

\begin{enumerate}
    \item The plaintiffs' son died when a patron at a carnival game was 
    winding up a pitch and hit him in the head.
    \item The trial court found in favor of the plaintiffs.
    \item The appellate court held that the key factor was how a 
    ``reasonably prudent individual'' would have acted or what precautions he 
    would have taken under similar circumstances. The court held 
    that the danger was foreseeable but not unreasonable, and therefore there 
    was no negligence.
\end{enumerate}

\subsubsection{Applying the Hand Formula: \emph{United States Fidelity \& 
Guaranty Co. v. Plovidba}}

If the probability of the accident occurring is very small, the court will 
likely not find the defendant to be negligent.

\begin{enumerate}
    \item Inside a dark room, the hatch to a cargo hold on a ship was left 
    open. A longshoreman fell through it and died.
    \item The trial court found for the defendant.
    \item Richard Posner, writing for the Seventh Circuit, applied the Hand 
    Formula, reasoning that B was relatively small (it would have been easy to 
    close the hatch or leave a light turned on) and L was high (the victim 
    died). P, however, was very small. There was no reason for the 
    longshoreman to enter the hold. In fact, he was probably there to steal 
    liquor, and the evidence suggests he knew the hatch was open and tried to 
    skirt around it. The shipowner was therefore not negligent.
\end{enumerate}
