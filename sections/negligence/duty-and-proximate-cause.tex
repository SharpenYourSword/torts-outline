\subsection{Duty and Proximate Cause}

\begin{enumerate}
    \item Most courts speak about duty and proximate cause as separate 
    elements. However, you could probably build a torts system with just one 
    or the other.
    \item \emph{Palsgraf}: four justices deal with it as a duty question, and 
    three in dissent view it as a proximate cause problem.
    \item Proximate cause: answers the question of whether there should be 
    liability when the defendant's negligence was a cause in fact of the harm. 
    It is akin to duty, where we ask whether the defendant should be immunized 
    from liability.
    \item Two views of proximate cause:
    \begin{enumerate}
        \item 1. Rigorous analytical meaning: scope of the risk analysis. 
        There are fact situations where we want to limit liability because the 
        actual harm was not one of the foreseeable harms that made us deem the 
        act to be negligence.
        \item 2. (Levy's preference.) There are certain fact situations where 
        even though the defendant was negligent, and it caused harm, we choose 
        for policy reasons to have no liability. Courts can conclude that 
        defendant was under no duty; in other cases, courts find that a 
        defendant's conduct was not a proximate cause.
    \end{enumerate}
    \item Intervening superseding events break the chain of causation. 
    Dependent or naturally occurring events do not.
    \item Cardozo: ``Danger invites rescue.''
\end{enumerate}

\subsubsection{Common Sense: \emph{Atlantic Coast Line R. Co. v. Daniels}}

\begin{enumerate}
    \item Cause and effect are infinite. An act is the proximate cause if it's 
    close enough. Courts and juries have to rely on reason and common sense to 
    judge whether a cause is proximate.
    \item Some sources, like the Restatement on Torts, prefer ``legal cause.''
    \item Proximate cause is a tool for protecting defendants.
\end{enumerate}

\subsubsection{\emph{Palsgraf v. The Long Island Railroad Company}}

Duty and proximate cause are two competing frameworks for examining negligence 
problems.

\begin{enumerate}
    \item A railroad employee caused a passenger's package to fall. The 
    package turned out to be full of fireworks. It exploded, causing a scale 
    to break and injure the plaintiff.
    \item The trial court found negligence. The Court of Appeals here 
    reversed.
    \item Cardozo employs a \textbf{duty} framework. Negligence requires the 
    defendant to have a duty to the plaintiff. There must be a point in the 
    chain of causation where an actor is no longer liable---otherwise, anybody 
    who jostles someone in a crowd could be liable. To be negligent, the actor 
    must have breached a reasonable standard of care. In this case, however, 
    the railroad employee could not have known that the package was full of 
    fireworks.
    \item Andrews, dissenting, employs a \textbf{proximate cause} framework. 
    The actor owes a duty of care to the public at large. Ultimately, 
    proximate cause is about expediency, not logic, and judges must rely on 
    common sense. In this case, the defendant's actions were a but-for cause 
    of the plaintiff's injuries. It's not possible to say that plaintiff's 
    injuries ``were not the proximate result of the negligence.''
\end{enumerate}

\subsubsection{Directness vs. Foreseeability: \emph{Overseas Tankship (U.K.) 
Ltd. v. Morts Dock \& Engineering Co. (The Wagon Mound) Privy Council}}

Defendants are not liable if the harm was not a \textbf{foreseeable 
consequence} of their negligence.

\begin{enumerate}
    \item The plaintiffs' ship, the \emph{Corrimel}, was moored for repairs. 
    The appellants' ship, the \emph{Wagon Mound}, was moored nearby. The crew 
    of the \emph{Wagon Mound} accidentally spilled a large amount of oil into 
    the bay. They left soon after without cleaning up the oil.
    \item The plaintiff checked with the manager of the wharf where the 
    \emph{Wagon Mound} was moored to see if the oil on the water was 
    flammable. They agreed it was not. Soon after, a small drop of molten 
    metal from the plaintiffs' worked ignited the oil, severely damaging the 
    \emph{Corrimal} and the wharf.
    \item \emph{In re Polemis} dealt with another scenario involving fire and 
    negligence. Although the fire was not a foreseeable consequence of the 
    negligence, it was clear that the defendant's action was the direct cause, 
    and the court held for the plaintiffs.
    \item The court here replaced the direct test from \emph{Polemis} with a 
    foreseeability test.
    \item The defendants could not have foreseen a massive fire to be the 
    result of their negligence. Ruling for the defendants.
    \item \emph{Kinsman}: foreseeability is a weaker requirement when the 
    consequences are direct and the damage is of the same sort that was 
    risked.\footnote{Casebook p. 258.}
\end{enumerate}

\subsubsection{Intervening Events: \emph{Thomas v. United States Soccer 
Fedn.}}

Superseding intervening events break the chain of causation.

\begin{enumerate}
    \item The plaintiff suffered injuries when a soccer game turned violent. 
    He sued the soccer federation for failing to provide a properly trained 
    referee and failing to maintain a safe playing environment. The defendants 
    moved for a summary judgment on the grounds that the alleged negligence 
    was not the proximate cause.
    \item The court held that when an intervening act occurs, liability will 
    turn on whether the defendant should have foreseen the act as a 
    consequence of the negligence. It reversed the lower courts and granted 
    the motion for dismissal.
    \item ``Superseding intervening forces are those new forces which are 
    extraordinarily unexpected.''\footnote{Casebook p. 261.}
    \item Intervening criminal acts are generally found to be unforeseeable 
    and therefore superseding.
    \item ``Dependent'' intervening forces are results of the defendant's 
    action (e.g., an ambulance driver's collision while rushing to the scene 
    of the defendant's accident). ``Independent'' intervening forces do not 
    have a causal connection to the defendant (e.g., a lightning bolt).
    \item ``...ultimately the determinative issue is whether or not the 
    intervening force is extraordinarily unexpected.''\footnote{Casebook p. 
    263.}
\end{enumerate}

\subsubsection{Uncertainty of Foreseeability\emph{Bigbee v. Pacific Telephone and Telegraph Co.}}

Whether a harm is foreseeable is often a jury question.

\begin{enumerate}
    \item Plaintiff was inside a telephone booth. He saw a car approaching, 
    and he claimed he tried to get out but couldn't. He alleged the telephone 
    booth company was negligent in (1) its manufacture of the booth, which 
    prevented his escape, and (2) its placement in proximity to a busy street, 
    where damage from an oncoming car was foreseeable.
    \item The lower courts granted and upheld a motion to dismiss.
    \item Here, the Supreme Court of California held that a 
    jury could find that the danger was reasonably foreseeable.
    \item Unlikely intervening events are often not found to be superseding 
    events. For instance, if an owner leaves the keys in her car in a high 
    crime area, she may be liable for the harm the car thief causes. (But 
    generally, car owners are not responsible for the actions of car thieves.)
\end{enumerate}

\subsubsection{The Egg-Shell Plaintiff Rule: \emph{Steinhauser v. Hertz 
Corporation}}

Extra-sensitive plaintiffs can recover full damages.

\begin{enumerate}
    \item The plaintiff was involved in a car accident. She suffered no 
    injuries, but the accident triggered serious schizophrenia.
    \item The court held that as long as there is a causal relationship 
    between the small accident and the catastrophic result, the defendant can 
    be held liable for the ``precipitating cause.'' The probability that the 
    condition would have developed is not a defense, but it can be considered 
    in fixing damages.
    \item The large injury from the small cause need not be foreseeable.
\end{enumerate}
