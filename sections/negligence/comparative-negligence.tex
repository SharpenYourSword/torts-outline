\subsection{Cause in Fact}

\begin{enumerate}
    \item Plaintiff must show that the defendant's negligence was a cause in 
    fact of the harm.
    \item Traditionally: plaintiff must prove that the harm would not have 
    occurred \textbf{but for} the defendant's actions.
    \item If there are multiple causes of harm, each can be but-for causes as 
    long as the harm would not have occurred without it.
    \item If there are multiple causes of harm, but none alone is a but-for 
    cause, courts can use the \textbf{substantial factor test}. See 
    \emph{Northington} below.
    \item Substantial factor test in the Second Restatement:
    \begin{enumerate}
        \item % todo: add
    \end{enumerate}
    \item The biggest different between but-for test and substantial factor 
    test is that under the substantial factor test, it's much more likely to 
    go to a jury.
    \item \textbf{Proximate cause} removes liability when ``the connection 
    between the plaintiff's harm and defendant's liability is unforeseeable or 
    so attenuated that public policy precludes liability.''\footnote{Casebook 
    p. 206.}
    \item When two people are ``acting in concert'' (i.e., trying to do the 
    same thing), and one is the negligent actor, the court can hold both 
    parties liable.
    \item \emph{Summers v. Tice}: CA Supreme Court adopted \textbf{alternative 
    liability test}: when one of two negligent defendants probably caused a 
    harm, and it has not been shown that it is more likely than not that 
    either caused it, then each will be held jointly and severally liable for 
    the full amount of the harm.
    \begin{enumerate}
        \item Restatement Second says the test applies when there are ``two or 
        more'' defendants.
        \item Also think about the effect of Prop 51 on \emph{Summers v. 
        Tice}: not clear whether it applies to these joint tortfeasor cases or 
        not.
    \end{enumerate}
    \item \emph{Sindell}, \textbf{market share liability}: each defendant 
    shall be held liable for the proportion of the judgment represented by its 
    share of the market unless it can demonstrate that it did not manufacture 
    the product that caused the plaintiffs' injuries. This is the case in CA, 
    but not in NY.
    \item Toxic torts: what to do when there is no harm, but only enhanced 
    risk? One approach: award damages, but to a lesser amount, based on the 
    percent chance of the harm. Seond approach (in CA): will not award general 
    damages, but will award damages for medical surveillance.
\end{enumerate}

\subsubsection{\emph{East Texas Theatres, Inc. v. Rutledge}}

\begin{enumerate}
    \item At the defendant's movie theater, somebody threw a bottle from a 
    balcony which struck and injured the plaintiff. The jury found the theater 
    liable because it negligently failed to remove ``rowdy persons'' from the 
    balcony during the game, and the Texas appellate court affirmed. The Texas 
    Supreme Court clarified that proximate cause has two elements: (1) 
    cause-in-fact and (2) foreseeability. The court held that the prosecution 
    failed to show that the injuries would have occurred but for the removal 
    of the ``rowdy persons.'' It reversed the lower court's ruling and held 
    for the defendants.
\end{enumerate}

\subsubsection{\emph{Anderson v. Minneapolis, St. P. \& S. S. M. Ry. Co.}}

\begin{enumerate}
    \item A spark from a railroad started a fire in a bog on one side of the 
    defendant's property. Another unrelated fire was burning on the other 
    side. The fire from the railroad destroyed the defendant's property, and a 
    few days later it joined with the other fire to make one big fire. The 
    railroad argues that it cannot be held liable because the defendant's 
    house would have been destroyed by the other fire anyway. The trial court 
    refused to instruct the jury to follow a rule from an earlier case, 
    \emph{Cook}, which held that there is no liability when two fires jointly 
    destroy property. On this basis, the trial court found for the plaintiff. 
    The railroad requested a motion for judgment notwithstanding the verdict, 
    which was denied. On appeal, the Supreme Court of Minnesota held that the 
    trial court was correct in refusing to apply the \emph{Cook} rule and 
    found for the plaintiffs.
    \item \textbf{Substantial factor test}: If two independent fires join to 
    cause property damage, there is joint liability, even if neither alone is 
    a but-for cause. Redundant causation is not necessary.
    \item Courts split on whether to use the substantial factor test when only 
    one actor is liable. California courts do use it (and reject the but-for 
    test).
\end{enumerate}

\subsubsection{\emph{Northington v. Marin}}

\begin{enumerate}
    \item The plaintiff, a prison inmate, sued the defendant, a prison guard, 
    for circulating rumors that labeled him a snitch and caused other inmates 
    to assault him. Other guards had spread the same rumors. The trial court 
    found that although the defendant's action was not a but-for cause (since 
    the harm would have occurred without his action), his contribution to the 
    harm was nonetheless a \textbf{substantial factor}. The Tenth Circuit 
    affirmed: ``Multiple tortfeasors who concurrently cause an indivisible 
    injury are jointly and severally liable; each can be held liable for the 
    entire injury.''
\end{enumerate}

\subsubsection{\emph{Herskovitz v. Group Health Cooperative of Puget Sound}}

\begin{enumerate}
    \item The plaintiff brought the action on behalf of her husband, a 
    deceased lung cancer patient, against a doctor that negligently failed to 
    diagnose the patient's lung cancer on his first visit, proximately causing 
    his chance of survival to drop from 39 percent to 25 percent. Neither fact 
    was in dispute. The defendant argued that the plaintiff must prove that 
    the patient ``probably'' would have lived but for the negligence---that 
    is, without the doctor's negligence, the patient's chance of survival must 
    have been more than 50 percent. The trial court granted summary judgment 
    for the defendant on this argument. The Supreme Court of Washington 
    reversed, arguing that any other decision would mean a ``blanket release'' 
    for doctors' negligence any time the patient's chance of survival was less 
    than 50 percent. The court reasoned that if a defendant's acts have 
    \emph{increased the risk} of harm to the plaintiff, a jury should decide 
    whether the increased risk actually caused the harm in question.
\end{enumerate}

\subsubsection{\emph{Summers v. Tice}}

\begin{enumerate}
    \item The \emph{Summers} rule applies where there are a small number of 
    defendants, only one of them committed the harm, and we don't know which 
    one.
    \item The plaintiff and the two defendants were hunting quail. The two 
    defendants shot at a quail in the direction of the plaintiff. The 
    plaintiff suffered injuries, but it's not clear which defendant's shot was 
    the cause. The court reasons that in this case, the burden of proof shifts 
    to the defendants to determine which one of them caused the injury. If 
    they cannot, ``each defendant is liable for the whole damage whether they 
    are deemed to be acting in concert or independently.'' The lower courts 
    found the defendants liable and the Supreme Court of California affirmed.
    \item Can you hold three defendants liable under the \emph{Summers} test?
    \item Another case with joint tortfeasors, see \emph{Drabek v. Sabley} 
    above (kids throwing snowballs at cars).
\end{enumerate}

\subsubsection{\emph{Sindell v. Abbott Laboratories}}

\begin{enumerate}
    \item The plaintiff was harmed by DES, a prenatal drug intended to protect 
    against miscarriages but which turned out to pose significant danger to 
    unborn children. The plaintiff did not know which company manufactured the 
    specific drug her mother took, but since several companies manufactured 
    the drug according to the same formula, she sued them all. The companies 
    won a dismissal at trial on the grounds that the plaintiff could not 
    identify which company caused the harm.
    \item The Supreme Court of California considered four theories of 
    liability:
    \begin{enumerate}
        \item The \emph{Summers} test: this fails because there are so many 
        defendants (over 200) that it is highly unlikely that any one of them 
        caused this specific injury.
        \item The ``concert of action'' theory: if the defendants had acted in 
        concert to cause the injury, they would be equally liable. In this 
        case, there is not sufficient evidence to show that the defendants had 
        a common plan to cause harm (e.g., by conducting inadequate safety 
        tests or giving insufficient safety warnings).
        \item ``Industry-wide'' or ``enterprise'' liability: if an entire 
        industry cooperates on an element of the harm in question---e.g., by 
        delegating safety testing to a trade association---they can be held 
        jointly liable. Here, the fact that DES manufacturers shared testing 
        and promotion methods does not establish industry-wide liability, 
        because (1) there are so many manufacturers and (2) safety standards 
        are mostly regulated by the FDA.
        \item \textbf{Market share liability}---a variation of the 
        \emph{Summers} test: each manufacturer's liability and share of the 
        damages are proportionate to its market share.
    \end{enumerate}
    \item Relying on the fourth theory, the Supreme Court of California 
    reversed, allowing the plaintiff to proceed with her cause of action.
    \item Most states have not adopted market share liability.
    \item Defendants are allowed prove definitively that they did not 
    contribute to the harm (e.g., if they can show that they did not produce 
    the drug at the time).
    \item Some states require defendants to be joined so that a significant 
    share of the market is represented, and that missing market share 
    proportionally reduces the plaintiff's compensation. Usually (but not 
    always) this must be the nationwide market.\footnote{Casebook p. 229 n. 
    2.}
\end{enumerate}

\subsubsection{\emph{Ayers v. Township of Jackson}}

\begin{enumerate}
    \item A town in New Jersey was found to have caused toxic exposure by its 
    ``palpably unreasonable'' management of a landfill. Plaintiffs did not 
    develop any illnesses, but they sought to recover (1) damages for the 
    enhanced risk of future illness due to exposure and (2) regular medical 
    testing for diseases from exposure. The Supreme Court of New Jersey found 
    that he task of litigating hypothetical injuries would unreasonably strain 
    the tort system (although it suggests that the state legislature could 
    pass a remedy that allowed damages if toxic exposure caused a 
    ``statistically significant incidence of disease''). On the second claim, 
    it held for the plaintiffs.
\end{enumerate}

\subsection{Duty and Proximate Cause}

\begin{enumerate}
    \item Most courts speak about duty and proximate cause as separate 
    elements. However, you could porbably build a torts system with just one 
    or the other.
    \item Palsgraf: four justices deal with it as a duty question, and three 
    in dissent view it as a proximate cause problem.
    \item Proximate cause: whether there should be liability, even though teh 
    defendant's negligence actually caused the harm. Akin to duty, where we 
    ask whether the defendant should be immunized from duty.
    \item Two views of proximate cause:
    \begin{enumerate}
        \item 1. Rigorous analytical meaning: scope of the risk analysis. 
        There are fact situations where we want to limit liability because the 
        actual harm was not one of the foreseeable harms that made us deem the 
        act to be negligence.  \item (Levy's preference). 2. There are certain 
        fact situations where even though the defendant was negligent, and it 
        caused harm, we choose for policy reasons to have no liability. Courts 
        can conclude that defendant was under no duty; in other cases, courts 
        find that a defendant's conduct was not a proximate cause.
    \end{enumerate}
    \item Intervening superseding events vs. dependent/naturally occurring.
    \item ``Danger invites rescue.''
\end{enumerate}

\subsubsection{\emph{Atlantic Coast Line R. Co. v. Daniels}}

\begin{enumerate}
    \item Cause in effect are infinite. An act is the proximate cause if it's 
    close enough. Courts and juries have to rely on reason and common sense to 
    judge whether a cause is proximate.
    \item Some sources, like the Restatement on Torts, prefer ``legal cause.''
    \item Proximate cause is a tool for protecting defendants.
\end{enumerate}

\subsubsection{\emph{Palsgraf v. The Long Island Railroad Company}}

\begin{enumerate}
    \item A railroad employee caused a passenger's package to fall. The 
    package turned out to be full of fireworks. It exploded, causing a scale 
    to break and injure the plaintiff.
    \item The trial court found negligence. The Court of Appeals here 
    reversed.
    \item Cardozo: negligence requires the defendant to have a duty to the 
    plaintiff. There must be a point in the chain of causation where an actor 
    is no longer liable---otherwise, anybody who jostles someone in a crowd 
    could be liable. To be negligent, the actor must have breached a 
    reasonable standard of care. In this case, however, the railroad employee 
    could not have known that the package was full of fireworks.
    \item Andrews, dissenting: The actor owes a duty of care to the public at 
    large. Ultimately, proximate cause is about expediency, not logic, and 
    judges must rely on common sense. In this case, the defendant's actions 
    were a but-for cause of the plaintiff's injuries. It's not possible to say 
    that plaintiff's injuries ``were not the proximate result of the 
    negligence.''
\end{enumerate}

\subsubsection{Directness vs. Foreseeability: \emph{Overseas Tankship (U.K.) 
Ltd. v. Morts Dock \& Engineering Co. (The Wagon Mound) Privy Council}}

\begin{enumerate}
    \item The plaintiffs' ship, the \emph{Corrimel}, was moored for repairs. 
    The appellants' ship, the \emph{Wagon Mound}, was moored nearby. The crew 
    of the \emph{Wagon Mound} accidentally spilled a large amount of oil into 
    the bay. They left soon after without cleaning up the oil.
    \item The plaintiff checked with the manager of the wharf where the 
    \emph{Wagon Mound} was moored to see if the oil on the water was 
    flammable. They agreed it was not. Soon after, a small drop of molten 
    metal from the plaintiffs' worked ignited the oil, severely damaging the 
    \emph{Corrimal} and the wharf.
    \item \emph{In re Polemis} dealt with another scenario involving fire and 
    negligence. Although the fire was not a foreseeable consequence of the 
    negligence, it was clear that the defendant's action was the direct cause, 
    and the court held for the plaintiffs.
    \item The court here replaced the direct test from \emph{Polemis} with a 
    foreseeability test.
    \item The defendants could not have foreseen a massive fire to be the 
    result of their negligence. Ruling for the defendants.
    \item \emph{Kinsman}: foreseeability is a weaker requirement when the 
    consequences are direct and the damage is of the same sort that was 
    risked.\footnote{Casebook p. 258.}
\end{enumerate}

\subsubsection{Intervening Events: \emph{Thomas v. United States Soccer 
Fedn.}}

\begin{enumerate}
    \item The plaintiff suffered injuries when a soccer game turned violent. 
    He sued the soccer federation for failing to provide a properly trained 
    referee and failing to maintain a safe playing environment. The defendants 
    moved for a summary judgment on the grounds that the alleged negligence 
    was not the proximate cause. The court reasoned that when an intervening 
    act occurs, liability will turn on whether the defendant should have 
    foreseen the act as a consequence of the negligence. It reversed the lower 
    courts and granted the motion for dismissal.
    \item ``Superseding intervening forces are those new forces which are 
    extraordinarily unexpected.''\footnote{Casebook p. 261.}
    \item Intervening criminal acts are generally found to be unforeseeable 
    and therefore superseding.
    \item ``Dependent'' intervening forces are results of the defendant's 
    action (e.g., an ambulance driver's collision while rushing to the scene 
    of the defendant's accident). ``Independent'' intervening forces do not 
    have a causal connection to the defendant (e.g., a lightning bolt).
    \item ``...ultimately the determinative issue is whether or not the 
    intervening force is extraordinarily unexpected.''\footnote{Casebook p. 
    263.}
\end{enumerate}

\subsubsection{\emph{Bigbee v. Pacific Telephone and Telegraph Co.}}

\begin{enumerate}
    \item Plaintiff was inside a telephone booth. He saw a car approaching, 
    and he claims he tried to get out but couldn't. He alleges the telephone 
    booth company was negligent in (1) its manufacture of the booth, which 
    prevented his escape, and (2) its placement in proximity to a busy street, 
    where damage from an oncoming car was foreseeable. The lower courts upheld 
    a motion to dismiss. Here, the Supreme Court of California held that a 
    jury could find that the danger was reasonably foreseeable. Reversed and 
    remanded.
    \item Unlikely intervening events are often not found to be superseding 
    events. For instance, if an owner leaves the keys in her car in a high 
    crime area, she may be liable for the harm the car thief causes. (But 
    generally, car owners are not responsible for the actions of car thieves.)
\end{enumerate}

\subsubsection{The Egg-Shell Plaintiff Rule: \emph{Steinhauser v. Hertz 
Corporation}}

\begin{enumerate}
    \item The plaintiff was involved in a car accident. She suffered no 
    injuries, but the accident triggered serious schizophrenia. The court held 
    that as long as there is a causal relationship between the small accident 
    and the catastrophic result, the defendant can be held liable for the 
    ``precipitating cause.'' The probability that the condition would have 
    developed is not a defense, but it can be considered in fixing damages.
    \item The large injury from the small cause need not be foreseeable.
\end{enumerate}

\subsection{Proof of Negligence: Res Ipsa Loquitur}

\begin{enumerate}
    \item Res ipsa loquitur: ``the thing speaks for itself.''
    \item It usually has three requirements (with variations among 
    jurisdictions):
    \begin{enumerate}
        \item The accident would not have occurred without negligence.
        \item The negligent act was within the actor's control.
        \item The plaintiff was not at fault (i.e., no contributory 
        negligence).
    \end{enumerate}
    \item It's an expansion of the common sense cookie jar rule: if a parent 
    returns to see a child next to a broken cookie jar, it's reasonable to 
    infer that the child broke the cookie jar.
    \item We can generally assume that a car in motion that hits a pedestrian 
    was negligent---you don't need res ipsa loquitur to show negligence.
    \item If there is no evidence of res ipsa loquitur, whether the state is a 
    presumption state or an inference state. If it's a presumption state, the 
    plaintiff can receive a directed verdict; if it's merely an inference, the 
    jury is free to draw the inference or not.
    \item If the defendant presents evidence of due care, then in all 
    jurisdictions the question would go to a jury.
    \item If one of the three conditions is undercut, the jury is given the 
    ``conditional res ipsa'' instruction: if you find A, B, and C, there is a 
    presumption of negligence.
    \item Some courts have relaxed the requirement that the defendant must 
    have had exclusive control of the accident.\footnote{Casebook p. 275 n. 
    4.}
    \item Some courts follow the \emph{Ybarra} rule, which expands the res 
    ipsa loquitur doctrine to medical cases with multiple defendants, where 
    multiple defendants did not have exclusive control of the accident and not 
    all of them were necessarily negligent. It's an extension of the situation 
    where a teacher punishes the entire class for breaking the goldfish bowl.
\end{enumerate}

\subsubsection{\emph{Krebs v. Corrigan}}

\begin{enumerate}
    \item The defendant inexplicably flew through the air and landed on the 
    plaintiff's plexiglass sculpture, destroying it. The trial court granted 
    the defendant's motion for a directed verdict.
    \item ``...human bodies do not generally go crashing into breakable 
    personal property,'' said the appellate court.
    \item Defendant argued (1) that res ipsa loquitur does not apply when the 
    instrumentality is a human body and (2) the doctrine does not apply 
    because there was an eyewitness. The court rejected both of these 
    arguments.
    \item The doctrine exists, the court reasoned, to deal with cases where 
    only the defendant knows the details of the negligent act.
    \item The appellate court held that the evidence was sufficient to raise 
    an inference of negligence, so it reversed the directed verdict for the 
    defendants.
\end{enumerate}

\subsubsection{\emph{Ybarra v. Spangard}}

\begin{enumerate}
    \item (Levy: this case is better thought of as involving a causation 
    issue.)
    \item The plaintiff underwent surgery for appendicitis. During the 
    procedure, he suffered a shoulder injury that caused paralysis and muscle 
    atrophy. The trial court entered a judgment of nonsuit for all defendants.
    \item The plaintiff argued that the doctrine of res ipsa loquitur should 
    apply to the defendants, all of whom were involved at different stages of 
    his medical care.
    \item The defendants argue that the plaintiff cannot show that any single 
    defendant caused the injury.
    \item As in \emph{Krebs}, the court noted that the purpose of the res ipsa 
    loquitur doctrine is to address cases where the circumstances of the 
    negligence were unknown to the plaintiff (in this case, because he was 
    unconscious).
    \item Classic examples where res ipsa loquitur would apply: passenger 
    sitting awake in a train car at the time of a collision; person walking 
    down the street and hit by a falling object.
    \item These sorts of cases ``raise the inference of negligence, and call 
    upon the defendant to explain the unusual result.''\footnote{Casebook p. 
    279.}
    \item It could be found in this case that some of the defendants are 
    liable and others are absolved. But that should not preclude the 
    application of res ipsa loquitur. It would not be reasonable to ask the 
    plaintiff to identify which of the individual defendants were responsible 
    for the harm.
    \item The defendants' argument would undermine the rights of patients to 
    recover for injuries suffered while unconscious.
    \item Judgment of nonsuit was reversed.
\end{enumerate}

\subsection{Limitations on Duty}

% \item See Ca9l. Civ. Code 1714.
% \item No duty to affirmatively act, with a few exceptions:
% \begin{enumerate}
%     \item One who causes injury may have a duty to rescue.
%     \item Relationship between P and D may create a duty: common carrier, 
%     land occupiers, innkeeper, parent...
%     \item Beginning an undertaking that places the victim in a position that 
%     makes them less likely to be rescued can lead to liability.
%     \item Good samaritan statutes protect from liability.
% \end{enumerate}
% 
%\subsubsection{Failure to Act: \emph{L. S. Ayres \& Co. v. Hicks}}
%
%\begin{enumerate}
%    \item todo
%\end{enumerate}
%
%\subsubsection{\emph{Miller v. Arnal Corp.}}
%
%\begin{enumerate}
%    \item todo
%\end{enumerate}
%
%\subsubsection{\emph{Wells v. Hickman}}
%
%\begin{enumerate}
%    \item todo
%\end{enumerate}
%
%\subsubsection{\emph{Tarasoff v. The Regents of the University of 
%California}}
%
%\begin{enumerate}
%    \item todo
%    \item [Levy lecture: ]Relationship between defendant and third party can 
%    create a duty to a stranger.
%\end{enumerate}
%
%\subsubsection{\emph{Davidson v. City of Westminster}}
%
%\begin{enumerate}
%    \item todo
%\end{enumerate}
%

