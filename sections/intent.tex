\section{Intent}

\begin{enumerate}
    \item Intent requires desire or substantial certainty.
    \item Can a child meet the requirements for intent? \textbf{\emph{Garratt v. Dailey}}: A five year old moved a chair from the place where the plaintiff was about to sit. The plaintiff fell and fractured her hip. The plaintiff's battery claim requires proof that the defendant had intent to cause contact that was not consensual or otherwise privileged. The Second Restatement indicates that intent exists if the actor is \textbf{substantially certain} that the harmful contact will occur. Court finds that it's unclear whether the defendant had substantial certainty. Remanded to trial court for clarification.
    \item Can an insane person meet the requirements for intent? \textbf{\emph{Williams v. Kearbey}}: Minor shot up a school and claimed insanity. Court held that defendant intended to commit the action (even if his motivation was irrational) and is therefore liable.
    \item Other notes on torts:
    \begin{enumerate}
        \item Torts are generally excepted from workers' comp immunity.
        \item In most jurisdictions, you can't insure against intentional torts.
        \item The constitution's Supremacy Clause leads to three kinds of preemption of federal laws over state laws:
        \begin{enumerate}
            \item \emph{Express preemption}: Explicit or implicit overriding of a state statute.
            \item \emph{Conflict preemption}: In case of direct conflict, federal law preempts state law.
            \item \emph{Field preemption}: Congress legislates for an entire field of regulation, leaving no room for states to regulate.
        \end{enumerate}
    \end{enumerate}
\end{enumerate}
