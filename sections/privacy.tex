\section{Privacy}

\begin{enumerate}
    \item There are traditionally four privacy torts:
    \begin{enumerate}
        \item Intrusion upon seclusion.
        \item Unauthorized use of name or likeness.
        \item Giving unreasonable publicity to private matters.
        \item Publicly characterizing a party in a false light.
    \end{enumerate}
\end{enumerate}

\subsection{Intrusion upon Seclusion}

\begin{enumerate}
    \item The intrusion must be highly offensive to a reasonable person.
    \item There is no requirement of publication or communication.
    \item % todo media liability 753
\end{enumerate}

\subsubsection{\emph{Pearson v. Dodd}}

\begin{enumerate}
    \item Members of Senator Thomas Dodd's staff surreptitiously copied 
    documents from the Senator's office and sent them to reporters, Pearson 
    and Anderson.
    \item The distric court denied summary judgment for invasion of privacy.
    \item The Fifth Circuit affirmed, holding that the intrusion constitutes
    the tort. Publication is not one of the elements.
    \item Dissent: information obtained through these means would not be 
    admissible as evidence in court, but we allow it for news media. ``There 
    is an anomaly lurking in this situation: the news media regard themselves 
    as quasi-public institutions yet they deman immunity from the restraints 
    which they vigorously demand be placed on government.''\footnote{Casebook 
    p. 746.}
\end{enumerate}

\subsubsection{\emph{Dietemann v. Time, Inc.}}

\begin{enumerate}
    \item Employees of Time collaborated with the District Attorney's office 
    to fraudulently gain access to Dietemann's home. Dietemann was suspect of 
    practicing medicine without a license. They secretly recorded 
    conversations and events. When Dietemann was arrested, the reporters took 
    pictures, which Dietemann consented to only because he thought the police 
    officers required it. Time published the material in an article titled 
    ``Crackdown on Quackery.''
    \item The district court held that the pictures taken without Dietemann's 
    consent inside his home constituted an invasion of privacy.
    \item The Ninth Circuit affirmed, holding that Dietemann had an 
    expectation of privacy within his home. An opposite holding would chill 
    candid speech.
    \item On the free speech question: ``The First Amendment is not a license 
    to trespass, to steal, or to intrude by electronic means into the 
    precincts of another's home or office. It does not become such a license 
    simply because the person subjected to the intrusion is reasonably 
    suspected of committing a crime.''\footnote{Casebook p. 752.}
\end{enumerate}

\subsection{Appropriation of Name or Likeness and Publicity of Private Life}

% todo def 757
% todo beyond name and picture 758
% todo right of publicity
% publicity given false light 764
% todo private facts 764
% todo highly offensive 764
% todo leg concern to the public 765
% publicity 765 


\subsubsection{\emph{Neff v. Time, Inc.}}

\begin{enumerate}
    \item % todo 754
\end{enumerate}

\subsubsection{\emph{Sipple v. Chronicle Publ'g Co.}}

\begin{enumerate}
    \item % todo 759
\end{enumerate}

\subsection{False Light}

\begin{enumerate}
    \item  % todo 769
    \item % todo vs. defamation 769
    \item % todo rejecting false light 770
\end{enumerate}

\subsubsection{\emph{Cantrell v. Forest City Publ'g Co.}}

\begin{enumerate}
    \item % todo 765
\end{enumerate}

\subsection{\emph{Hustler Magazine v. Falwell}}

\begin{enumerate}
    \item % todo 56
\end{enumerate}
