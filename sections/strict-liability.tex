\section{Strict Liability}

\begin{enumerate}
    \item \textbf{Strict liability}: the plaintiff does not need to prove the 
    defendant's negligence.
    \item Applies to abnormally dangerous activities, e.g., transporting 
    gasoline.
    \item Abnormally dangerous products (e.g., guns): the fact that a product 
    may be used in an abnormally dangerous way does not hold the manufacturer 
    liable for harms under strict liability. \emph{Kelley}.
    \item Strict liability only applies to harms resulting ``from that which 
    makes the activity ultrahazardous.'' \emph{Foster}.
\end{enumerate}

\subsection{Traditional Strict Liability}

\begin{enumerate}
    \item Strict liability means that the plaintiff's prima facie case does not 
    need to prove that the defendant acted in a blameworthy fashion.
    \item Plaintiff has to prove other elements: cause-in-fact, proximate 
    cause.
    \item Most important areas: legislative programs (e.g., workers' 
    comp.---not usually referred to as strict liability, but that's how it 
    operates, because fault is generally not an issue), and certain 
    ``abnormally dangerous'' activities (e.g., keeping wild animals, blasting, 
    use of poisons), strict products liability.
    \item An anomally in the field of torts historically---but liability 
    without fault is present in other areas of the law: contracts (breach of 
    contract generally does not involve fault).
    \item Premised on the need for greater loss distribution that what would 
    occur under negligence.
\end{enumerate}

\subsubsection{Abnormally Dangerous Activities: \emph{Siegler v. Kuhlman}}

Strict liability applies to abnormally dangerous activities.

\begin{enumerate}
    \item Overturned gas trailer caused fire.
    \item Transporting gas by truck is abnormally dangerous. It possesses all of 
    the Restatement factors for strict liability.
    \item Trial court: defendants overcame charges of negligence.
    \item Holding: reversed and remanded for retrial on strict liability.
\end{enumerate}

\subsubsection{Negligence vs. Strict Liability: \emph{Indiana Harbor Belt 
Railroad Co. v. American Cyanamid Co.}}

\begin{enumerate}
    \item Cyanamid loaded 20k gallons of acrylonitrile in a leased railroad car.
    \item Indiana Harbor Belt asserted (1) negligent maintenance of the train 
    car and (2) strict liability because transport of bulk acrylonitrile through 
    Chicago is abnormally dangerous.
    \item Distinction from \emph{Siegler}: defendant there was the transporter; 
    here it is the shipper.
    \item Harm here was the result of carelessness, not inherent danger.  
    Negligence would have been an effective deterrent.
    \item Reversed and remanded to be tried on negligence.
\end{enumerate}

\subsubsection{Abnormally Dangerous Products vs. Activities: \emph{Kelley v. 
R.G. Industries, Inc.}}

The fact that a product may be used in an abnormally dangerous way does not 
hold the manufacturer liable for harms under strict liability.

\begin{enumerate}
    \item Gunshot victim claimed the manufacturing and marketing of handguns is 
    abnormally dangerous. Court rejected the argument because under SL, the 
    activity must be dangerous in relation to the area where it occurs.
\end{enumerate}

\subsubsection{Hazard and Causation: \emph{Foster v. Preston Mill Co.}}

Strict liability only applies to harms resulting ``from that which makes the 
activity ultrahazardous.''

\begin{enumerate}
    \item Blasting operations caused a mink to kill its kittens. Court rejected 
    the strict liability argument because liability only exists for harms 
    resulting ``from that which makes the activity ultrahazardous.''
\end{enumerate}

\subsection{Products Liability}

\begin{enumerate}
    \item A manufacturer (or anyone in the chain of distribution) is strictly 
    liable when an article is placed on the market knowing that it is to be used 
    without further inspection for defects and proves to have a defect that 
    causes injury.
    \item Strict liability is not based on fault. The rationale is loss 
    distribution.
    \item In addition to strict liability, manufacturers can also be held liable 
    for negligence, express/implied warranty, and representation.
    \item Defect is defined as an imperfection that impairs the operation or 
    safety of a product.
    \item Issues in defining defect have split jurisdictions:
    \begin{enumerate}
        \item Must the product be ``unreasonably dangerous''? The majority of 
        states require it, but CA says no, because it introduces a negligence 
        standard. Can there be liability for an open an obvious defect? CA no, 
        others yes.
        \item Must the product be in a condition not anticipated by the buyer, 
        i.e., beyond the consumer's expectation? Some jurisdictions make this a 
        requirement. In CA, that's one method of getting to strict liability 
        (\emph{Barker}---see the next bullet), but it's not necessary. Some say this introduced 
        negligence, but (1) it's in hindsight, not foresight and (2) the burden 
        of proof is on the defendant, not the plaintiff. But see \emph{Sewell}: 
        if the product is too complex, the consumer expectations test does not 
        apply.
        \item The \emph{Barker} test: a product is defective if (1) it failed 
        to perform as safely as an ordinary consumer would expect whe used in 
        an intended or reasonably foreseeable manner, or (2) if the benefits 
        don't outweigh the risk of danger inherent in the design.
    \end{enumerate}
    \item No strict liability for prescription pharmaceuticals.
    \item \textbf{State-of-the-art defense}: is the defendant relieved of 
    liability if at the time of the manufacture, nobody could have made it 
    more safe. Some 
    states have adopted this rule; in CA it's only adopted in a few 
    areas---pharmaceuticals, warning defects (\emph{Anderson}).
    \item Manufacturing defect: the product is different than all the others 
    produced.
    \item Warning defect: inadequate labels or instructions. Purposes: inform 
    the consumer of dangers to let her avoid buying it or to use it more safely.
    \item Restatement (Third) of torts would radically shift products liability 
    in favor of manufacturers. Plaintiff would have to prove the existence of a 
    reasonable alternative design. The \emph{Potter} court rejected the rule as 
    unduly requiring plaintiffs to retain expert witnesses. No chance that this 
    rule will be adopted in CA in the foreseeable future.
    \item Restatement (Third) also tries to combine products liability into a 
    single principle (\S\ 550.1). Levy fears this would wipe away much of 
    existing products liability law.
    \item Strict liability does not allow recovery for economic damages.
    \item If plaintiff is negligent, we will apply comparative negligence, even 
    in strict products liability cases. There's dispute about whether fault can 
    logically apply in strict liability contexts.
    \item Preemption: when do federal rules preempt state law? There are three 
    types of preemption under the Supremacy Clause: express, conflict, field.
    \item Can there be liability for component parts of a product? Cases are not 
    all in agreement. Best rule: (1) if the component part is defective, there 
    can be liability. (2) If the component parts manufacturer was intimately 
    involved in the design of the whole product, it can be held liable for the 
    whole product.
    \item The ``sophisticated/professional user'' defense: a manufacturer 
    generally owes no duty to warn professionals against the danger if the 
    danger is generally known to the profession.
    \item \emph{Daly}, comparative negligence: can you apply assumption of risk 
    to a strict products liability case? The answer will likely be yes, though 
    there is no California Supreme Court case that directly addresses it.
    \item Focuses on the product itself, not the manufacturer's conduct.
\end{enumerate}

\subsubsection{Strict Liability for Food and Drugs: \emph{Pillars v. R. J. 
Reynolds Tobacco Co.}}

\begin{enumerate}
    \item Human toe in chewing tobacco triggered strict liability.
\end{enumerate}

\subsubsection{Origins of Strict Products Liability: \emph{Greenman v. Yuba 
Power Products, Inc.}}

\begin{enumerate}
    \item This is the first case to find strict products liability for defective 
    products.
    \item A piece of wood flew out of a woodworking tool, the Shopsmith, 
    injuring the plaintiff, Greenman.
    \item 10.5 months later, he sued the manufacturer, Yuba, and the retailer 
    for breach of warranty and negligence.
    \item The court found that (1) the retailer was not negligent and did not 
    breach an express warranty, and (2) the manufacturer did not breach an 
    implied warranty. Thus, the only valid causes of action were (1) a breach of 
    implied warranty against the retailed and (2) negligence and a breach of 
    express warranty against the manufacturer. The jury found for the retailer 
    and found \$65,000 against Yuba.
    \item Yuba appealed; Greenman sought appeal against the retailer only if the 
    judgment against Yuba was reversed.
    \item The jury could have reasonably found that Yuba negligently 
    manufactured the Shopsmith.\footnote{Casebook p. 520.}
    \item The requirement that consumers need not give notice of injury to 
    manufacturers with whom they have not directly dealt. Thus, the plaintiff's 
    cause of action was not barred.
    \item The manufacturer can be held strictly liable for a defective product 
    even in the absence of an express warranty: \textbf{``A manufacturer is 
    strictly liable in tort when an article he places on the market, knowing 
    that it is to be used without inspection for defects, proves to have a 
    defect that causes injury to a human being.''}\footnote{Casebook p. 521.}
    \item Liability for defective products is governed by strict liability, not 
    contract warranties.  \item The purpose of strict liability for defective 
    products is to ensure that manufacturers bear the costs of injuries to 
    consumers.
    \item \textbf{Warranties}:
    \begin{enumerate}
        \item \emph{Express warranties}: created when the seller makes factual 
        assertions about a product.
        \item \emph{Implied warranties}: (1) ``implied warranty of 
        merchantability'' is a guarantee that products conform to their 
        description and are safe for their intended use; (2) ``implied warranty 
        of fitness for a particular purpose'' is created when the seller has 
        reason to know that the buyer buys the goods for a particular purpose. 
        \footnote{Casebook p. 523.}
        \item The advantage of basing a products liability case on a warranty 
        theory is that liability is strict and there can be compensate for pure 
        economic loss. The disadvantage is that sellers can limit remedies or 
        disclaim warranties altogether. Warranties also historically require 
        prompt notice of dissatisfaction to the defendant.\footnote{Casebook p. 
        524.}
    \end{enumerate}
    \item \textbf{Misrepresentation}: another theory for product liability (in 
    addition to negligence and warranty theory). It holds manufacturers liable 
    for harm caused by justified reliance on the misrepresentation.
    \item \textbf{Strict product liability}: a fourth theory. It imposes 
    liability on manufacturers for defective products that proximately cause 
    personal and property injuries. [What about economic injuries?] This is the 
    theory in \emph{Greenman}.
\end{enumerate}

\subsubsection{No Privity Required: \emph{Lee v. Crookston Coca-Cola Bottling 
Co.}}

Consumers can sue manufacturers directly without involving others in the 
distribution chain.

\begin{enumerate}
    \item Coke bottle exploded in waitress's hands.
    \item Four policy justifications for strict products liability:
    \begin{enumerate}
        \item Discourage marketing of defective products.
        \item Put burden of loss on manufacturer.
        \item Maximize legal protections for consumers.
        \item Allow injured parties to bring actions directly against those who 
        caused the injuries without involving others in the distribution chain.
    \end{enumerate}
\end{enumerate}

\subsubsection{Foreseeable Dangers: \emph{Gray v. Manitowoc Company}}

Strict liability only applies of the dangers were beyond what an ``ordinary 
consumer'' would anticipate.

\begin{enumerate}
    \item Crane hit construction worker, who argued that mirrors should have 
    been provided. Court found that the safety hazards of this type of crane 
    were well known in the industry and thus was not ``dangerous to a degree not 
    anticipated by the ordinary consumer of this product.''\footnote{Casebook p. 
    531.}
\end{enumerate}

\subsubsection{Tobacco-Related Health Problems: \emph{Roysdon v. R.J. Reynolds 
Tobacco Co.}}

Cigarettes' health risks do not make them defective.

\begin{enumerate}
    \item Roysdon suffered from tobacco-related health problems. The Sixth 
    Court held that a product that is generally harmful to health is not the 
    same as a product that is defectively manufactured.
    \item ``...we think that a reasonable jury could not find that that the 
    cigarettes are defective.''\footnote{Casebook p. 534.}
    \item If the dangers of smoking are \emph{not} \textbf{common knowledge}, 
    then a jury \emph{may} be able to find that cigarettes are unreasonably 
    dangerous.
\end{enumerate}

\subsubsection{Replacing the Consumer Expectations Test: \emph{Barker v. Lull 
Engineering Co., Inc.}}

Products liability does not depend on a consumer's expectation of 
safety---i.e., it should not be limited to the ``unreasonably dangerous'' 
standard where liability would only apply if was less safe than the consumer 
expected.

\begin{enumerate}
    \item The plaintiff was injured while operating a high-lift loader. He 
    alleged defective design as the proximate cause.
    \item The court developed a two-prong test for determining whether a 
    productive is defective in design:
    \begin{enumerate}
        \item If it fails to perform as safely as an ordinary consumer would 
        expect when used in an intended or reasonably foreseeable manner;, or
        \item When the benefits of the design do not outweigh the inherent 
        dangers, i.e., if the design embodies ``excessive preventable 
        danger.''\footnote{Casebook p. 542.}
    \end{enumerate}
    \item The court rejected the idea that a manufacturer can be held strictly 
    liable only if a product is ``unreasonably dangerous,'' i.e., if it is 
    ``more dangerous than contemplated by the average consumer.''
\end{enumerate}

\subsubsection{State-of-the-Art Defense: \emph{Beshasda v. Johns-Manville 
Products Corp.}}

The state-of-the-art defense shields manufacturers from liability if they 
could not have known of the dangers their product posed at the time of 
manufacturing. Asbestos is a paradigmatic case. Although New Jersey here 
rejected the defense, most jurisdictions (including California) allow it.

\begin{enumerate}
    \item This was a consolidated case against six asbestos manufacturers. The 
    defendants' ``state-of-the-art'' defense argued that the dangers of asbestos 
    were unknowable at the time the injuries in question occurred.
    \item The trial court denied the plaintiffs' motion to strike the 
    state-of-the-art defense.
    \item The plaintiffs claimed strict liability for failure to warn. ``The 
    issue is whether the medical community's presumed unawareness of the dangers 
    of asbestos is a defense to plaintiffs' claims.''\footnote{Casebook p. 549.}
    \item The court distinguished negligence, which is conduct oriented, from 
    strict liability, which is product oriented.
    \item There is a two-part \textbf{risk equity} test to determine whether a 
    product is safe:]\
    \begin{enumerate}
        \item Does its utility outweigh its risk?
        \item Has that risk been reduced to the greatest extent possible 
        consistent with the product's utility?\footnote{Casebook p. 551.}
    \end{enumerate}
    \item In strict liability cases, there is no need to prove that the 
    manufacturer knew or should have known of the product's danger. Knowledge is 
    imputed to the manufacturer. ``...in strict liability cases, culpability is 
    irrelevant.''\footnote{Casebook p. 552.} The state-of-the-art defense is a 
    negligence defense because it rests on the defendant's conduct.
    \item There are three reasons for imposing strict liability for failure to 
    warn:
    \begin{enumerate}
        \item \emph{Risk spreading}: spreading costs of harm to manufacturers 
        and purchasers is preferable to imposing it on innocent consumers.
        \item \emph{Accident avoidance}: industries play an important role in 
        safety research, and we want them to maximize it.
        \item \emph{Fact finding}: the dangers of asbestos \emph{could have been 
        known}, but weren't. Regardless, it's better to leave out the negligence 
        concept of knowability, because the framework here is strict liability, 
        not negligence.
    \end{enumerate}
    \item The court granted the plaintiffs' motion to strike the 
    state-of-the-art defense.
    \item (In contrast to New Jersey, the majority trend is to allow the 
    state-of-the-art defense, including in California.\footnote{Casebook p.  
    557.})
\end{enumerate}

\subsubsection{Federal Preemption: \emph{Riegel v. Medtronic, Inc.}}

Medical device regulations preempt state tort law actions. The rationale is 
that states and individual juries should not be able to undermine the FDA's 
regulatory authority.

\begin{enumerate}
    \item The Medical Device Amendments (MDA) to the FDCA established various 
    levels of federal oversight for medical devices depending on their risks.  
    Devices that were already on the market were grandfathered, and new devices 
    that were ``substantially similar'' to the existing devices could also 
    sidestep premarket approval.
    \item Here, the doctor inflated a balloon catheter beyond the pressure limit 
    indicated on its label, causing injury to the plaintiff.
    \item The district court held (1) that the MDA preempted the plaintiff's 
    common law tort claims and (2) that the MDA preempted the plaintiff's 
    negligent manufacturing claim because it did not claim that the manufacturer 
    violated federal law.\footnote{Casebook p. 562.}
    \item Justice Scalia:
    \begin{enumerate}
        \item The MDA only preempts state requirements that are different or in 
        addition to the applicable federal requirements. The court here, relying 
        on \emph{Lohr}, found that state law negligence and strict liability 
        claims are different and therefore the MDA preempts them.
        \item If federal regulations did not preempt state common law, then 
        states and individual juries would be able to undermine the FDA's expert 
        evaluations and policies.
        \item The consequence is that the FDA's approval of the device preempts 
        state tort law actions based on negligence and strict 
        liability.\footnote{Casebook p. 565.}
    \end{enumerate}
\end{enumerate}

\subsubsection{More Federal Preemption: \emph{McKenney v. PurePac Pharmaceutical}}

If the state law cause of action is in \emph{parallel} to the federal cause of 
action, there is no preemption. 

\begin{enumerate}
    \item PurePac manufactured the generic drug metoclopramide. McKenney claimed 
    she was injured because of ``false or misleading statements'' in the drug's 
    labeling.\footnote{Casebook p. 89.}
    \item The CA Superior Court sustained PurePac's demurrer and entered summary 
    judgment in its favor.
    \item In its demurrer, PurePac contended that McKenney's claim was barred by 
    the defense of federal preemption. Because it submitted its labeling to the 
    FDA and won approval, PurePac argued it could not be held liable for state 
    tort law claims regarding any deficiencies in the labeling.
    \item \emph{Brown} and \emph{Carlin} affirmed strict tort liability for 
    pharmaceutical manufacturers in California.
    \item The court found that FDA approval of labeling does not preempt state 
    tort claims against manufacturers.
    \item Reversed (demurrer rejected).
    \item Reconciling \emph{Riegel} and \emph{McKenney}: courts will likely 
    allow state causes of action that are parallel to the federal rules.
\end{enumerate}

\subsubsection{Restatement (Third) Approach: \emph{Potter v. Chicago Pneumatic 
Tool Co.}}

The Restatement (Third) requires plaintiffs claiming design defects to propose 
a ``reasonable alternative design'' and holds that a product is defective only 
if there are foreseeable risks that a reasonable alternative design would have 
avoided. The court rejected this standard as too onerous for plaintiffs.

\begin{enumerate}
    \item Plaintiffs claim they were injured from excessive vibrations as a 
    result of defective warnings on the defendant's product.
    \item Courts are divided on the definition of design defects.
    \item The Restatement (Third) requires plaintiff to prove the existence of a 
    ``reasonable alternative design.''\footnote{Casebook p. 566.} The defendants 
    argue that the court should adopt this standard.
    \item The court here reasoned that the Restatement (Third) approach puts an 
    undue burden on plaintiffs by requiring expert witnesses even in cases where 
    a lay jury could infer a design defect. Moreover, cases exist where a 
    product is defective even though no alternative design exists.
    \item The Restatement (Third) holds that a product is defective only if 
    there are foreseeable risks that a reasonable alternative design would have 
    avoided. Thus, it allows the state-of-the-art defense and imposes a burden 
    on plaintiffs more onerous than ordinary negligence (because under ordinary 
    negligence, the plaintiff only needs to prove a foreseeable risk, but not 
    the existence of an alternative design).
\end{enumerate}

\subsubsection{Economic Damages: \emph{Two Rivers Company v. Curtiss Breeding 
Service}}

The court held that plaintiffs cannot recover for economic loss alone under 
strict liability. However, if there is \emph{also} personal or property 
injury, there can be recovery for pure economic loss.

\begin{enumerate}
    \item Plaintiff sued defendant for economic damages from allegedly 
    defective cattle semen.
    \item The court distinguished four types of property loss:
    \begin{enumerate}
        \item Personal injury to the user or the user's property.
        \item Pure economic loss. Earlier courts held that \textbf{strict liability 
        does not apply to pure economic loss}. Instead, individuals claiming 
        economic loss must claim under UCC breach of implied warranty or common 
        law negligence.
        \item Economic loss to the purchased product itself.
        \item Hybrid: harm to the plaintiff's other property as well as to the 
        product itself.
    \end{enumerate}
    \item The court held that plaintiffs cannot recover for economic loss 
    alone under strict liability. However, if there is \emph{also} personal or 
    property injury, there can be recovery for pure economic loss.
\end{enumerate}
 
\subsubsection{Comparative Negligence in Strict Products Liability: \emph{Daly 
v. General Motors Corp.}}

In California, courts can take into account the plaintiff's comparative 
negligence (\emph{Li}) \emph{does} in actions based on strict products 
liability.

\begin{enumerate}
    \item Decedent crashed into a fence along the highway. The door of his 
    Opel was thrown open, causing him to be forcibly ejected from the car. He 
    sustained fatal head injuries. It was undisputed that his injuries would 
    have been relatively minor if he had remained inside the car.
    \item Daly was comparatively negligent because he did not use his seat 
    belt or the door lock.
    \item Strict products liability in California is based on the ``problems 
    of proof'' in proving negligence or breach of warranty. Since injured 
    consumers are powerless to protect themselves, it is fair to place the 
    burden of loss on manufacturers.\footnote{Casebook p. 577.}
    \item Held: courts can consider a plaintiff's comparative negligence in 
    adjudicating strict liability claims.
\end{enumerate}

\subsubsection{Applying the \emph{Barker} Test to Component Parts: 
\emph{Gonzales v. Autoliv}}

Component parts manufacturers can be strictly liable for design defects if the 
benefits of the design do not outweigh the inherent dangers.

\begin{enumerate}
    \item Gonzalez suffered an eye injury when an airbag manufactured by 
    Autoliv deployed in a minor collision.
    \item The court applied the \emph{Barker} test, asking whether the 
    benefits of the product's design outweighed the inherent risks. Autoliv 
    offered no evidence that the benefits outweighed the dangers. Summary 
    judgment for Autoliv was therefore inappropriate.
\end{enumerate}

