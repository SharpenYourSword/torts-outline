\section{Negligence}

\subsection{Overview}

\begin{enumerate}
    \item ``The failure to exercise the standard of care that a \textbf{reasonably prudent person} would have exercised in a similar situation.''\footnote{Black's Law.}
    \item Negligence requires proof that the defendant acted unreasonably.
    \item The standard of care is objective.
    \item Negligence has six factors:
    \begin{enumerate}
        \item Duty.
        \item Breach of duty.
        \begin{enumerate}
            \item Negligence.
            \item Breach of the standard of care.
            \item Failure to act as a reasonably careful person would under the circumstances.
        \end{enumerate}
        \item Cause-in-fact.
        \item Proximate cause.
        \item Damages.
    \end{enumerate}
    \item The Hand Formula is Judge Learned Hand's test for determining negligence. It works best in scenarios where the actor takes a calculated risk (e.g., business decisions). It is less useful in cases where the actor was simply not paying attention.
    \begin{enumerate}
        \item B = Burden of precautions necessary to prevent an accident.
        \item P = Probability that an accident will occur.
        \item L = Magnitude of the loss if the accident occurs.
        \item Negligence exists if B\textless PL---i.e., if the burden of precautions is less than the harm multiplied by the probability of occurrence.
    \end{enumerate}
\end{enumerate}

\subsubsection{\emph{Pitre v. Employers Liability Assurance Corporation}}

\begin{enumerate}
    \item The plaintiffs' son died when a patron at a carnival game was winding up a pitch and hit him in the head. The trial court found in favor of the plaintiffs. The determining factor, the Court of Appeal reasoned, is how a ``reasonably prudent individual'' would have acted or what precautions he would have taken under similar circumstances. \textbf{Negligence occurs only if the danger is both foreseeable and unreasonable.} The court held that the danger was foreseeable but not unreasonable, and therefore there was no negligence.
\end{enumerate}

\subsubsection{\emph{United States Fidelity \& Guaranty Company v. Plovidba}}

\begin{enumerate}
    \item Inside a dark room, the hatch to a cargo hold on a ship was left open. A longshoreman fell through it and died. The trial court found for the defendant. Here, Richard Posner writing for the Seventh Circit applies the Hand Formula, reasoning that B was relatively small (it would have been easy to close the hatch or leave a light turned on) and L was high (the victim died). P, however, was very small. There was no reason for the longshoreman to enter the hold. In fact, he was probably there to steal liquor, and the evidence suggests he knew the hatch was open and tried to skirt around it. The shipowner was therefore not liable for negligence.
\end{enumerate}

\subsection{Standard of Conduct}

\subsubsection{\emph{Cordas v. Peerless Transp. Co.}}

\begin{enumerate}
    \item A man was mugged at gunpoint by two other men in New York City. He chased after them. One of the muggers jumped into a taxi, held the driver at gunpoint, and told him to drive. While the cab was in motion, the driver jumped out, and a few seconds later, so did the hijacker. The cab crashed into a sidewalk and injured the defendants. The trial court held that the driver was not negligent because he acted as a reasonable person would act under similar circumstances.
    \item Courts are divided on the question of whether juries should receive special instructions regarding negligence claims in emergency circumstances. On the one hand, it is redundant to reiterate that a defendant must be held to the standard of what a reasonable person would do in a similar emergency situation. Others claim it helps clarify the standard.
    \item \textbf{Conditional privilege}: choose the lesser of two harms.
\end{enumerate}

\subsubsection{\emph{Breunig v. American Family Insurance Company}}

\begin{enumerate}
    \item A schizophrenic woman had a psychotic episode while driving her car. The question was whether she had foreknowledge of her susceptibility to such attacks. The general rule is that insanity or another mental deficiency does not limit liability for negligence. In other words, \textbf{insane people are held to the reasonable person standard}. The court here notes that may be too harsh to exclude the insanity defense when a driver is suddenly overcome without warning. The Supreme Court agrees with the lower courts that the defendant did have the necessary foreknowledge, and held for the plaintiff. 
    \item Two frameworks for assessing liability from the sudden onset of mental illness:
    \begin{enumerate}
        \item Fairness: it's not fair to punish someone who could not have avoided having a seizure.
        \item Loss distribution: if someone has to bear the cost of repairing the harm, it should be the perpetrator, not the victim.
    \end{enumerate}
\end{enumerate}

\subsubsection{\emph{Neumann v. Shlansky}}

\begin{enumerate}
    \item Should courts hold underage defendants to the standard of reasonable adults?
    \item An eleven-year-old hit a golf ball that struck the defendant in the knee, causing serious injury. Generally, children are held to the standard of a \textbf{reasonable person of like age, intelligence, and experience under the circumstances.} In this case, however, the child was engaging in an ``adult activity,'' and therefore the court held him to the adult reasonable person standard.
    \item Some states are moving from ``adult activity'' to ``inherently dangerous activity.''
\end{enumerate}

\subsubsection{\emph{Melville v. Southward}}

\begin{enumerate}
    \item The defendant, a podiatrist, operated on the plaintiff's foot. The plaintiff sued for malpractice, and introduced the testimony of an orthopedist, who questioned the necessity and sanitation of the operation. The question is whether the orthopedist, a practitioner from a different school of medicine, should have been allowed to testify about the standard of care in podiatry. The trial court allowed the orthopedist to testify. The Supreme Court of Colorado here agreed with the appellate court that the testimony should not have been allowed because it was ``nothing more than an expression of opinion that that the general practice of podiatry did not meet the standard of care observed by an orthopedic surgeon.'' It remanded the case for a new trial.
    \item In malpractice cases, the ``competent professional'' standard replaces the ``reasonable person'' standard.
    \item There is disagreement about whether doctors in rural areas should be held to different standards than urban doctors.
    \item Medical specialists in the same geographic region are often reluctant to testify against each other---a ``conspiracy of silence.''
    \item In a limited range of cases, a jury of laypeople can determine whether a practice met an acceptable standard of care.
\end{enumerate}

\subsubsection{\emph{Cobbs v. Grant}}

\begin{enumerate}
    \item Doctors are required to obtain \textbf{informed consent} from patients. The plaintiff here sued a doctor who operated on a stomach ulcer but did not discuss the surgery's inherent risks. Complications developed, another operation was required, more complications developed, and so on. The plaintiff argued that (1) the doctor acted negligently in the performance of the surgery (which the jury found in favor of the plaintiff) and (2) that the doctor failed to obtain informed consent. The Supreme Court of California here notes that courts are divided as to whether this type of tort should be deemed a \textbf{battery or negligence}. The court aligned itself with a ``majority trend'' that advocates reserving battery for cases where a doctor performs an operation without the patient's consent. Generally, physicians are required to tell patients about major risks (but not every minor risk) and obtain the patient's consent. In this case, the court finds that there is not enough evidence to show that the doctor acted negligently, and the case is remanded for a new trial.
    \item \textbf{Failure to obtain informed consent can subject a physician to negligence liability.} Unless the physician misrepresents the entire procedure, most courts will not characterize the behavior as intentional battery.
\end{enumerate}

\subsection{Rules of Law}

\begin{enumerate}
    \item Juries typically determine what constitutes reasonable conduct under the circumstances. Judges, however, ill sometimes establish a rule for what constitutes negligent conduct under particular circumstances.
    \item For instance, in \emph{Baltimore \& Ohio R.R.}, Justice Holmes established a rule requiring drivers to get out of their cars and examine railroad crossings.
\end{enumerate}

\subsubsection{\emph{Akins v. Glens Falls City School District}}

\begin{enumerate}
    \item A foul ball injured a spectator at a baseball game. She sued the ballpark's owners, the local school district, for negligence. The trial court helf in favor of the plaintiff. The appellate court reversed, finding that the school district had not acted negligently, and establishing a specific rule for ballpark backstops. The dissent argued that such a rule ``robs the jury'' of the ability to consider important circumstances and locks the law in ``a position that is certain to become outdated.''
\end{enumerate}

\subsection{Negligence Per Se}

\begin{enumerate}
    \item In some states, a jury \textbf{must presume} negligence when a statute is breached. The defendant is free to rebut. California basically follows this rule, with a few exceptions.\footnote{Cal. Evid. C. § 669. See course reader p. 11.} 
    \item In states that do not follow negligence per se, juries are free to (but need not) \textbf{infer} that breach of statute constitutes negligence---e.g., a car doesn't slow down and hits a pedestrian in a crosswalk.
    \item Plaintiff can, and usually will, plead both common law negligence and negligence per se.
    \item Compliance with a statute is generally not proof of due care.
\end{enumerate}

\subsubsection{\emph{Wawanesa Mutual Insurance Co. v. Matlock}}

\begin{enumerate}
    \item A minor bought cigarettes for another minor, who later dropped the cigarette and caused a fire that led to significant property damage. The insurer sued the first minor's father, and the trial court found for the insurer. The appellate court overturned the ruling. It argued that the statute in question was meant to protect against the health hazards of tobacco, not the fire hazard, and therefore cannot be used to establish a standard of conduct in this case.
\end{enumerate}

\subsubsection{\emph{Stachniewicz v. Mar-Cam Corporation}}

\begin{enumerate}
    \item A patron injured in bar brawl sued the bar owner. The plaintiff relied on (1) an Oregon statute which prohibits giving alcohol to an intoxicated person and (2) an Oregon regulation that prevents bar owners from allowing disorderly conduct on their premised. The trial court found for the defendant. The appellate court overturned, reasoning that (1) the statute is inapplicable because the brawler was already drunk when he arrived, so there is no way to tell if another drink caused the brawl, but (2) the regulation was intended specifically to protect customers from injury, and therefore can be an appropriate standard for negligence in this case.
\end{enumerate}

\subsubsection{\emph{Gorris v. Scott}}

\begin{enumerate}
    \item Several sheep on a ship were swept overboard. The plaintiff sued the shipowner, arguing that the Contagious Diseases (Animals) Act required the shipowner to enclose the sheep in pens of certain dimensions, which the shipowner failed to do. The court found in favor of the shipowner, reasoning that the Act was intended to prevent the spread of contagious diseases, not to prevent sheep from falling overboard.
    \item \textbf{Statutory purpose doctrine}: for the statute to be relevant, the harm must be one of the harms the statute was meant to prevent.
\end{enumerate}

\subsection{Cause in Fact}

\begin{enumerate}
    \item Plaintiff must prove that the harm would not have occurred \textbf{but for} the defendant's actions.
    \item If there are multiple causes of harm, each can be but-for causes as long as the harm would not have occurred without it.
    \item If there are multiple causes of harm, but none alone is a but-for cause, courts can use the substantial factor test. See \emph{Northington} below.
    \item \textbf{Proximate cause} removes liability when ``the connection between the plaintiff's harm and defendant's liability is unforeseeable or so attenuated that public policy preclused liability.''\footnote{Casebook p. 206.}
\end{enumerate}

\subsubsection{\emph{East Texas Theatres, Inc. v. Rutledge}}

\begin{enumerate}
    \item At the defendant's movie theater, somebody threw a bottle from a balcony which struck and injured the plaintiff. The jury found the theater liable because it negligently failed to remove ``rowdy persons'' from the balcony during the game, and the Texas appellate court affirmed. The Texas Supreme Court clarified that proximate cause has two elements: (1) cause-in-fact and (2) foreseeability. The court held that the prosecution failed to show that the injuries would have occurred but for the removal of the ``rowdy persons.'' It reversed the lower court's ruling and held for the defendants.
\end{enumerate}

\subsubsection{\emph{Anderson v. Minneapolis, St. P. \& S. S. M. Ry. Co.}}

\begin{enumerate}
    \item A spark from a railroad started a fire in a bog on one side of the defendant's property. Another unrelated fire was burning on the other side. The fire from the railroad destroyed the defendant's property, and a few days later it joined with the other fire to make one big fire. The railroad argues that it cannot be held liable because the defendant's house would have been destroyed by the other fire anyway. The trial court refused to instruct the jury to follow a rule from an earlier case, \emph{Cook}, which held that there is no liability when two fires jointly destroy property. On this basis, the trial court found for the plaintiff. The railroad requested a motion for judgment notwithstanding the verdict, which was denied. On appeal, the Supreme Court of Minnesota held that the trial court was correct in refusing to apply the \emph{Cook} rule and found for the plaintiffs.
    \item \textbf{Substantial factor test}: If two independent fires join to cause property damage, there is joint liability, even if neither alone is a but-for cause. Redundant causation is not necessary.
    \item Courts split on whether to use the substantial factor test when only one actor is liable. California courts do use it (and reject the but-for test).
\end{enumerate}

\subsubsection{\emph{Northington v. Marin}}

\begin{enumerate}
    \item The plaintiff, a prison inmate, sued the defendant, a prison guard, for circulating rumors that labeled him a snitch and caused other inmates to assault him. Other guards had spread the same rumors. The trial court found that although the defendant's action was not a but-for cause (since the harm would have occurred without his action), his contribution to the harm was nonetheless a \textbf{substantial factor}. The Tenth Circuit affirmed: ``Multiple tortfeasors who concurrently cause an indivisible injury are jointly and severally liable; each can be held liable for the entire injury.''
\end{enumerate}

\subsubsection{\emph{Herskovitz v. Group Health Cooperative of Puget Sound}}

\begin{enumerate}
    \item The plaintiff brought the action on behalf of her husband, a deceased lung cancer patient, against a doctor that negligently failed to diagnose the patient's lung cancer on his first visit, proximately causing his chance of survival to drop from 39 percent to 25 percent. Neither fact was in dispute. The defendant argued that the plaintiff must prove that the patient ``probably'' would have lived but for the negligence---that is, without the doctor's negligence, the patient's chance of survival must have been more than 50 percent. The trial court granted summary judgment for the defendant on this argument. The Supreme Court of Washington reversed, arguing that any other decision would mean a ``blanket release'' for doctors' negligence any time the patient's chance of survival was less than 50 percent. The court reasoned that if a defendant's acts have \emph{increased the risk} of harm to the plaintiff, a jury should decide whether the increased risk actually caused the harm in question.
\end{enumerate}

\subsubsection{\emph{Summers v. Tice}}

\begin{enumerate}
    \item The \emph{Summers} rule applies where there are a small number of defendants, only one of them committed the harm, and we don't know which one.
    \item The plaintiff and the two defendants were hunting quail. The two defendants shot at a quail in the direction of the plaintiff. The plaintiff suffered injuries, but it's not clear which defendant's shot was the cause. The court reasons that in this case, the burden of proof shifts to the defendants to determine which one of them caused the injury. If they cannot, ``each defendant is liable for the whole damage whether they are deemed to be acting in concert or independently.'' The lower courts found the defendants liable and the Supreme Court of California affirmed.
    \item Can you hold three defendants liable under the \emph{Summers} test?
    \item Another case with joint tortfeasors, see \emph{Drabek v. Sabley} above (kids throwing snowballs at cars).
\end{enumerate}

\subsubsection{\emph{Sindell v. Abbott Laboratories}}

\begin{enumerate}
    \item The plaintiff was harmed by DES, a prenatal drug intended to protect against miscarriages but which turned out to pose significant danger to unborn children. The plaintiff did not know which company manufactured the specific drug her mother took, but since several companies manufactured the drug according to the same formula, she sued them all. The companies won a dismissal at trial on the grounds that the plaintiff could not identify which company caused the harm.
    \item The Supreme Court of California considered four theories of liability:
    \begin{enumerate}
        \item The \emph{Summers} test: this fails because there are so many defendants (over 200) that it is highly unlikely that any one of them caused this specific injury.
        \item The ``concert of action'' theory: if the defendants had acted in concert to cause the injury, they would be equally liable. In this case, there is not sufficient evidence to show that the defendants had a common plan to cause harm (e.g., by conducting inadequate safety tests or giving insufficient safety warnings).
        \item ``Industry-wide'' or ``enterprise'' liability: if an entire industry cooperates on an element of the harm in question---e.g., by delegating safety testing to a trade association---they can be held jointly liable. Here, the fact that DES manufacturers shared testing and promotion methods does not establish industry-wide liability, because (1) there are so many manufacturers and (2) safety standards are mostly regulated by the FDA.
        \item \textbf{Market share liability}---a variation of the \emph{Summers} test: each manufacturer's liability and share of the damages are proportionate to its market share.
    \end{enumerate}
    \item Relying on the fourth theory, the Supreme Court of California reversed, allowing the plaintiff to proceed with her cause of action.
    \item Most states have not adopted market share liability.
    \item Defendants are allowed prove definitively that they did not contribute to the harm (e.g., if they can show that they did not produce the drug at the time).
    \item Some states require defendants to be joined so that a significant share of the market is represented, and that missing market share proportionally reduces the plaintiff's compensation. Usually (but not always) this must be the nationwide market.\footnote{Casebook p. 229 n. 2.}
\end{enumerate}

\subsubsection{\emph{Ayers v. Township of Jackson}}

\begin{enumerate}
    \item A town in New Jersey was found to have caused toxic exposure by its ``palpably unreasonable'' management of a landfill. Plaintiffs did not develop any illnesses, but they sought to recover (1) damages for the enhanced risk of future illness due to exposure and (2) regular medical testing for diseases from exposure. The Supreme Court of New Jersey found that he task of litigating hypothetical injuries would unreasonably strain the tort system (although it suggests that the state legislature could pass a remedy that allowed damages if toxic exposure caused a ``statistically significant incidence of disease''). On the second claim, it held for the plaintiffs.
\end{enumerate}

\subsection{Duty and Proximate Cause}

\subsubsection{\emph{Atlantic Coast Line R. Co. v. Daniels}}

\begin{enumerate}
    \item Cause in effect are infinite. An act is the proximate cause if it's close enough. Courts and juries have to rely on reason and common sense to judge whether a cause is proximate.
    \item Some sources, like the Restatement on Torts, prefer ``legal cause.''
    \item Proximate cause is a tool for protecting defendants.
\end{enumerate}

\subsubsection{\emph{Palsgraf v. The Long Island Railroad Company}}

\begin{enumerate}
    \item A railroad employee caused a passenger's package to fall. The package turned out to be full of fireworks. It exploded, causing a scale to break and injure the plaintiff.
    \item The trial court found negligence. The Court of Appeals here reversed.
    \item Cardozo: negligence requires the defendant to have a duty to the plaintiff. There must be a point in the chain of causation where an actor is no longer liable---otherwise, anybody who jostles someone in a crowd could be liable. To be negligent, the actor must have breached a reasonable standard of care. In this case, however, the railroad employee could not have known that the package was full of fireworks.
    \item Andrews, dissenting: The actor owes a duty of care to the public at large. Ultimately, proximate cause is about expediency, not logic, and judges must rely on common sense. In this case, the defendant's actions were a but-for cause of the plaintiff's injuries. It's not possible to say that plaintiff's injuries ``were not the proximate result of the negligence.''
\end{enumerate}

\subsubsection{Directness vs. Foreseeability: \emph{Overseas Tankship (U.K.) Ltd. v. Morts Dock \& Engineering Co. (The Wagon Mound) Privy Council}}

\begin{enumerate}
    \item The plaintiffs' ship, the \emph{Corrimel}, was moored for repairs. The appellants' ship, the \emph{Wagon Mound}, was moored nearby. The crew of the \emph{Wagon Mound} accidentally spilled a large amount of oil into the bay. They left soon after without cleaning up the oil.
    \item The plaintiff checked with the manager of the wharf where the \emph{Wagon Mound} was moored to see if the oil on the water was flammable. They agreed it was not. Soon after, a small drop of molten metal from the plaintiffs' worked ignited the oil, severely damaging the \emph{Corrimal} and the wharf.
    \item \emph{In re Polemis} dealt with another scenario involving fire and negligence. Although the fire was not a foreseeable consequence of the negligence, it was clear that the defendant's action was the direct cause, and the court held for the plaintiffs.
    \item The court here replaced the direct test from \emph{Polemis} with a foreseeability test.
    \item The defendants could not have foreseen a massive fire to be the result of their negligence. Ruling for the defendants.
    \item \emph{Kinsman}: foreseeability is a weaker requirement when the consequences are direct and the damage is of the same sort that was risked.\footnote{Casebook p. 258.}
\end{enumerate}

\subsubsection{Intervening Events: \emph{Thomas v. United States Soccer Fedn.}}

\begin{enumerate}
    \item The plaintiff suffered injuries when a soccer game turned violent. He sued the soccer federation for failing to provide a properly trained referee and failing to maintain a safe playing environment. The defendants moved for a summary judgment on the grounds that the alleged negligence was not the proximate cause. The court reasoned that when an intervening act occurs, liability will turn on whether the defendant should have foreseen the act as a consequence of the negligence. It reversed the lower courts and granted the motion for dismissal.
    \item ``Superseding intervening forces are those new forces which are extraordinarily unexpected.''\footnote{Casebook p. 261.}
    \item Intervening criminal acts are generally found to be unforeseeable and therefore superseding.
    \item ``Dependent'' intervening forces are results of the defendant's action (e.g., an ambulance driver's collision while rushing to the scene of the defendant's accident). ``Independent'' intervening forces do not have a causal connection to the defendant (e.g., a lightning bolt).
    \item ``...ultimately the determinative issue is whether or not the intervening force is extraordinarily unexpected.''\footnote{Casebook p. 263.}
\end{enumerate}

\subsubsection{\emph{Bigbee v. Pacific Telephone and Telegraph Co.}}

\begin{enumerate}
    \item Plaintiff was inside a telephone booth. He saw a car approaching, and he claims he tried to get out but couldn't. He alleges the telephone booth company was negligent in (1) its manufacture of the booth, which prevented his escape, and (2) its placement in proximity to a busy street, where damage from an oncoming car was foreseeable. The lower courts upheld a motion to dismiss. Here, the Supreme Court of California held that a jury could find that the danger was reasonably foreseeable. Reversed and remanded.
    \item Unlikely intervening events are often not found to be superseding events. For instance, if an owner leaves the keys in her car in a high crime area, she may be liable for the harm the car thief causes. (But generally, car owners are not responsible for the actions of car thieves.)
\end{enumerate}

\subsubsection{The Egg-Shell Plaintiff Rule: \emph{Steinhauser v. Hertz Corporation}}

\begin{enumerate}
    \item The plaintiff was involved in a car accident. She suffered no injuries, but the accident triggered serious schizophrenia. The court held that as long as there is a causal relationship between the small accident and the catastrophic result, the defendant can be held liable for the ``precipitating cause.'' The probability that the condition would have developed is not a defense, but it can be considered in fixing damages.
    \item The large injury from the small cause need not be foreseeable.
\end{enumerate}

\subsection{Proof of Negligence: Res Ipsa Loquitur}

\begin{enumerate}
    \item Res Ipsa Loquitur: ``the thing speaks for itself.''
    \item It usually has three requirements (with variations among jurisdictions):
    \begin{enumerate}
        \item The accident would not have occurred without negligence.
        \item The negligent act was within the actor's control.
        \item The plaintiff was not at fault.
    \end{enumerate}
    \item It's an expansion of the common sense cookie jar rule: if a parent returns to see a child next to a broken cookie jar, it's reasonable to infer that the child broke the cookie jar.
    \item Some courts have relaxed the requirement that the defendant must have had exclusive control of the accident.\footnote{Casebook p. 275 n. 4.}
    \item Some courts follow the \emph{Ybarra} rule, which expands the res ipsa loquitur doctrine to medical cases with multiple defendants, where multiple defendants did not have exclusive control of the accident and not all of them were necessarily negligent. It's an extension of the situation where a teacher punishes the entire class for breaking the goldfish bowl.
\end{enumerate}

\subsubsection{\emph{Krebs v. Corrigan}}

\begin{enumerate}
    \item The defendant inexplicably flew through the air and landed on the plaintiff's plexiglass sculpture, destroying it. The trial court granted the defendant's motion for a directed verdict.
    \item ``...human bodies do not generally go crashing into breakable personal propert,'' said the appellate court.
    \item Defendant argued (1) that res ipsa loquitur does not apply when the instrumentatality is a human body and (2) the doctrine does not apply because there was an eyewitness. The court rejected both of these arguments.
    \item The doctrine exists, the court reasoned, to deal with cases where only the defendant knows the details of the negligent act.
    \item The appellate court held that the evidence was suffficient to raise an inference of negligence, so it reversed the directed verdict for the defendants.
\end{enumerate}

\subsubsection{\emph{Ybarra v. Spangard}}

\begin{enumerate}
    \item The plaintiff underwent surgery for appendicitis. During the procedure, he suffered a shoulder injury that caused paralysis and muscle atrophy. The trial court entered a judgment of nonsuit for all defendants.
    \item The plaintiff argued that the doctrine of res ipsa loquitur should apply to the defendants, all of whom were involved at different stages of his medical care.
    \item The defendants argue that the plaintiff cannot show that any single defendant caused the injury.
    \item As in \emph{Krebs}, the court noted that the purpose of the res ipsa loquitur doctrine is to address cases where the circumstances of the negligence were unknown to the plaintiff (in this case, because he was unconscious).
    \item Classic examples where res ipsa loquitur would apply: passenger sitting awake in a train car at the time of a collision; person walking down the street and hit by a falling object.
    \item These sorts of cases ``raise the inference of negligence, and call upon the defendant to explain the unusual result.''\footnote{Casebook p. 279.}
    \item It could be found in this case that some of the defendants are liable and others are absolved. But that should not preclude the application of res ipsa loquitur. It would not be reasonable to ask the plaintiff to identify which of the individual defendants were responsible for the harm.
    \item The defedants' argument would undermind the rights of patients to recover for injuries suffered while unconscious.
    \item Judgment of nonsuit was reversed.
\end{enumerate}

\subsection{Limitations on Duty}

\subsubsection{Failure to Act: \emph{L. S. Ayres \& Co. v. Hicks}}

\begin{enumerate}
    \item todo
\end{enumerate}

\subsubsection{\emph{Miller v. Arnal Corp.}}

\begin{enumerate}
    \item todo
\end{enumerate}

\subsubsection{\emph{Wells v. Hickman}}

\begin{enumerate}
    \item todo
\end{enumerate}

\subsubsection{\emph{Tarasoff v. The Regents of the University of California}}

\begin{enumerate}
    \item todo
\end{enumerate}

\subsubsection{\emph{Davidson v. City of Westminster}}

\begin{enumerate}
    \item todo
\end{enumerate}

\subsubsection{Mental Distress: \emph{Thing v. La Chusa}}

\begin{enumerate}
    \item todo
\end{enumerate}

\subsubsection{\emph{Potter v. Firestone Tire and Rubber Co.}}

\begin{enumerate}
    \item todo
\end{enumerate}

\subsubsection{C.C.P. § 377}

\begin{enumerate}
    \item todo
\end{enumerate}

\subsubsection{Wrongful Death and Survival Actions: \emph{Gary v. Schwartz}}

\begin{enumerate}
    \item todo
\end{enumerate}

\subsubsection{\emph{Selders v. Armentrout}}

\begin{enumerate}
    \item todo
\end{enumerate}

\subsubsection{\emph{Compania Dominicana de Aviacion v. Knapp}}

\begin{enumerate}
    \item todo
\end{enumerate}

\subsubsection{Loss of Consortium and Society: \emph{Borer v. American Airlines, Inc.}}

\begin{enumerate}
    \item todo
\end{enumerate}

\subsubsection{Wrongful Life and Wrongful Birth: \emph{Turpin v. Sortini}}

\begin{enumerate}
    \item todo
\end{enumerate}

\subsubsection{\emph{Rowland v. Christian}}

\begin{enumerate}
    \item todo
\end{enumerate}

\subsubsection{\emph{Ann M. v. Pacific Plaza Shopping Center}}

\begin{enumerate}
    \item todo
\end{enumerate}

\subsubsection{\emph{Wiener v. Southeast Childcare}}

\begin{enumerate}
    \item todo
\end{enumerate}

\subsubsection{Negligent Misrepresentation: \emph{Bily v. Arthur Young \& Co.}}

\begin{enumerate}
    \item todo
\end{enumerate}

\subsubsection{Economic Loss: \emph{J'Aire Corp. v. Gregory}}

\begin{enumerate}
    \item todo
\end{enumerate}

\subsubsection{\emph{Li v. Yellow Cab Co.}}

\begin{enumerate}
    \item todo
\end{enumerate}

\subsubsection{Assumption of Risk: \emph{Murphy v. Steeplechase Amusement Co.}}

\begin{enumerate}
    \item todo
\end{enumerate}

\subsubsection{\emph{Rush v. Commercial Realty Co.}}

\begin{enumerate}
    \item todo
\end{enumerate}

\subsubsection{\emph{Emmette L. Barran, III v. Kappa Alpha Order, Inc.}}

\begin{enumerate}
    \item todo
\end{enumerate}

\subsubsection{\emph{Knight v. Jewett}}

\begin{enumerate}
    \item todo
\end{enumerate}

\subsubsection{\emph{Priebe v. Nelson}}

\begin{enumerate}
    \item todo
\end{enumerate}

\subsubsection{\emph{Shin v. Ahn}}

\begin{enumerate}
    \item todo
\end{enumerate}

\subsubsection{\emph{Metcalfe v. County of San Joaquin}}

\begin{enumerate}
    \item 
\end{enumerate}

\subsection{Joint and Several Liability}

\subsubsection{\emph{American Motorcycle Association v. Superior Court}}

\begin{enumerate}
    \item todo
\end{enumerate}

\subsubsection{Fair Responsibility Act of 1986}

\begin{enumerate}
    \item todo
\end{enumerate}
