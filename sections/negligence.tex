\section{Negligence}

\subsection{Overview}

\begin{enumerate}
    \item ``The failure to exercise the standard of care that a \textbf{reasonably prudent person} would have exercised in a similar situation.''\footnote{Black's Law.}
    \item The standard of care is objective.
    \item Negligence has six factors:
    \begin{enumerate}
        \item Duty.
        \item Standard of care.
        \item Breach of duty.
        \item Cause-in-fact.
        \item Proximate cause.
        \item Damages.
    \end{enumerate}
    \item The Hand Formula is Judge Learned Hand's test for determining negligence.
    \begin{enumerate}
        \item B = Burden of precautions necessary to prevent an accident.
        \item P = Probability that an accident will occur.
        \item L = Magnitude of the loss if the accident occurs.
        \item Negligence exists if B\textless PL---i.e., if the burden of precautions is less than the harm multiplied by the probability of occurrence.
    \end{enumerate}
\end{enumerate}

\subsubsection{\emph{Pitre v. Employers Liability Assurance Corporation}}

\begin{enumerate}
    \item The plaintiffs' son died when a patron at a carnival game was winding up a pitch and hit him in the head. The determining factor, the court reasoned, is how a ``reasonably prudent individual'' would have acted or what precautions he would have taken under similar circumstances. Negligence occurs only if the danger is both foreseeable and unreasonable. The court held that the danger was foreseeable but not unreasonable, and therefore there was no negligence.
\end{enumerate}

\subsubsection{\emph{United States Fidelity \& Guaranty Company v. Plovidba}}

\begin{enumerate}
    \item Inside a dark room, the hatch to a cargo hold on a ship was left open. A longshoreman fell through it and died. The court uses the Hand Formula, reasoning that B was relatively small (it would have been easy to close the hatch or leave a light turned on) and L was high (the victim died). P, however, was very small. There was no reason for the longshoreman to enter the hold. In fact, he was probably there to steal liquor, and the evidence suggests he knew the hatch was open and tried to skirt around it. The shipowner was therefore not liable for negligence.
\end{enumerate}

% \subsection{Standard of Conduct}
% 
% \subsubsection{\emph{Cordas v. Peerless Transp. Co.}}
% 
% \begin{enumerate}
%     \item TODO
% \end{enumerate}
% 
% \subsubsection{\emph{Breunig v. American Family Insurance Company}}
% 
% \begin{enumerate}
%     \item TODO
% \end{enumerate}
% 
% \subsubsection{\emph{Neumann v. Shlansky}}
% 
% \begin{enumerate}
%     \item TODO
% \end{enumerate}
% 
% \subsubsection{\emph{Melville v. Southward}}
% 
% \begin{enumerate}
%     \item TODO
% \end{enumerate}
% 
% \subsubsection{\emph{Cobbs v. Grant}}
% 
% \begin{enumerate}
%     \item TODO
% \end{enumerate}

\subsection{Rules of Law}

\begin{enumerate}
    \item Juries typically determine what constitutes reasonable conduct under the circumstances. Judges, however, ill sometimes establish a rule for what constitutes negligent conduct under particular circumstances.
    \item For instance, in \emph{Baltimore \& Ohio R.R.}, Justice Holmes established a rule requiring drivers to get out of their cars and examine railroad crossings.
\end{enumerate}

\subsubsection{\emph{Akins v. Glens Falls City School District}}

\begin{enumerate}
    \item A foul ball injured a spectator at a baseball game. She sued the ballpark's owners, the local school district, for negligence. The trial court helf in favor of the plaintiff. The appellate court affirmed, establishing a specific rule for ballpark backstops. The dissent argued that such a rule ``robs the jury'' of the ability to consider important circumstances and locks the law in ``a position that is certain to become outdated.''
\end{enumerate}

\subsection{Negligence Per Se}

\begin{enumerate}
    \item Courts will occasionally adopt a standard of conduct from criminal statutes, administrative regulations, or other legislative enactments.
    \item The jury must still determine whether the defendant transgressed the legislative standard.
\end{enumerate}

\subsubsection{\emph{Wawanesa Mutual Insurance Co. v. Matlock}}

\begin{enumerate}
    \item A minor bought cigarettes for another minor, who later dropped the cigarette and caused a fire that led to significant property damage. The insurer sued the first minor's father, and the trial court found for the insurer. The appellate court overturned the ruling. It argued that the statute in question was meant to protect against the health hazards of tobacco, not the fire hazard, and therefore cannot be used to establish a standard of conduct in this case.
\end{enumerate}

\subsubsection{\emph{Stachniewicz v. Mar-Cam Corporation}}

\begin{enumerate}
    \item A patron injured in bar brawl sued the bar owner. The plaintiff relied on (1) an Oregon statute which prohibits giving alcohol to an intoxicated person and (2) an Oregon regulation that prevents bar owners from allowing disorderly conduct on their premised. The trial court found for the defendant. The appellate court overturned, reasoning that (1) the statute is inapplicable because the brawler was already drunk when he arrived, so there is no way to tell if another drink caused the brawl, but (2) the regulation was intended specifically to protect customers from injury, and therefore can be an appropriate standard for negligence in this case.
\end{enumerate}

\subsubsection{\emph{Gorris v. Scott}}

\begin{enumerate}
    \item Several sheep on a ship were swept overboard. The plaintiff sued the shipowner, arguing that the Contagious Diseases (Animals) Act required the shipowner to enclose the sheep in pens of certain dimensions, which the shipowner failed to do. The court found in favor of the shipowner, reasoning that the Act was intended to prevent the spread of contagious diseases, not to prevent sheep from falling overboard.
\end{enumerate}
