\section{Negligence}

\subsection{Overview}

\begin{enumerate}
    \item ``The failure to exercise the standard of care that a \textbf{reasonably prudent person} would have exercised in a similar situation.''\footnote{Black's Law.}
<<<<<<< HEAD
    \item Negligence requires proof that the defendant acted unreasonably.
=======
>>>>>>> f06a33479f74dd68c6a748e1da4331f6fa68db32
    \item The standard of care is objective.
    \item Negligence has six factors:
    \begin{enumerate}
        \item Duty.
        \item Standard of care.
        \item Breach of duty.
        \item Cause-in-fact.
        \item Proximate cause.
        \item Damages.
    \end{enumerate}
    \item The Hand Formula is Judge Learned Hand's test for determining negligence. It works best in scenarios where the actor takes a calculated risk (e.g., business decisions). It is less useful in cases where the actor was simply not paying attention.
    \begin{enumerate}
        \item B = Burden of precautions necessary to prevent an accident.
        \item P = Probability that an accident will occur.
        \item L = Magnitude of the loss if the accident occurs.
        \item Negligence exists if B\textless PL---i.e., if the burden of precautions is less than the harm multiplied by the probability of occurrence.
    \end{enumerate}
\end{enumerate}

\subsubsection{\emph{Pitre v. Employers Liability Assurance Corporation}}

\begin{enumerate}
    \item The plaintiffs' son died when a patron at a carnival game was winding up a pitch and hit him in the head. The trial court found in favor of the plaintiffs. The determining factor, the Court of Appeal reasoned, is how a ``reasonably prudent individual'' would have acted or what precautions he would have taken under similar circumstances. \textbf{Negligence occurs only if the danger is both foreseeable and unreasonable.} The court held that the danger was foreseeable but not unreasonable, and therefore there was no negligence.
\end{enumerate}

\subsubsection{\emph{United States Fidelity \& Guaranty Company v. Plovidba}}

\begin{enumerate}
    \item Inside a dark room, the hatch to a cargo hold on a ship was left open. A longshoreman fell through it and died. The trial court found for the defendant. Here, Richard Posner writing for the Seventh Circit applies the Hand Formula, reasoning that B was relatively small (it would have been easy to close the hatch or leave a light turned on) and L was high (the victim died). P, however, was very small. There was no reason for the longshoreman to enter the hold. In fact, he was probably there to steal liquor, and the evidence suggests he knew the hatch was open and tried to skirt around it. The shipowner was therefore not liable for negligence.
\end{enumerate}

\subsection{Standard of Conduct}

\subsubsection{\emph{Cordas v. Peerless Transp. Co.}}

\begin{enumerate}
    \item A man was mugged at gunpoint by two other men in New York City. He chased after them. One of the muggers jumped into a taxi, held the driver at gunpoint, and told him to drive. While the cab was in motion, the driver jumped out, and a few seconds later, so did the hijacker. The cab crashed into a sidewalk and injured the defendants. The trial court held that the driver was not negligent because he acted as a reasonable person would act under similar circumstances.
    \item Courts are divided on the question of whether juries should receive special instructions regarding negligence claims in emergency circumstances. On the one hand, it is redundant to reiterate that a defendant must be held to the standard of what a reasonable person would do in a similar emergency situation. Others claim it helps clarify the standard.
\end{enumerate}

\subsubsection{\emph{Breunig v. American Family Insurance Company}}

\begin{enumerate}
    \item A schizophrenic woman had a psychotic episode while driving her car. The question was whether she had foreknowledge of her susceptibility to such attacks. The general rule is that insanity or another mental deficiency does not limit liability for negligence. In other words, \textbf{insane people are held to the reasonable person standard}. The court here notes that may be too harsh to exclude the insanity defense when a driver is suddenly overcome without warning. The Supreme Court agrees with the lower courts that the defendant did have the necessary foreknowledge, and held for the plaintiff. 
\end{enumerate}

\subsubsection{\emph{Neumann v. Shlansky}}

\begin{enumerate}
    \item An eleven-year-old hit a golf ball that struck the defendant in the knee, causing serious injury. Generally, children are held to the standard of a \textbf{reasonable person of like age, intelligence, and experience under the circumstances.} In this case, however, the child was engaging in an ``adult activity,'' and therefore the court held him to the adult reasonable person standard.
\end{enumerate}

\subsubsection{\emph{Melville v. Southward}}

\begin{enumerate}
    \item The defendant, a podiatrist, operated on the plaintiff's foot. The plaintiff sued for malpractice, and introduced the testimony of an orthopedist, who questioned the necessity and sanitation of the operation. The question is whether the orthopedist, a practitioner from a different school of medicine, should have been allowed to testify about the standard of care in podiatry. The trial court allowed the orthopedist to testify. The Supreme Court of Colorado here agreed with the appellate court that the testimony should not have been allowed because it was ``nothing more than an expression of opinion that that the general practice of podiatry did not meet the standard of care observed by an orthopedic surgeon.'' It remanded the case for a new trial.
    \item In malpractice cases, the ``competent professional'' standard replaces the ``reasonable person'' standard.
    \item There is disagreement about whether doctors in rural areas should be held to different standards than urban doctors.
    \item Medical specialists in the same geographic region are often reluctant to testify against each other---a ``conspiracy of silence.''
    \item In a limited range of cases, a jury of laypeople can determine whether a practice met an acceptable standard of care.
\end{enumerate}

\subsubsection{\emph{Cobbs v. Grant}}

\begin{enumerate}
    \item Doctors are required to obtain \textbf{informed consent} from patients. The plaintiff here sued a doctor who operated on a stomach ulcer but did not discuss the surgery's inherent risks. Complications developed, another operation was required, more complications developed, and so on. The plaintiff argued that (1) the doctor acted negligently in the performance of the surgery (which the jury found in favor of the plaintiff) and (2) that the doctor failed to obtain informed consent. The Supreme Court of California here notes that courts are divided as to whether this type of tort should be deemed a \textbf{battery or negligence}. The court aligned itself with a ``majority trend'' that advocates reserving battery for cases where a doctor performs an operation without the patient's consent. Generally, physicians are required to tell patients about major risks (but not every minor risk) and obtain the patient's consent. In this case, the court finds that there is not enough evidence to show that the doctor acted negligently, and the case is remanded for a new trial.
    \item \textbf{Failure to obtain informed consent can subject a physician to negligence liability.} Unless the physician misrepresents the entire procedure, most courts will not characterize the behavior as intentional battery.
\end{enumerate}

\subsection{Rules of Law}

\begin{enumerate}
    \item Juries typically determine what constitutes reasonable conduct under the circumstances. Judges, however, ill sometimes establish a rule for what constitutes negligent conduct under particular circumstances.
    \item For instance, in \emph{Baltimore \& Ohio R.R.}, Justice Holmes established a rule requiring drivers to get out of their cars and examine railroad crossings.
\end{enumerate}

\subsubsection{\emph{Akins v. Glens Falls City School District}}

\begin{enumerate}
    \item A foul ball injured a spectator at a baseball game. She sued the ballpark's owners, the local school district, for negligence. The trial court helf in favor of the plaintiff. The appellate court affirmed, establishing a specific rule for ballpark backstops. The dissent argued that such a rule ``robs the jury'' of the ability to consider important circumstances and locks the law in ``a position that is certain to become outdated.''
\end{enumerate}

\subsection{Negligence Per Se}

\begin{enumerate}
    \item Courts will occasionally adopt a standard of conduct from criminal statutes, administrative regulations, or other legislative enactments.
    \item The jury must still determine whether the defendant transgressed the legislative standard.
\end{enumerate}

\subsubsection{\emph{Wawanesa Mutual Insurance Co. v. Matlock}}

\begin{enumerate}
    \item A minor bought cigarettes for another minor, who later dropped the cigarette and caused a fire that led to significant property damage. The insurer sued the first minor's father, and the trial court found for the insurer. The appellate court overturned the ruling. It argued that the statute in question was meant to protect against the health hazards of tobacco, not the fire hazard, and therefore cannot be used to establish a standard of conduct in this case.
\end{enumerate}

\subsubsection{\emph{Stachniewicz v. Mar-Cam Corporation}}

\begin{enumerate}
    \item A patron injured in bar brawl sued the bar owner. The plaintiff relied on (1) an Oregon statute which prohibits giving alcohol to an intoxicated person and (2) an Oregon regulation that prevents bar owners from allowing disorderly conduct on their premised. The trial court found for the defendant. The appellate court overturned, reasoning that (1) the statute is inapplicable because the brawler was already drunk when he arrived, so there is no way to tell if another drink caused the brawl, but (2) the regulation was intended specifically to protect customers from injury, and therefore can be an appropriate standard for negligence in this case.
\end{enumerate}

\subsubsection{\emph{Gorris v. Scott}}

\begin{enumerate}
    \item Several sheep on a ship were swept overboard. The plaintiff sued the shipowner, arguing that the Contagious Diseases (Animals) Act required the shipowner to enclose the sheep in pens of certain dimensions, which the shipowner failed to do. The court found in favor of the shipowner, reasoning that the Act was intended to prevent the spread of contagious diseases, not to prevent sheep from falling overboard.
\end{enumerate}

\subsection{Cause in Fact}

\begin{enumerate}
    \item Plaintiff must prove that the harm would not have occurred \textbf{but for} the defendant's actions.
    \item If there are multiple causes of harm, each can be but-for causes as long as the harm would not have occurred without it.
    \item If there are multiple causes of harm, but none alone is a but-for cause, courts can use the substantial factor test. See \emph{Northington} below.
    \item \textbf{Proximate cause} removes liability when ``the connection between the plaintiff's harm and defendant's liability is unforeseeable or so attenuated that public policy preclused liability.''\footnote{Casebook p. 206.}
\end{enumerate}

\subsubsection{\emph{East Texas Theatres, Inc. v. Rutledge}}

\begin{enumerate}
    \item At the defendant's movie theater, somebody threw a bottle from a balcony which struck and injured the plaintiff. The jury found the theater liable because it negligently failed to remove ``rowdy persons'' from the balcony during the game, and the Texas appellate court affirmed. The Texas Supreme Court clarified that proximate cause has two elements: (1) cause-in-fact and (2) foreseeability. The court held that the prosecution failed to show that the injuries would have occurred but for the removal of the ``rowdy persons.'' It reversed the lower court's ruling and held for the defendants.
\end{enumerate}

\subsubsection{\emph{Anderson v. Minneapolis, St. P. \& S. S. M. Ry. Co.}}

\begin{enumerate}
    \item A spark from a railroad started a fire in a bog on one side of the defendant's property. Another unrelated fire was burning on the other side. The fire from the railroad destroyed the defendant's property, and a few days later it joined with the other fire to make one big fire. The railroad argues that it cannot be held liable because the defendant's house would have been destroyed by the other fire anyway. The trial court refused to instruct the jury to follow a rule from an earlier case, \emph{Cook}, which held that there is no liability when two fires jointly destroy property. On this basis, the trial court found for the plaintiff. The railroad requested a motion for judgment notwithstanding the verdict, which was denied. On appeal, the Supreme Court of Minnesota held that the trial court was correct in refusing to apply the \emph{Cook} rule and found for the plaintiffs.
    \item \textbf{If two independent fires join to cause property damage, there is joint liability, even if neither alone is a but-for cause}
\end{enumerate}

\subsubsection{\emph{Northington v. Marin}}

\begin{enumerate}
    \item The plaintiff, a prison inmate, sued the defendant, a prison guard, for circulating rumors that labeled him a snitch and caused other inmates to assault him. Other guards had spread the same rumors. The trial court found that although the defendant's action was not a but-for cause (since the harm would have occurred without his action), his contribution to the harm was nonetheless a \textbf{substantial factor}. The Tenth Circuit affirmed: ``Multiple tortfeasors who concurrently cause an indivisible injury are jointly and severally liable; each can be held liable for the entire injury.''
\end{enumerate}

\subsubsection{\emph{Herskovitz v. Group Health Cooperative of Puget Sound}}

\begin{enumerate}
    \item The plaintiff brought the action on behalf of her husband, a deceased lung cancer patient, against a doctor that negligently failed to diagnose the patient's lung cancer on his first visit, proximately causing his chance of survival to drop from 39 percent to 25 percent. Neither fact was in dispute. The defendant argued that the plaintiff must prove that the patient ``probably'' would have lived but for the negligence---that is, without the doctor's negligence, the patient's chance of survival must have been more than 50 percent. The trial court granted summary judgment for the defendanto n this argument. The Supreme Court of Washington reversed, arguing that any other decision would mean a ``blanket release'' for doctors' negligence any time the patient's chance of survival was less than 50 percent. The court reasoned that if a defendant's acts have \emph{increased the risk} of harm to the plaintiff, a jury should decide whether the increased risk actually caused the harm in question.
\end{enumerate}

\subsubsection{\emph{Summers v. Tice}}

\begin{enumerate}
    \item
\end{enumerate}

\subsubsection{\emph{Sindell v. Abbott Laboratories}}

\begin{enumerate}
    \item
\end{enumerate}

\subsubsection{\emph{Ayers v. Township of Jackson}}

\begin{enumerate}
    \item
\end{enumerate}
