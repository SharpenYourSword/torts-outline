\section{Battery}

\begin{enumerate}
    \item Battery requires intent to cause \textbf{harmful or offensive contact} and that harmful or offensive contact directly or indirectly results.
    \item Is battery actionable for very small harms? \textbf{\emph{Leichtman v. WLW Jacor Communications, Inc.}}: A cigar smoker blew smoke in the face of an anti-smoking advocate. The court finds that ``No matter how trivial the incident, a battery is actionable...'' But it rejects the ``smoker's battery'' (which imposes liability if there is substantial certainty that second-hand smoke will touch a nonsmoker).
    \item Does compliance with safety standards affect liability for intentional torts? Can radiation constitute contact? \textbf{\emph{Bohrmann v. Maine Yankee Atomic Power Co.}}: University of Southern Maine students took a tour of a nuclear power plant. Plaintiffs allege the company knew a flushing procedure would release radioactive gases during the tour, and that tour guides knowingly took students through plumes of unfiltered radioactive gases. Plaintiffs also allege the company falsely told them they had not been exposed to ``bad'' radiation. The court holds that compliance with federal safety standards does not affect the defendant's liability for intentional acts.
\end{enumerate}
