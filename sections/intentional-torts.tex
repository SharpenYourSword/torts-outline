\section{Intentional Torts}

\subsection{Intent}

\begin{enumerate}
    \item Intent requires \textbf{desire} or \textbf{substantial certainty}.
    \item Reckless behavior can sometimes suffice for intent---e.g., IIED, 
    intentional misrepresentation. Substantial certainty is a higher threshold 
    than recklessness.
    \item ``...the law of torts is not criminal law and does not condemn, but 
    only shifts the burdens of economic loss.''\footnote{\emph{Understanding 
    Torts}, p. 6.}
    \item Intentional torts are generally excepted from worker's compensation 
    immunity.
    \item In most jurisdictions, you can't insure against intentional torts.
    \item Restatement (Second) blends purpose and knowledge (i.e., substantial 
    certainty) into one rule. Restatement (Third) proposes separating them 
    into two distinct rules, since limiting liability to ``purpose'' can have 
    consequences in areas like workplace litigation.
    \item Inadvertent results of actions are not intentional. (But, mistakes 
    usually constitute intent---see below. Reasonable mistakes are not allowed 
    except in self defense.)
    \item The substantial certainty test significantly expands the use of 
    intentional torts in workplace and environmental litigation.
    \item A few jurisdictions have rejected the substantial certainty 
    rule.\footnote{Casebook p. 6.}
\end{enumerate}

\subsubsection{Infancy: \emph{Garratt v. Dailey}}

Infancy is not a defense to intentional torts.

\begin{enumerate}
    \item A five year old moved a chair from the place where the plaintiff was 
    about to sit. The plaintiff fell and fractured her hip.
    \item The court held that the plaintiff's battery claim required proof 
    that the defendant intended to cause contact that was not consensual or 
    otherwise privileged. Restatement (Second) indicates that intent exists if 
    the actor is \textbf{substantially certain} that the harmful contact 
    \textbf{will} (not might) occur.
    \item The court found that it was unclear whether the defendant was 
    substantially certain that the result would occur. Remanded to the trial 
    court for clarification.
\end{enumerate}

\subsubsection{Insanity: \emph{Williams v. Kearbey}}

Insanity is not a defense to intentional torts.

\begin{enumerate}
    \item Kearbey, a minor, shot up a school and claimed insanity.
    \item The court held that Kearbey intended to commit the action, even if 
    his motivation was irrational, and was therefore liable.
\end{enumerate}

\subsection{Battery}

\begin{enumerate}
    \item Battery requires \textbf{intent to cause harmful or offensive 
    contact} and that harmful or offensive contact directly or indirectly 
    results.
    \begin{enumerate}
        \item ``Unpermitted'' touching can be enough---see \emph{White v.  
        University of Idaho}, where a piano teacher touched a student's back 
        and caused significant injury.
        \item Any touching in anger can also be enough.
    \end{enumerate}
    \item The proposed Third Restatement would limit intent liability based on 
    substantial certainty to small, localized groups of people. So, for 
    instance, tobacco companies would not be liable for second-hand smoke 
    damages.
\end{enumerate}

\subsubsection{Very Small Harms: \emph{Leichtman v. WLW Jacor Commc'ns, Inc.}}

Minuscule contact can constitute battery, though recovery for damages will 
likely be minimal.

\begin{enumerate}
    \item A cigar smoker and talkshow host employed by WLW Jacor blew smoke in 
    the face of Leichtman, an anti-smoking advocate.
    \item The court found that ``[n]o matter how trivial the incident, a 
    battery is actionable...''\footnote{Casebook p. 11.} But it rejected the 
    ``smoker's battery,'' which imposes liability if there is substantial 
    certainty that second-hand smoke will touch a nonsmoker.\footnote{Casebook 
    p. 16 n. 6.}
\end{enumerate}

\subsubsection{Compliance with Safety Standards: \emph{Bohrmann v. Maine 
Yankee Atomic Power Co.}}

Compliance with safety standards has no bearing on liability for intentional 
torts.

\begin{enumerate}
    \item The plaintiffs, several University of Southern Maine students, took 
    a tour of a nuclear power plant. They alleged the power company knew a 
    flushing procedure would release radioactive gases during the tour and 
    that tour guides knowingly took students through plumes of unfiltered 
    radioactive gases. They also allege the company falsely told them they had 
    not been exposed to ``bad'' radiation.
    \item The court held that compliance with federal safety standards did not 
    affect the power company's liability for its intentional acts.
\end{enumerate}

\subsection{Assault}

\begin{enumerate}
    \item Assault is the threat or use of force on another that causes that 
    person to have a \textbf{reasonable apprehension of imminent harmful or 
    offensive contact}.
    \item Restatement (Second) does not require apprehension to be 
    ``reasonable,'' but most courts do.
    \item \textbf{Assault in torts is different than assault in criminal law.} 
    The criminal law definition requires an attempt to inflict harmful or 
    offensive contact, but it does not require the victim's 
    perception.\footnote{Casebook p. 21.}
\end{enumerate}

\subsubsection{Liability without Physical Harm: \emph{I de S et Ux v. W de S}}

Damages be awarded if physical harm did not occur. 

\begin{enumerate}
    \item The Defendant tried to buy wine from the plaintiff. He beat on the 
    door with a hatchet. When the plaintiff's wife asked him to stop, he tried 
    to hit her with the hatchet but did not hit her.
    \item The court ruled that the plaintiff was entitled to damages even 
    though no physical harm was done.
\end{enumerate}
    
\subsubsection{Forward Looking Verbal Threats: \emph{Castro v. Local 1199, 
Nat'l Health \& Human Servs. Emps. Union}}

To constitute assault, verbal threats must accompany ``circumstances inducing 
a reasonable apprehension of bodily harm.''

\begin{enumerate}
    \item Plaintiff had asthma, which prevented her from working in extremely 
    hot or cold situations. After a disability leave, she attended a meeting 
    where she didn't receive her usual assignment. She asked what was going 
    on, and Frankel (another employee) replied, ``if I was you, I would take 
    whatever they give me, because you could lose more than your job.'' When 
    asked he was threatening her life, Frankel said, ``Take it any way you 
    want.''\footnote{Casebook p. 18.}
    \item The court held that verbal threats, without ``circumstances inducing 
    a reasonable apprehension of bodily harm,'' do not constitute an assault. 
    Here, the threat was ``forward-looking'' and did not suggest imminent 
    harm.\footnote{Casebook p. 19.} The court granted the defendant's motion 
    for summary judgment.
    \item Prior behavior can furnish the necessary attendant circumstances.  
    See \emph{Campbell v. Kansas State University}, where a university head 
    said ``he felt like hitting his assisstant on the buttocks, after he had 
    already slapped her on the buttocks,'' which the court held to be 
    assault.\footnote{Casebook p. 20.}
\end{enumerate}

\subsection{Transferred Intent}

\begin{enumerate}
    \item Generally, \textbf{intent towards anyone for any intentional tort is 
    intent towards anyone else for any other tort}.
    \item Historically, transferred intent means that intent to commit any of 
    the five traditional torts (battery, assault, false imprisonment, trespass 
    to land, trespass to chattel---because these are torts where you would 
    file a writ of trespass in old English contexts) can constitute the 
    necessary intent to commit any of the other five.
    \item Transferred intent is a legal fiction.
    \item Restatement (Second) only incorporates transferred intent for 
    assault and battery.
\end{enumerate}

\subsubsection{\emph{Alteiri v. Colasso}}

The intended target doesn't matter as long as the defendant intended to commit 
the act.

\begin{enumerate}
    \item Colasso threw a rock that hit Alteiri in the eye, but he intended to 
    scare somebody else. He did not intend to hit anyone, and he did not throw 
    the rock recklessly.
    \item The court held that there was no error in the jury's verdict for 
    willful battery.
\end{enumerate}

\subsection{Mistake Doctrine}

\begin{enumerate}
    \item \textbf{Mistake is not a defense to intentional torts}, even if the 
    mistake was reasonable.
    \item \emph{Ranson v. Kittner}: the defendant was liable for shooting a 
    dog, even though he believed it was a wolf.
    \item However, reasonable mistakes are usually permitted in self defense.
\end{enumerate}

\subsection{False Imprisonment}

\begin{enumerate}
    \item False imprisonment is \textbf{intentional, unlawful, and unconsented 
    restraint}.
    \item According to the Restatement (Second), confinement may be caused by:
    \begin{enumerate}
        \item Physical barriers.
        \item Force or threat of immediate force.
        \item Omissions where there is a duty to act.
        \item False arrest.
    \end{enumerate}
    \item Victim must be confined in a bounded area (e.g., if movement is 
    allowed in any direction, even if it's not the desired direction, false 
    imprisonment did not occur).
    \item The victim must usually be conscious of confinement, but not always 
    (e.g., infant abduction).
    \item False imprisonment usually does not recognize highly coercive but 
    non-physical threats (e.g., economic retaliation).
    \item Lawful restraint does not constitute false imprisonment
    \item \emph{Shopkeeper's privilege}: Shopkeepers can detain suspected 
    shoplifters.
    \item ``[A] form of false imprisonment whereby the improper assertion of 
    legal authority can unlawfully restrain a victim.''\footnote{Casebook p.  
    38 n. 1.}
\end{enumerate}

\subsubsection{Implied Threat of Physical Restraint: \emph{Dupler v. Seubert}}

\begin{enumerate}
    \item Dupler was fired from her job. Her superiors, including Seubert, 
    kept her against her will in a 1.5-hour meeting. Dupler claimed that 
    Seubert and the other defendant screamed and shouted at her.
    \item The trial jury found Seubert liable for false imprisonment and 
    awarded damages of \$7,500. The trial judge offered a remittitur of \$500, 
    and Seubert appealed. The Supreme Court of Wisconsin affirmed the order, 
    holding that false imprisonment occurred when Dupler was held against her 
    will after her hours of employment had ended at 5 PM (in contrast to 
    \emph{Weiler v. Herzfeld-Phillipson}, where the imprisonment occurred 
    during work hours).
\end{enumerate}

\subsection{Malicious Prosecution}

\begin{enumerate}
    \item Restatement (Second) requires:
    \begin{enumerate}
        \item Initiation of proceedings \textbf{without probable cause and for 
        a purpose other than bringing the offender to justice}.
        \item \textbf{Termination of the proceedings in favor of the 
        accused}---so, a defendant who is sued and loses can't claim malicious 
        prosecution (i.e., the defendant must have been exonerated to have a 
        cause of action for malicious prosecution).\footnote{Understanding 
        Torts p. 48.}
    \end{enumerate}
    \item Anti-SLAPP statutes also help prevent against frivolous litigation.
    \item Some jurisdictions recognize malicious prosecution only in criminal 
    contexts, with the parallel civil tort ``wrongful institution of civil 
    proceedings.''\footnote{Casebook p. 39 n. 2.}
    \item The ``American Rule'' dictates that the loser in a suit doesn't have 
    to pay the winner's legal fees (in contrast to the ``British Rule'').
\end{enumerate}

\subsubsection{Distinguishing Abuse of Process and Malicious Prosecution: 
\emph{Maniaci v. Marquette Univ.}}

There can be no malicious prosecution without malice.

\begin{enumerate}
    \item Saralee Maniaci, a student at Marquette University, decided to leave 
    the school. She got her father's permission.
    \item School administrators tried to persuade her not to leave. When they 
    were unsuccessful, they requested that the Milwaukee police bring papers 
    for temporary emergency detention in a mental hospital for people 
    considered ``irresponsible and dangerous.'' The school physician, the Dean 
    of Women, and a registered nurse signed the application for temporary 
    custody. Maniaci was held for a night until her father demanded her 
    release.
    \item She and her father filed suit on multiple charges, all of which were 
    dismissed except false imprisonment. The jury assessed compensatory and 
    punitive damages, which the court reduced on motions after the verdict.
    \item On appeal, the defendants argued that the only legitimate cause of 
    action was \textbf{malicious prosecution}, that the evidence was 
    insufficient to prove malicious prosecution, and that the damages were 
    excessive.
    \item The appellate court agreed that there was no cause of action for 
    false imprisonment because the restraint was ``lawful.'' It did not find a 
    cause of action for malicious prosecution because there was no malice 
    since the defendants ``had a genuine concern for the plaintiff's 
    welfare.'' \item However, the court found support---``skeletally at 
    least''---for a cause of action for abuse of process. The defendants did 
    not have serious concerns about Maniaci's mental health. Rather, their 
    purpose was to restrain her until her parents had been notified of her 
    decision to leave school, and had either given their permission or 
    directed Saralee to stay in school.
    \item Reversed.  \end{enumerate}

\subsection{Abuse of Process}

\begin{enumerate}
    \item Abuse of process is the \textbf{misuse of legal process for an 
    ulterior purpose}.
    \item Unlike malicious prosecution, it does not require termination of the 
    legal process in favor of the one bringing the complaint (or even 
    termination at all).
    \item See \emph{Maniaci}, above.
\end{enumerate}

\subsection{Intentional Infliction of Emotional Distress}

\begin{enumerate}
    \item ``Intentional infliction of emotional distress \textbf{occurs when 
    the defendant, through extreme and outrageous conduct, intentionally or 
    recklessly causes the victim severe emotional 
    distress.}''\footnote{Casebook p. 44 n. 1.}
    \item It can overlap with other torts---e.g., wrongful termination, 
    sexual/racial harassment.
    \item It is not a historic tort, but a product of the 20th century. The 
    torts (above) all have rigid factors. Courts invented intentional 
    infliction to get around these restrictions.
    \item There is no need to prove physical injury. % todo verify
    \item Most states require a showing from the defendant of outrageous 
    behavior beyond all reasonable bounds of decency.
    \item The relationship between the plaintiff and defendant can impact the 
    court's characterization of the conduct as extreme or outrageous.
    \item The plaintiff's sensitivity usually isn't enough---e.g., 
    \emph{Nickerson v. Hodges}, where a woman believed her dead relatives had 
    buried a pot of gold in her backyard. The defendants buried a pot of dirt, 
    which she opened at the bank, expecting gold. The court found that the 
    joke caused her extreme distress.
\end{enumerate}

\subsubsection{High Threshold for IIED: \emph{Slocum v. Food Fair Stores of 
Florida, Inc.}}

There is a high threshold for behavior that constitutes gross recklessness or 
intent to cause severe distress.

\begin{enumerate}
    \item A shopper in a store asked the price of an item. An employee 
    replied, ``if you want to know the price, you'll have to find out the best 
    way you can...you stink to me.'' She had a heart attack and sued for 
    intentional infliction of emotional distress.
    \item The appellate court denied the claim, reasoning that the language 
    did not constitute ``gross recklessness,'' nor was it intended to cause 
    ``severe emotional distress.''
    \item Would racial identities have affected the court's holding?
    \item Levy thinks \emph{Slocum} is wrong---the appellate court should have 
    taken the jury's verdict into account.
\end{enumerate}

\subsubsection{IIED and Employment: \emph{Rulon-Miller v. International 
Business Machines Corporation}}

Restricting an employee's rights can constitute IIED.

\begin{enumerate}
    \item The plaintiff, a longtime IBM employee carried on a relationship 
    with an employee at a rival office products firm, QYX. Her managers at 
    first indicated they did not think the relationship constituted a conflict 
    of interest---``I don't have any problem with that.'' But then her manager 
    told her to end the relationship or lose her job, giving her ``a couple of 
    days to a week'' to think about it. The next day, he said ``he had made up 
    her mind for her'' and dismissed her.
    \item The court held that the manager ``intended to emphasize that she was 
    powerless to to do anything to assert her rights,'' affirming the judgment 
    for intentional infliction of emotional distress.
\end{enumerate}

\subsubsection{IIED and Sexual Harassment: \emph{Jones v. Clinton}}

Sexual harassment does not necessarily indicate IIED.

\begin{enumerate}
    \item Paula Jones claimed Bill Clinton's ``actual exposure of an intimate 
    private body part'' constituted extreme and outrageous conduct.
    \item The court found no evidence that the incident caused any significant 
    lasting emotional distress and rejected the claim in a summary judgment.
\end{enumerate}

\subsection{Defenses to Intentional Torts}

\begin{enumerate}
    \item The burden of proof is on the defendant (i.e., the one raising the 
    defense).
\end{enumerate}

\subsubsection{Self Defense}

\begin{enumerate}
    \item Force intended to inflict death or serious injury must be necessary 
    and is only reasonable in response to the \textbf{immediate threat of 
    serious bodily injury or death}.
    \item The Restatement of Torts also requires retreat if safely possible 
    (except from the victim's own dwelling) before the victim can respond with 
    force intended to inflict serious bodily injury or death. Most courts 
    disagree.\footnote{Casebook p. 64.}
    \item If the threat is not immediate, self-defense is not valid. There is 
    dispute about spousal abuse cases, however---should the smaller spouse be 
    required to wait until the physical threat is immediate before asserting 
    the right to self-defense?
    \item The immediate threat requirement is controversial in spousal abuse 
    cases.
    \item Reasonable mistakes in perceiving threats can be valid bases for 
    self defense.
    \item There is a limited right to self defense against excessive police 
    force.
    \item Should good samaritans be encouraged to intervene? The Second 
    Restatement allows bystanders to assert self defense if they reasonably 
    believe that the third party has a privilege of self defense and that 
    intervention is necessary to protect him. The traditional rule, however, 
    only allows intervention when the third party is actually privileged. The 
    Second Restatement would allow reasonable mistakes, but the common law 
    rule does not.
    \item Self defense includes protection of members of the defendant's 
    family and workplace.
    \item Reasonable force is allowed to protect property. Force intended to 
    inflict death or serious bodily injury (e.g., spring guns in barns) is 
    never allowed.
    \item Defense of others generally does not apply to unborn fetuses.
    \item Private citizens can use reasonable force to arrest others who 
    committed felonies or when the felony occurred and the citizen reasonably 
    believes the person arrested is guilty. Private citizens can also arrest 
    others they witness committing misdemeanors.
\end{enumerate}

\paragraph{\emph{Drabek v. Sabley}}

\begin{enumerate}
    \item Ten-year-old Drabek and friends were throwing snowballs at passing 
    cars. One driver, Sabley, stopped, caught Drabek, took him by the arm to 
    his car, and drove him back to the village of Williams Bay. He turned 
    Drabek over to the police. Drabek was with Sabley for a total of 15-20 
    minutes.
    \item The court held that Sabley was justified in preventing the 
    commission of a crime, and so it was reasonable to admonish Drabek and 
    march him home. But it was not reasonable to detain him and take him to 
    the police station, so Sabley was liable for false imprisonment and 
    nominal battery.
    \item Remanded to determine compensatory (but not punitive) damages.
\end{enumerate}

\subsubsection{Necessity}

\begin{enumerate}
    \item ``\textbf{Private necessity} is a privilege which allows the 
    defendant to interfere with the property interests of an innocent party in 
    an effort to avoid a greater injury. The privilege is incomplete since the 
    actor must still compensate the victim for the 
    property.''\footnote{Casebook pp.  69--70.}
    \item The defendant must have \textbf{reasonably perceived} the need to 
    appropriate the victim's property to avoid a greater damage to property or 
    life.
    \item ``The defense of \textbf{public necessity} allows the appropriation 
    of property to avoid a greater harm to the public''\footnote{Casebook p.  
    71.}---e.g., destroying a building to prevent a fire from spreading to the 
    rest of the city. Compensation to the property owner is not required.
\end{enumerate}

\paragraph{\emph{Vincent v. Lake Erie Transp. Co.}}

\begin{enumerate}
    \item The defendant was moored at the plaintiff's dock to unload goods 
    when a severe storm struck.  He kept his boat secured (and repeatedly 
    replaced damaged or broken lines) to the dock throughout the storm, 
    causing \$500 in damages to the dock.
    \item The court held that private necessity meant the defendant was 
    justified in using another's property due to the extreme circumstances but 
    was responsible for the damages he incurred.
\end{enumerate}

% TODO can you claim necessity to protect another's property?

%%%%%%%%%%%%%%%%%%%%%%%%%%%%%%%%%%%%%%%%%%%%%%%%%%%%%%%%%%%%%%%%%%%%%%%%%%%%%%%%%%%%%%%%%%%%

\subsection{Intentional Interference with Contractual and Economic Relations}

\begin{enumerate}
    \item Economic torts: we want competition, but not too much.
    \item Contractual interference torts are rooted in anti-labor motivations.
    \item According to the Second Restatement, the elements of these two torts 
    are:
    \begin{enumerate}
        \item A valid contract or economic expectancy.
        \item Defendant's knowledge of the contract or economic expectancy.
        \item Defendant's intent to interfere.
        \item Interference.
        \item Damage to the plaintiff.\footnote{Casebook p. 82 n. 1.}
    \end{enumerate}
    \item The plaintiff bears the burden of proof. The plaintiff must show 
    that the interference was not justified.
    \item Many courts recognize various justifications:
    \begin{enumerate}
        \item Fair competition or proper protection of one's own financial 
        interest (as long as the contract is freely terminable at will).
        \item Protecting the welfare of another for whom one is responsible.
        \item Providing truthful or honest information if requested.
        \item Assertion of a bona fide property right (e.g., preventing a 
        thief from selling your car).
        \item Interfering with an agreement that is illegal or against public 
        policy.
    \end{enumerate}
    \item Not all courts treat these as distinct torts, though California 
    does.
    \item For non-legal reasons (e.g., public relations), these torts are 
    rarely brought (do you really want to sue several people for leaving your 
    firm?).
    \item \end{enumerate}

\subsubsection{\emph{Calbom v. Knudtzon}}

\begin{enumerate}
    \item Mr. Henderson died and left Mrs. Henderson to execute his estate.  
    Harry Calbom, a lawyer, had been hired to help sort out the legal issues.  
    Mrs. Henderson's accountant, Mr. Knudtzon, told Mrs. Henderson that Calbom 
    was unsatisfactory and provided a list of other attorneys. Mrs. Henderson 
    found another attorney.
    \item Calbom sued for intentional interference with his employment 
    contract. The court held that an attorney-client relationship existed, 
    which Calbom had every right to expect would continue. It found that the 
    ``defendants' interference was malicious, intentional, and without 
    justification,'' affirming the judgment for Calbom.
    \begin{enumerate}
        \item Levy: this case is wrong. Knudtzon gave multiple suggestions for 
        other attorneys, and there is no evidence of favoritism or kickbacks.
    \end{enumerate}
\end{enumerate}

\subsubsection{Justified Interference: \emph{Lowell v. Mother's Cake \& Cookie 
Co.}}

``...intentional interference with prospective economic advantage constitutes 
actionable wrong \emph{if} it results in damages to the plaintiff, and the 
defendant's conduct is not excused by a legally recognized privilege or 
justification.''\footnote{Casebook p. 87.}

\begin{enumerate}
    \item The owner of Lowell Freight Lines had a longstanding oral contract 
    with Mother's. He planned to sell the company. Mother's told prospective 
    purchasers that it would terminate the delivery contract if Lowell sold 
    the company. Lowell sold the company for \$17,400 instead of the alleged 
    true market value of \$200,000.
    \item The trial court granted Mother's demurrer.
    \item The appellate court reversed, holding that Lowell stated a cause of 
    action for tortious interference with prospective business advantage and 
    that Mother's justification failed to appear on the face of the complaint.
\end{enumerate}

\subsubsection{Knowledge of a Contract: \emph{Texaco, Inc. v. Pennzoil, Inc.}}

Knowledge of a contract (even if only an oral contract) and intent to cause 
its breach are sufficient to constitute intentional interference.

\begin{enumerate}
    \item Pennzoil was negotiating an oral contract with Getty in which 
    Pennzoil would purchase Getty stock. The trial jury found that the 
    contract had been established and that Texaco intentionally interfered 
    with the agreement. It awarded \$7.53 billion in compensatory damages and 
    \$3 billion in punitive damages.
    \item On appeal, the issues were (1) whether there was a binding contract 
    between Getty and Pennzoil and (2) whether Texaco knowingly induced a 
    breach of the contract.
    \item The appellate court found that the contract was valid and 
    enforceable.
    \item The appellate court also found that knew of the agreement and 
    actively induced its breach.
    \item The appellate court affirmed but reduced the punitive damage award.
\end{enumerate}

\subsubsection{Boycotts: \emph{Environmental Planning \& Information Council 
(EPIC) of Western El Dorado County, Inc. v. Superior Court}}

Political expression is protected from intentional interference actions if the 
defendant provides only truthful information.

\begin{enumerate}
    \item Detmold, a newspaper, sued EPIC for criticizing Detmold's editorial 
    policies on environmental issued, urging its readers to boycott business 
    that advertised in Detmold's paper. The state court denied EPIC's motion 
    for summary judgment.
    \item On appeal, the California Supreme Court held that EPIC was 
    advocating a political boycott and was thus protected. Reversed (granting 
    summary judgment for EPIC.)
\end{enumerate}

\subsection{Wrongful Termination of Employee}

\begin{enumerate}
    \item An employer can be liable for wrongful termination if the 
    termination contradicts significant public policy.
    \item Unless otherwise indicated explicitly or implicitly, the employment 
    contract is at-will.
    \item Traditionally, a breach of an employment contract would only lead to 
    contract damages, which are smaller than tort damages. Leading plaintiff's 
    lawyers developed the tort of wrongful discharge, which holds the employer 
    liable for emotional distress and punitive damages (in addition to 
    breach-of-contract damages).
    \item A cause of action exists only if the firing breaches constitutional 
    or statutory public policy.
\end{enumerate}

\subsubsection{Termination for Protecting the Public Interest: \emph{Foley v. 
Interactive Data Corp.}}

Employees can recover damages if their firing resulted from actions they took 
to protect the public interest. However, it is not enough to merely protect 
the employer's private interests.

\begin{enumerate}
    \item A well-regarded employee, Foley, became concerned when he learned 
    that the person hired to be a new Vice President was under FBI 
    investigation for embezzlement from Bank of America, his previous 
    employer. Foley was fired within a few days
    \item The court found that there was no wrongful termination because 
    Foley's disclosure benefited only the private interests of his employer, 
    not the public.
\end{enumerate}

%%%%%%%%%%%%%%%%%%%%%%%%%%%%%%%%%%%%%%%%%%%%%%%%%%%%%%%%%%%%%%%%%%%%%%%%%%%%%%%%%%%%%%%%%%%
\subsection{Tortious Breach of the Covenant of Good Faith and Fair Dealing}

\begin{enumerate}
    \item \textbf{Every contract imposes a duty of good faith and fair 
    dealing}. Some courts hold that a breach of this covenant constitutes a 
    tort, allowing tort damages (e.g., punitive damages and compensation for 
    mental distress) as well as breach-of-contract remedies.
    \item CA was the first state to develop the tort. It arose in the context 
    of bad faith breaches of insurance contexts and then later extended to 
    other contexts. But, after 1986 political swing in the California courts, 
    the rule became that actions for bad faith are only allowed in insurance 
    cases.
    \item There are two types of insurance: first party (which bets on whether 
    an event will happen---disability, life, etc.) and third party (which 
    addresses incidents involving third parties---liability, homeowners, auto, 
    etc.).
    \item Many courts have limited this tort to insurance contexts. However, 
    as many as 16 of the 36 states that recognize the tort have applied it to 
    non-insurance contexts.\footnote{Casebook p. 115.}
\end{enumerate}

\subsubsection{Insurance Company Obligations to Policyholders: \emph{Egan v.  
Mutual of Omaha Insurance Co.}}

\begin{enumerate}
    \item Egan purchased a disability insurance policy from Omaha Insurance.  
    When the plaintiff became disabled, the insurance company  withheld 
    payments, calling the plaintiff a ``fraud.''
    \item The court found that Omaha wrongly and maliciously withheld payment.  
    It held that an insurer ``cannot reasonably and in good faith deny 
    payments to its insured without thoroughly investigating the foundation 
    for its denial.'' The court found for the plaintiff (but deemed the 
    punitive damages of \$5 million to be excessive).
\end{enumerate}

\subsection{Intentional Misrepresentation}

\begin{enumerate}
    \item Restatement (Second): ``One who fraudulently makes a [material] 
    misrepresentation of fact, opinion, intention or law for the purpose of 
    inducing another to act or to refrain from action in reliance upon it, is 
    subject to liability to the other in deceit for pecuniary loss caused to 
    him by his justifiable reliance upon the 
    misrepresentation.''\footnote{Casebook p. 121.}
    \item \textbf{Misrepresentation must be intentional or reckless}. It 
    usually has to be a statement of fact, not opinion, except in the case of 
    a fiduciary.  \item Cal. Civ. Code \S\ 1710 defines actionable 
    deceit.\footnote{A deceit, within the meaning of the last section, is 
    either:
    \begin{enumerate}
        \item The suggestion, as a fact, of that which is not true, by one who 
        does not believe it to be true;
        \item The assertion, as a fact, of that which is not true, by one who 
        has no reasonable ground for believing it to be true;
        \item The suppression of a fact, by one who is bound to disclose it, 
        or who gives information of other facts which are likely to mislead 
        for want of communication of that fact; or,
        \item A promise, made without any intention of performing it.
    \end{enumerate}}
    \item Courts have traditionally \textbf{not included} broken promises 
    within this tort (though they may constitute breach of contract). However, 
    some courts and the Restatement (Second) have begun to distinguish between 
    promises that are lies (which are tortious) and sincere 
    promises.\footnote{Casebook p. 124.}
    \item Failure to disclose can constitute concealment.
\end{enumerate}

\subsubsection{Authentic Reliance: \emph{Nader v. Allegheny Airlines, Inc.}}

To successfully recover, a plaintiff must have authentically relied on the 
misrepresentation.

\begin{enumerate}
    \item Allegheny Airlines bumped Nader from a flight, causing him to miss a 
    speaking engagement. The airline intentionally overbooked the flight but 
    told all passengers that they had ``confirmed reservations.''
    \item Allegheny argued that ``confirmed'' was reasonable language because 
    the probability of being bumped was very low.
    \item The trial court held that the airline's nondisclosure was 
    misleading. It awarded \$10 in compensatory damages to Nader and \$15,000 
    in punitive damages.
    \item The appellate court reversed, arguing that Nader's reliance was not 
    justifiable because he had been bumped many times before and knew about 
    the airline's policy. He had not authentically relied on the 
    misrepresentation.
\end{enumerate}
