\subsection{Land Occupiers' Duty}

\begin{enumerate}
    \item Landowners and occupiers (e.g., tenants in possession) traditionally 
    owe no duty of care to visitors to the land.  
    \item The landowner or occupier's duty depends on the legal status of the 
    visitor.  There are three types:\footnote{Casebook pp. 381--82.}
    \begin{enumerate}
        \item \textbf{Trespassers}: need not have explicit consent.  
        Historically, trespassers have very little protection, except in cases 
        of intentional or willful injury. Some jurisdictions, and the 
        Restatement (Second), impose a reasonable person standard of care on 
        landowners when trespassers or present or should reasonably be 
        anticipated to be present.
        \item \textbf{Invitees}: entering for business purposes or when the 
        land is open to the public. The landowner \textbf{must act reasonably 
        to ensure safe conditions for invitees} (although he only needs to act 
        as a reasonable person would---he does not need to eliminate all 
        risks).  There is therefore no limit on the landowner's duty to 
        invitees.
        \item \textbf{Licensees}: entering with permission or privilege, e.g., 
        a social guest, or firefighters and police. Licensees traditionally 
        receive very little protection, but today many jurisdictions require 
        the landowner to act reasonably (as with trespassers). Many also 
        require the landowner to warn of concealed conditions posing an 
        unreasonable risk, but not of obvious dangers.
    \end{enumerate}
    \item Some jurisdictions reject the trespasser's status as the basis of 
    duty, i.e., they treat trespassers, licensees, and landowners alike.  
    Others keep a special category for trespassers.
    \item Courts weigh several factors (from \emph{Rowland}, below) in 
    determining when a land occupier has a duty of care:
    \begin{enumerate}
        \item Foreseeability of the harm.
        \item Degree of certainty of the harm.
        \item Closeness of the connection between the landowner's conduct and 
        the plaintiff's harm.
        \item Policy of preventing future harms.
        \item Burden of the duty rule on the defendant.
        \item Availability of insurance.
    \end{enumerate}
    \item Child trespassers are often granted special 
    protections.\footnote{Casebook p. 387.}
\end{enumerate}

\subsubsection{Merging Trespassers, Licensees, and Invitees: \emph{Rowland v.  
Christian}}

A landowner's general duty of care to visitors overrides the traditional 
immunities based on common law classifications (trespasser, licensee, 
invitee).

\begin{enumerate}
    \item Nancy Christian had contacted her building manager about a broken 
    bathroom faucet. Rowland was a social guest. She did not warn him about 
    the broken faucet. While using the bathroom, the faucet broke in his hand, 
    causing injuries. He sued for negligence. It was not clear from trial 
    whether the crack in the faucet handle was obvious or concealed.
    \item Christian won summary judgment at trial.
    \item The California Supreme Court found that a jury could have determined 
    that Christian was aware of the broken faucet, that she should have 
    expected that Rowland would not have discovered the danger, that she did 
    not eliminate the danger or warn him of it, and he did not know or have 
    reason to know of the danger. Summary judgment was therefore 
    inappropriate.
    \item Landowners traditionally did not owe a special duty to licensees.  
    However, the traditional classifications of licensee, trespasser, and 
    invitee do not hold up well. Several factors (see above) can alter the 
    traditional rules of duty between the landowner and the visitor.
    \item ``...everyone is responsible for an injury caused to another by want 
    of his ordinary care or skill in the management of his property.'' Common 
    law classifications do not warrant ``wholesale 
    immunities.''\footnote{Reader p. 20.}
\end{enumerate}

\subsubsection{Duty and Foreseeability: \emph{Ann M. v. Pacific Plaza Shopping 
Center}}

A landlord's duty depends on the foreseeability of harm. A shopping center's 
general duty to maintain and control its land in a reasonably safe condition 
does not require it to provide security guards.

\begin{enumerate}
    \item Ann M. worked at a photo store in a strip mall. A man entered the 
    store and raped her.
    \item Ann M. sued Pacific Plaza for negligently failing to provide 
    adequate security to protect against unreasonable harm.
    \item The trial court granted summary judgment on the grounds that Pacific 
    Plaza owed her no duty of care. The appellate court affirmed on the 
    grounds that Pacific Plaza \emph{did} owe a duty, but that no reasonable 
    jury could have concluded that Pacific Plaza acted unreasonably in failing 
    to provide security patrols.
    \item The California Supreme Court held that landlords generally owe a 
    duty to tenants to secure common areas against foreseeable criminal acts 
    of third parties. Here, however, ``violent criminal assaults were not 
    sufficiently foreseeable to impose a duty on Pacific Plaza to provide 
    security guards in the common areas.''\footnote{Reader p. 28.} 
    \textbf{Courts must balance the foreseeability of the harm against the 
    burden of the duty to be imposed.}The lack of similar prior incidents 
    indicated unforeseeability. Affirmed.
\end{enumerate}

\subsubsection{Affirming \emph{Ann M.}: \emph{Wiener v. Southeast Childcare 
Ctrs., Inc.}}

Affirming \emph{Ann M.}, criminal acts must be foreseeable to create a duty 
for a landowner to take additional steps to prevent them.

\begin{enumerate}
    \item Abrams drove his car through a fence at Southeast Childcare, killing 
    to children. The children's parents sued the childcare center for 
    ``provid[ing] inadequate protection [a four-foot-tall chain link fence] 
    against intrusion into the child care center.''\footnote{Reader p. 28.}
    \item The trial court granted summary judgment, holding that Abrams's 
    rampage was ``wholly unforeseeable.'' The appellate court reversed on the 
    grounds that an errant motorist careening through the fence was 
    foreseeable.
    \item The California Supreme Court held that the ``defendants owed no duty 
    to plaintiffs because Abrams's brutal criminal act was unforeseeable.'' 
    \footnote{Reader p. 36.} Reversed.
\end{enumerate}

