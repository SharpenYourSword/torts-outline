\subsection{Wrongful Death and Survival Actions}

%
%\subsubsection{Wrongful Death and Survival Actions: \emph{Gary v. Schwartz}}
%
\begin{enumerate}
%    \item todo
    \item Common law: no recovery for wrongful killing. Also, if a person had 
    a tort action and then died, the action would end.
    \item Certain named categories of relatives can recover for wrongful 
    death. They can recover money that was used to support them. In all but a 
    few jurisdictions, damages are limited to pecuniary losses, not including 
    pain and suffering. IN CA, however, survivors can collect for monetary 
    value of loss of companionship--a way of softening the effect of not 
    allowing pain and suffering.
    \item Survival actions: suits brought in the name of the estate of the 
    deceased. How much did the estate lose because of the deceased? and No 
    recovery for pain and suffering of victim before daeth
    \item Loss of consortium: loss of love, companionshiop, society, sex, 
    services--all because of injury to another. In CA, only a spouse can bring 
    a loss of consortium action.
    \item Wrongful birth: if a couple gets sterilized or an abortion, which is 
    performed negligently: if teh child is healthy, some states do not allow 
    recovery. in CA, one can recover for the cost of raising the child, but 
    defendant can ask that it be reduced for the joy and benefits the parents 
    received. levy has problems with that bc of question........ [??????]
    \item There can also be recoery if malpracitce failed to disclose 
    condition fo the fetus. If child is severly unhealthy, recovery can 
    include special needed care.
    \item wrongful life? Can a child bring its own action? yes, if it's 
    necessary for care after the parents' death. CA does not allow recovery 
    for pain of life, but does allow recovery for special medical care.
\end{enumerate}
%
%\subsubsection{\emph{Selders v. Armentrout}}
%
%\begin{enumerate}
%    \item todo
%\end{enumerate}
%
%\subsubsection{\emph{Compania Dominicana de Aviacion v. Knapp}}
%
%\begin{enumerate}
%    \item todo
%\end{enumerate}
%
