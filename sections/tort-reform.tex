\section{Tort Reform}

\subsection{\emph{Fein v. Permanente Medical Group}}

\begin{enumerate}
    \item Lawrence Fein felt chest pain and went to his doctor's office, where a 
    nurse practitioner told him that his pain was due to a muscle spasm and sent 
    him home with Valium. The chest pains returned that night. He went to the 
    emergency room, where the doctor also diagnosed the problem as muscle 
    spasms, giving him a Demerol injection and a codeine prescription. The next 
    day, he went back to the emergency room, where an EKG showed he was 
    suffering from a heart attack.
    \item Fein sued Kaiser for malpractice, arguing at trial that the failure to 
    initially diagnose his heart attack caused much of his heart muscle to die, 
    reducing his life expectancy by at least 16 years. The trial court awarded 
    \$1 million in economic damages.
    \item On appeal, Fein argued that the trial court erred in applying two 
    provisions of the Medical Injury Compensation Reform Act (MICRA), (1) limiting 
    non-economic malpractice damages to \$250,000 (Cal. Civ. Code \S\ 3333.2) and (2) modifying the 
    collateral source rule in malpractice cases (Cal. Civ. Code \S\ 3333.1.
    \item Fein contended that \S\ 3333.2 (\$250,000 limit on non-economic 
    damages) denied due process. The court held that the legislature pursued a 
    legitimate statute interest in enacting the statutory limits on recovery. 
    Fein also argued that the limit denied equal protection by discriminating 
    against malpractice victims and against malpractice plaintiffs with 
    non-economic damages above \$250,000. The court rejected both 
    arguments.\footnote{Casebook p. 627.}
    \item Fein also raised a constitutional challenge to \S\ 3333.1 (allowing 
    collateral source evidence and preventing the collateral source from 
    obtaining subrogation). The court held that although the provision affected 
    a plaintiff's recovery, it was constitutional because it promoted the 
    legitimate state interest of containing health care costs.
    \item Affirmed.
\end{enumerate}

\subsection{Eisenberg and Sieger, ``The Doctor Won't See You Now''}

Rising insurance costs prompt malpractice recovery reforms.

\subsection{Treaster and Brinkley, ``Behind those Medical Malpractice 
Rates''}

Increases in insurance premiums do not closely correlate with increases in the 
number of malpractice claims or the size of damage awards. Rather, the increases 
may be due to unsuccessful insurance company investments.

\subsection{Colliver, ``We Spend Far More, but Our Healthcare is Falling 
Behind''}

U.S. healthcare costs skyrocket while relative quality drops.

\subsection{Sack, ``Doctors Say `I'm Sorry' before `See You in Court'''}

Disclosing mistakes to patients dramatically reduces malpractice litigation 
which, in turn, reduces health care costs.

\subsection{Patient Protection and Affordable Care Act \S\ 6801}

The Senate supports exploring alternatives to civil litigation of malpractice 
disputes.
