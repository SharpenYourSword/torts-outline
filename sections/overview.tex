\section{Overview}

\begin{enumerate}
    \item The goal of tort law is to shift the burden of economic loss.
    \item Tort law rests on three frameworks:
    \begin{enumerate}
        \item Fairness.
        \item Loss distribution.
        \item Law and economics.
    \end{enumerate}
    \item The Supremacy Clause leads to three kinds of 
    \textbf{preemption} of federal laws over state laws:
    \begin{enumerate}
        \item \emph{Express preemption}: A federal law explicitly or 
        implicitly overrides a state statute.
        \item \emph{Conflict preemption}: In case of direct conflict, 
        federal law preempts state law.
        \item \emph{Field preemption}: Congress legislates for an entire 
        field of regulation, leaving no room for states to regulate.
    \end{enumerate}
\end{enumerate}

\subsection{Intentional Torts}

\begin{enumerate}
    \item Definition of intent, infancy, insanity, battery, assault, 
    transferred intent, mistake, false imprisonment, malicious prosecution, 
    abuse of process, intentional infliction of emotional distress, 
    intentional interference with economic and contractual relationships, 
    wrongful termination of employee, tortious breach of the covenant of good 
    faith and fair dealing, intentional misrepresentation.
    \item Defenses: self defense, necessity.
\end{enumerate}

\subsection{Negligence}

\begin{enumerate}
    \item \textbf{Overview}.
    \begin{enumerate}
        \item Elements of negligence: duty, breach of duty (breach of the 
        standard of care or failure to act as a reasonable person), 
        cause-in-fact, proximate cause, damages. 
        \item Hand formula, foreseeable and unreasonable.
    \end{enumerate}
    \item \textbf{Standard of care}: standard of conduct: emergencies, mental 
    illness, child standard, adult activities, professional standard, informed 
    consent.
    \item \textbf{Rules of law and negligence per se}: rules of law, 
    negligence per se, statutory purpose doctrine, interpreting legislative 
    intent.
    \item \textbf{Cause in fact}: cause in fact, foreseeability, proximate 
    case, acting in concert, \emph{Summers v. Tice}/alternative liability 
    test, \emph{Sindell}/market share liability, toxic exposure/uncertain 
    harm.
    \item \textbf{Duty and proximate cause}: duty, proximate cause, two 
    \emph{Palsgraf} frameworks, two views of proximate cause, intervening 
    superseding events vs. dependent/naturally occurring intervening events, 
    egg-shell plaintiff rule.
    \item \textbf{Proof of negligence/res ipsa loquitur}: res ipsa loquitur, 
    presumption vs. inference, \emph{Ybarra} rule, Cal Evid. Code \S\ 646.  
    \item \textbf{Limitations on duty}: no duty to act, common law 
    relationships/status, voluntary interventions, parental duty, 
    \emph{Tarasoff}/therapists' duty, police duty.
    \item \textbf{Emotional distress}: \emph{Amaya}/``zone of danger,'' 
    \emph{Dillon} (guidelines), \emph{La Chusa} (requirements), toxic 
    exposure,
    \item \textbf{Wrongful death}: common law vs. modern jurisdictions, named 
    categories of relatives who can recover, damages 
    (pecuniary, pain and suffering, grief), C.C.P. \S\ 377, survival actions, 
    damages in wrongful death vs. damages in survival actions.
    \item \textbf{Loss of consortium}: loss of consortium, wrongful life, 
    wrongful conception, wrongful birth.
    \item \textbf{Land occupiers' duty}: common law vs. modern jurisdictions, 
    three legal statuses of visitors (trespassers, invitees, licensees), 
    rejection of visitor's status, \emph{Rowland} factors, child trespassers.
    \item \textbf{Negligent misrepresentation}: hesitancy on awarding damages 
    for pure economic loss (and \emph{J'Aire}), business relationship 
    requirement, third party recovery.
    \item \textbf{Comparative negligence}: contributory negligence, modified 
    vs. pure comparative negligence, last clear chance, assumption of risk 
    (primary vs. secondary), \emph{Li}, \emph{Knight}, veterinarians' rule, 
    immunity and government liability, firefighters' rule.
    \item \textbf{Joint and several liability}: joint liability, several 
    liability, joint and several liability, comparative indemnification (vs. 
    earlier common law rule), contribution vs. indemnification, Prop 51.
    \item \textbf{Insurer's failure to settle within policy limits}.
\end{enumerate}

\subsection{Strict Liability}

\begin{enumerate}
    \item \textbf{Generally}: policy rationales, abnormally dangerous activities, 
    legislative programs, dangerous products vs. dangerous activities, hazard 
    and causation.
    \item \textbf{Products liability}: definition, fault vs. loss 
    distribution, comparison to other bases for liability (negligence, 
    express/implied warranty, representation), privity, types of defect 
    (design, manufacturing, warning), issues in defining defect, prescription 
    pharmaceuticals, state-of-the-art defense, Restatement (Third) revisions, 
    recovery for economic damages, comparative negligence, federal preemption, 
    liability for component parts, sophisticated/professional user defense, 
    assumption of risk, tobacco strict liability, \emph{Barker} 
    test/risk--utility.
\end{enumerate}

\subsection{Damages}

\begin{enumerate}
    \item \textbf{Compensatory damages}: loss of enjoyment, pain and 
    suffering, loss of services, collateral source rule.
    \item \textbf{Punitive damages}: deterrence/retribution, criteria for 
    review (\emph{Gore}), Cal. Civ. Code \S\ 3294.
\end{enumerate}

\subsection{Vicarious Liability}

\begin{enumerate}
    \item Respondeat superior, going-and-coming rule, independent contractors, 
    children.
\end{enumerate}

\subsection{Tort Reform}

\begin{enumerate}
    \item MICRA, collateral source rule, subrogation, doctor apologies.
\end{enumerate}

\subsection{Workers' Compensation}

\begin{enumerate}
    \item Strict liability, no compensatory damages for intangibles, no 
    punitive damages, intentional torts exceptions, bar on negligence claims, 
    going-and-coming rule, special risk exception.
\end{enumerate}

\subsection{Automobile No-Fault}

\begin{enumerate}
    \item Arguments for and against, pure, partial, choice, neo-partial.
\end{enumerate}

\subsection{Defamation}

\begin{enumerate}
    \item Libel vs. slander, common law vs. Restatement (Second), harm to reputation, truth
    defense, libel per se vs. libel per quod, colloquium, right-thinking 
    person, opinion vs. fact, defamation in fiction, proof of special harm in 
    slander and exceptions, New York Times malice, public officials and public 
    figures, absolute privilege, qualified privilege, Speech or Debate Clause, 
    news media privilege.
\end{enumerate}

\subsection{Privacy}

\begin{enumerate}
    \item Four traditional privacy torts, intrusion upon seclusion, 
    appropriation of name or likeness, publicity of private life, public 
    characterization in a false light, newsworthiness, common law vs. New York 
    Times malice, IIED and public figures.
\end{enumerate}
