\subsection{Emotional Distress}

\begin{enumerate}
    \item At common law, there was no recovery for emotional distress unless 
    there was physical contact. Bystanders could not recover.
    \item Although rejected in California in \emph{Dillon}, the \emph{Amaya} 
    rule allowed those in the ``zone of danger'' to recover.\footnote{Casebook 
    p. 333 n. 2.} A slight majority of courts still adhere to the zone of  
    danger rule.
    \item The \emph{Dillon} factors provided \emph{guidelines} for recovery:
    \begin{enumerate}
        \item Was the plaintiff at or near the scene?
        \item Was the distress caused by sensory and contemporaneous 
        observance?
        \item Did the plaintiff have a close relationship with the victim?
        \item (\emph{Dillon} did not have requirements for physical injury or 
        manifestation of emotional distress.)
    \end{enumerate}
    \item In \emph{La Chusa}, the California Supreme Court turned the 
    \emph{Dillon} guidelines into \emph{requirements} (a ``jurisprudence of 
    categories'').
    \item The \emph{Dillon} approach is becoming a majority view. Some 
    jurisdictions allow non-visual perception of the accident or perception of 
    its aftermath. The ``close relationship'' requirement generally includes 
    only blood relatives and family members. California recently extended it 
    to include domestic partners. Most \emph{Dillon} jurisdictions require a 
    physical manifestation of the emotional distress, e.g., a heart attack or 
    stomach pains, but not crying or insomnia.\footnote{Casebook pp.  
    334--336.}
\end{enumerate}

\subsubsection{Three Factors for NIED Recovery: \emph{Thing v. La Chusa}}

The court changed the \emph{Dillon} guidelines into requirements for recovery 
for negligent infliction of emotional distress.

\begin{enumerate}
    \item The mother of a child struck by a car sued the driver for negligent 
    infliction of emotional distress (NIED).
    \item The trial court granted a summary judgment for the defendant.
    \item The appellate court reversed, holding that the mother may recover.
    \item The California Supreme Court reversed the appellate court, holding 
    that the trial court was correct in granting summary judgment.
    \item In \emph{Amaya v. Home Ice, Fuel, \& Supply Co.}, the court held 
    that plaintiffs must have been within the ``zone of danger'' to recover 
    NIED damages.
    \item Five years later, the court overruled \emph{Amaya} in \emph{Dillon 
    v. Legg}. In that case, the mother of the victim may have been endangered 
    by the defendant's conduct, but the sister was not, leading to an 
    incongruous result from the ``zone of danger'' test. The court held that 
    recovery should be based on the traditional tort principles of 
    foreseeability, proximate cause, and consequential injury. The 
    \emph{Dillon} framework considered three factors:
    \begin{enumerate}
        \item Whether the plaintiff was \textbf{located near the scene}.
        \item Whether the plaintiff's shock resulted from the emotional impact 
        of \textbf{``contemporaneous observance''} of the incident.
        \item Whether the plaintiff and victim are \textbf{closely related}.
    \end{enumerate}
    \item Under \emph{Dillon}, the jury will decide on a case-by-case basis 
    ``what the ordinary man under such circumstances should reasonably have 
    foreseen.''\footnote{Casebook p. 325.}
    \item The court here argued that \emph{Dillon} created massive 
    uncertainty. The court replaced \emph{Dillon} with a new rule for finding 
    NIED, which requires three factors:\footnote{Casebook p. 323.}
    \begin{enumerate}
        \item Plaintiff must be \textbf{closely related} to the victim.
        \item Plaintiff must be \textbf{present at the scene} and aware that 
        an injury has occurred.
        \item Plaintiff must suffer \textbf{emotional distress beyond that 
        which would be anticipated in a disinterested witness}.
    \end{enumerate}
    \item The court's motivation was to ``limit liability and establish 
    meaningful rules for application by litigants and lower 
    courts.''\footnote{Casebook p. 329.} Policy reasons included guarding 
    against fraudulent claims and limiting defendants' 
    liability.\footnote{Casebook p. 324.}
    \item The dissent argued that \emph{Dillon} was meant to be a flexible 
    test based on the basic principles of torts The court here has replaced it 
    with an arbitrary rule. The ``policy reasons'' for replacing the 
    foreseeability requirement are not convincing.
    \item NIED was originally limited to cases where the plaintiff was 
    physically impacted, i.e., for emotional distress associated with personal 
    injuries. Virtually all courts have abandoned this rule.\footnote{Casebook 
    p. 333.}
\end{enumerate}

\subsubsection{Fear of Disease from Toxic Exposure: \emph{Potter v. Firestone 
Tire and Rubber Co.}}

NIED recovery for toxic exposure is allowed only if the plaintiff has more 
than a 50\% chance of developing a disease. A 30\% chance is not recoverable.

\begin{enumerate}
    \item Several landowners lived near a landfill, where it was discovered 
    that Firestone had negligently disposed of its toxic waste. The landowners 
    experienced prolonged exposure to carcinogens, and ``each faces an 
    enhanced but unquantified risk of developing cancer.''\footnote{Casebook 
    p. 337.}
    \item The plaintiffs sued for the emotional distress of the fear of 
    developing cancer in the future from exposure to carcinogens.
    \item The court held that plaintiffs can recover for negligent infliction 
    of emotional distress from exposure to carcinogens if (1) the defendant's 
    negligence caused the exposure and (2) it is \textbf{more likely than not} 
    that the plaintiffs will develop cancer.
\end{enumerate}
