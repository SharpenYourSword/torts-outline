\subsection{Comparative Negligence}

\begin{enumerate}
    \item Under \textbf{contributory negligence}, a plaintiff who acted at all 
    negligently is barred from recovery.
    \item Under \textbf{comparative negligence}, the plaintiff's recovery is 
    reduced according to his fault. Under \emph{modified} comparative 
    negligence, the plaintiff cannot recover if she is more than 50\% at 
    fault, while \emph{pure} comparative negligence allows a plaintiff who is 
    99\% at fault to recover 1\%.
    \item Courts must compare the plaintiff's fault to the fault of \emph{all 
    actors}, regardless of whether they are parties to the suit.
    \item Exculpatory clauses in contracts require a plaintiff to waive the 
    right to sue for certain injuries. California courts have held that 
    liability for gross negligence is not waivable.
    % TODO comparing the defenses 428
    % TODO last clear chance 429
    \item \textbf{Last clear chance} doctrine: even if the plaintiff was 
    negligent, if the defendant had the last clear chance to avert the harm, 
    the plaintiff could recover.
    \item \textbf{Assumption of risk} is an implied and knowing waiver of 
    liability for damages the defendant causes. The party assuming the risk 
    must have foreseen it with a high degree of specificity. In \emph{Li}, the 
    California Supreme Court awkwardly tried to distinguish between reasonable 
    and unreasonable risks. Later, in \emph{Knight}, the court revised the 
    assumption of risk rule to distinguish two types:
    \begin{enumerate}
        \item \textbf{Primary assumption of risk}: % TODO
        \item \textbf{Secondary assumption of risk}: % TODO
    \end{enumerate}
    \item In California, there is no comparative negligence if the defendant's 
    conduct was intentional. Comparative negligence applies if the defendant's 
    conduct was merely willful, wanton, or reckless.
    \item Calculating liability under comparative negligence (this approach 
    changes after Proposition 51 (below)): X suffers \$100,000 in damages. If X 
    was 40\% at fault, the award is reduced by \$40,000.
    % TODO a complete defense 432
    % TODO defense necessary? 432
    % TODO voluntary exposure to risk
    % TODO implied ass of risk 454
    % TODO merge into comparative fault 454
    % TODO implied ass of risk as complete defense 455
    % TODO limited duty 455
    % TODO firefighters rule 456
\end{enumerate}

\subsubsection{Pure Comparative Negligence: \emph{Li v. Yellow Cab Co.}}

The California Supreme Court replaces contributory negligence with pure 
comparative negligence.

\begin{enumerate}
    \item Li made a left turn across traffic, colliding with an oncoming taxi. 
    The trial court found her contributorily negligent and held for the 
    plaintiffs.
    \item The California Supreme Court applied the standard of pure 
    comparative negligence, under which Li would be allowed to recover damages 
    minus an amount in proportion to her fault.
    \item ``In all actions for negligence resulting in injury to person or 
    property, the contributory negligence of the person injured in person or 
    property shall not bar recovery, but the damages awarded shall be 
    diminished in proportion to the amount of negligence attributable to the 
    person recovering.''\footnote{Casebook p. 428.}
    \item The court reasoned that contributory negligence has a ``lottery aspect.'' ``Modified'' comparative negligence, which allows plaintiffs to recover if they are less than 50\% at fault, only shifts this aspect to a different place.\footnote{Casebook p. 427} (Most cases use the modified version.\footnote{Casebook p. 428.})
    \item The court removed the last clear chance rule, reasoning that it was 
    unnecessary under pure comparative negligence.
    \item The court distinguished between ``reasonable'' and ``unreasonable'' 
    assumption of risk, holding that unreasonable assumption of risk should be merged 
    into comparative negligence but that reasonable assumption of risk should 
    be retained as a distinct defense. This point proved confusing and the 
    court replaced it in \emph{Knight} with primary and secondary assumption 
    of risk.
    \item Previously, the California legislature decided not to adopt a 
    comparative negligence statute. The court believed it had authority on its 
    own to adopt comparative negligence judicially.
\end{enumerate}

\subsubsection{Assumption of Risk: \emph{Murphy v. Steeplechase Amusement 
Co.}} % TODO 429

\begin{enumerate}
    \item Plaintiff fell on ``the Flopper'' and injured himself.
    \item Trial court found for the plaintiff.
    \item Appellate court affirmed trial court.
    \item Defendant has the burden of proof in assumption of risk. To prove the assumption of risk defense, the defendant must prove that the plaintiff took a \textbf{knowing and voluntary} risk. It's a subjective test.
    \item New York Supreme Court reversed. 
    \item (If the defendant had been contributorily negligent, at the time he would not have been able to recover, because the court had not yet adopted comparative negligence.)
\end{enumerate}

\subsubsection{\emph{Rush v. Commercial Realty Co.}} % TODO 432

\begin{enumerate}
    \item Plaintiff fell through a trap door in an outhouse.
    \item Court found that the plaintiff had no choice but to use the outhouse. Therefore, there was no assumption of risk. Lower courts are affirmed.
    % \item Cf. McDermott p. 437. TODO add -- no choice --> no assumption risk
\end{enumerate}

\subsubsection{\emph{Emmette L. Barran, III v. Kappa Alpha Order, Inc.}} % 
% TODO 433

\begin{enumerate}
    \item todo
\end{enumerate}

\subsubsection{\emph{Knight v. Jewett}} % TODO 445

\begin{enumerate}
    \item todo
\end{enumerate}

\subsubsection{\emph{Priebe v. Nelson}} % TODO reader

\begin{enumerate}
    \item todo
\end{enumerate}

\subsubsection{\emph{Shin v. Ahn}} % TODO reader

\begin{enumerate}
    \item todo
\end{enumerate}

\subsubsection{Immunity: \emph{Metcalfe v. County of San Joaquin}} % TODO 
reader

\begin{enumerate}
    \item Affirmative defense: defendant responds to a plaintiff's complaint.
    \item Immunity: defendant cannot be sued because (1) their status makes them free from liability, and (2) the relationship between plaintiff and defendant--e.g., until recently, a parent couldn't sue a child.
    \item Governments are immune by default. Governmental liability is based entirely on statute.\footnote{The main statute is the CA Government Claims Act.}
    \item There are shortened claim statutes and statutes of limitations for suits against the government. For claims against the government, you must submit the claim to a governmental agency within six months. There are then another six months to bring suit. (The general statute of limitations is two years, but not in cases of governmental liability.)
    \item Historical immunities from tort liability:
    \begin{enumerate}
        \item Charities (until the 50s), because people didn't want donations to be used to pay for suits.
        \item Intra-family: parents often couldn't sue children and spouses couldn't sue each other.
        \item Guest statutes: passengers couldn't sue for driver's liability. (Two justifications: (1) not seemly to sue your host and (2) prevents collusion to collect insurance--same justifications for intra-family immunities)
    \end{enumerate}
\end{enumerate}
