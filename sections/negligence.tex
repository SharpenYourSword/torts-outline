\section{Negligence}

\subsection{Overview}

\begin{enumerate}
    \item ``The failure to exercise the standard of care that a \textbf{reasonably prudent person} would have exercised in a similar situation.''\footnote{Black's Law.}
    \item Negligence has six factors:
    \begin{enumerate}
        \item Duty.
        \item Standard of care.
        \item Breach of duty.
        \item Cause-in-fact.
        \item Proximate cause.
        \item Damages.
    \end{enumerate}
    \item The Hand Formula is Judge Learned Hand's test for determining negligence.
    \begin{enumerate}
        \item B = Burden of precautions necessary to prevent an accident.
        \item P = Probability that an accident will occur.
        \item L = Magnitude of the loss if the accident occurs.
        \item Negligence exists if B\textless PL---i.e., if the burden of precautions is less than the harm multiplied by the probability of occurrence.
    \end{enumerate}
\end{enumerate}

\subsubsection{\emph{Pitre v. Employers Liability Assurance Corporation}}

\begin{enumerate}
    \item The plaintiffs' son died when a patron at a carnival game was winding up a pitch and hit him in the head. The determining factor, the court reasoned, is how a ``reasonably prudent individual'' would have acted or what precautions he would have taken under similar circumstances. Negligence occurs only if the danger is both foreseeable and unreasonable. The court held that the danger was foreseeable but not unreasonable, and therefore there was no negligence.
\end{enumerate}

\subsubsection{\emph{United States Fidelity \& Guaranty Company v. Plovidba}}

\begin{enumerate}
    \item Inside a dark room, the hatch to a cargo hold on a ship was left open. A longshoreman fell through it and died. The court uses the Hand Formula, reasoning that B was relatively small (it would have been easy to close the hatch or leave a light turned on) and L was high (the victim died). P, however, was very small. There was no reason for the longshoreman to enter the hold. In fact, he was probably there to steal liquor, and the evidence suggests he knew the hatch was open and tried to skirt around it. The shipowner was therefore not liable for negligence.
\end{enumerate}
