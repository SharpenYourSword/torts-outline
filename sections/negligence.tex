\section{Negligence}

\subsection{Overview}

\begin{enumerate}
    \item ``The failure to exercise the standard of care that a \textbf{reasonably prudent [careful] person} would have exercised in a similar situation.''\footnote{Black's Law.}
    \item Negligence requires proof that the defendant acted unreasonably.
    \item The standard of care is objective.
    \item Negligence has five factors:
    \begin{enumerate}
        \item Duty.
        \item Breach of duty.
        \begin{enumerate}
            \item Negligence.
            \item Breach of the standard of care.
            \item Failure to act as a reasonably careful person would under the circumstances.
        \end{enumerate}
        \item Cause-in-fact.
        \item Proximate cause.
        \item Damages.
    \end{enumerate}
    \item Tort law generally doesn't lower the standard of care for people who are unable to meet it.
    \item The Hand Formula is Judge Learned Hand's test for determining negligence. It works best in scenarios where the actor takes a calculated risk (e.g., business decisions). It is less useful in cases where the actor was simply not paying attention.
    \begin{enumerate}
        \item B = Burden of precautions necessary to prevent an accident.
        \item P = Probability that an accident will occur.
        \item L = Magnitude of the loss if the accident occurs.
        \item Negligence exists if B\textless PL---i.e., if the burden of precautions is less than the harm multiplied by the probability of occurrence.
    \end{enumerate}
    \item Prosser: compare the utility of the risk with the gravity of the loss.
\end{enumerate}

\subsubsection{\emph{Pitre v. Employers Liability Assurance Corporation}}

\begin{enumerate}
    \item The plaintiffs' son died when a patron at a carnival game was winding up a pitch and hit him in the head. The trial court found in favor of the plaintiffs. The determining factor, the Court of Appeal reasoned, is how a ``reasonably prudent individual'' would have acted or what precautions he would have taken under similar circumstances. \textbf{Negligence occurs only if the danger is both foreseeable and unreasonable.} The court held that the danger was foreseeable but not unreasonable, and therefore there was no negligence.
\end{enumerate}

\subsubsection{\emph{United States Fidelity \& Guaranty Company v. Plovidba}}

\begin{enumerate}
    \item Inside a dark room, the hatch to a cargo hold on a ship was left open. A longshoreman fell through it and died. The trial court found for the defendant. Here, Richard Posner writing for the Seventh Circit applies the Hand Formula, reasoning that B was relatively small (it would have been easy to close the hatch or leave a light turned on) and L was high (the victim died). P, however, was very small. There was no reason for the longshoreman to enter the hold. In fact, he was probably there to steal liquor, and the evidence suggests he knew the hatch was open and tried to skirt around it. The shipowner was therefore not liable for negligence.
\end{enumerate}

\subsection{Standard of Conduct}

\subsubsection{\emph{Cordas v. Peerless Transp. Co.}}

\begin{enumerate}
    \item In emergencies, people are held to a lower standard.
    \item A man was mugged at gunpoint by two other men in New York City. He chased after them. One of the muggers jumped into a taxi, held the driver at gunpoint, and told him to drive. While the cab was in motion, the driver jumped out, and a few seconds later, so did the hijacker. The cab crashed into a sidewalk and injured the defendants. The trial court held that the driver was not negligent because he acted as a reasonable person would act under similar circumstances.
    \item Courts are divided on the question of whether juries should receive special instructions regarding negligence claims in emergency circumstances. On the one hand, it is redundant to reiterate that a defendant must be held to the standard of what a reasonable person would do in a similar emergency situation. Others claim it helps clarify the standard.
    \item \textbf{Conditional privilege}: choose the lesser of two harms.
\end{enumerate}

\subsubsection{\emph{Breunig v. American Family Insurance Company}}

\begin{enumerate}
    \item A schizophrenic woman had a psychotic episode while driving her car. The question was whether she had foreknowledge of her susceptibility to such attacks. The general rule is that insanity or another mental deficiency does not limit liability for negligence. In other words, \textbf{insane people are held to the reasonable person standard}. The court here notes that may be too harsh to exclude the insanity defense when a driver is suddenly overcome without warning. The Supreme Court agrees with the lower courts that the defendant did have the necessary foreknowledge, and held for the plaintiff. 
    \item Two frameworks for assessing liability from the sudden onset of mental illness:
    \begin{enumerate}
        \item Fairness: it's not fair to punish someone who could not have avoided having a seizure.
        \item Loss distribution: if someone has to bear the cost of repairing the harm, it should be the perpetrator, not the victim.
    \end{enumerate}
    \item One view: insanity constitutes a defense if there was no warning.
    \item Majority view (including CA): insanity does not create any defense as to compensatory damages. Physical ailments, however, are taken into account.
\end{enumerate}

\subsubsection{Child Standard and Adult Activity: \emph{Neumann v. Shlansky}}

\begin{enumerate}
    \item Should courts hold underage defendants to the standard of reasonable adults?
    \item An eleven-year-old hit a golf ball that struck the defendant in the knee, causing serious injury. Generally, children are held to the standard of a \textbf{reasonable person of like age, intelligence, and experience under the circumstances.} In this case, however, the child was engaging in an ``adult activity,'' and therefore the court held him to the adult reasonable person standard.
    \item Some states are moving from ``adult activity'' to ``inherently dangerous activity.''
\end{enumerate}

\subsubsection{Professional Standards: \emph{Melville v. Southward}}

\begin{enumerate}
    \item The defendant, a podiatrist, operated on the plaintiff's foot. The plaintiff sued for malpractice, and introduced the testimony of an orthopedist, who questioned the necessity and sanitation of the operation. The question is whether the orthopedist, a practitioner from a different school of medicine, should have been allowed to testify about the standard of care in podiatry. The trial court allowed the orthopedist to testify. The Supreme Court of Colorado here agreed with the appellate court that the testimony should not have been allowed because it was ``nothing more than an expression of opinion that that the general practice of podiatry did not meet the standard of care observed by an orthopedic surgeon.'' It remanded the case for a new trial.
    \item In malpractice cases, the ``competent professional'' standard replaces the ``reasonable person'' standard.
    \item There is disagreement about whether doctors in rural areas should be held to different standards than urban doctors.
    \item Medical specialists in the same geographic region are often reluctant to testify against each other---a ``conspiracy of silence.''
    \item In a limited range of cases, a jury of laypeople can determine whether a practice met an acceptable standard of care.
    \item Why let a profession set its own standards?
\end{enumerate}

\subsubsection{\emph{Cobbs v. Grant}}

\begin{enumerate}
    \item Doctors are required to obtain \textbf{informed consent} from patients. The plaintiff here sued a doctor who operated on a stomach ulcer but did not discuss the surgery's inherent risks. Complications developed, another operation was required, more complications developed, and so on.
    \item The plaintiff argued that (1) the doctor acted negligently in the performance of the surgery (which the jury found in favor of the plaintiff) and (2) that the doctor failed to obtain informed consent.
    \item The Supreme Court of California here noted that courts are divided as to whether this type of tort should be deemed a \textbf{battery or negligence}. The court aligned itself with a ``majority trend'' that advocates reserving battery for cases where a doctor performs an operation without the patient's consent. Generally, physicians are required to tell patients about major risks (but not every minor risk) and obtain the patient's consent. In this case, the court finds that there is not enough evidence to show that the doctor acted negligently, and the case is remanded for a new trial.
    \item \textbf{Failure to obtain informed consent can subject a physician to negligence liability.} Unless the physician misrepresents the entire procedure, most courts will not characterize the behavior as intentional battery.
\end{enumerate}

\subsection{Proving Negligence: Rules of Law}

\begin{enumerate}
    \item Juries typically determine what constitutes reasonable conduct under the circumstances. Judges, however, will sometimes establish a rule for what constitutes negligent conduct under particular circumstances.
    \item For instance, in \emph{Baltimore \& Ohio R.R.}, Justice Holmes established a rule requiring drivers to get out of their cars and examine railroad crossings.
    \item Most courts do not use this approach, because it's premised in repetition of fact patterns. Fact patterns are rarely identical. Judges may also be bad at making these rules---e.g., Holmes's stop, look, and listen rule (and get out of the car).
\end{enumerate}

\subsubsection{\emph{Akins v. Glens Falls City School District}}

\begin{enumerate}
    \item A foul ball injured a spectator at a baseball game. She sued the ballpark's owners, the local school district, for negligence. The trial court helf in favor of the plaintiff. The appellate court reversed, finding that the school district had not acted negligently, and establishing a specific rule for ballpark backstops. The dissent argued that such a rule ``robs the jury'' of the ability to consider important circumstances and locks the law in ``a position that is certain to become outdated.''
\end{enumerate}

\subsection{Negligence Per Se}

\begin{enumerate}
    \item Negligence exists when the actor violates a safety statute.
    \item In some states, a jury \textbf{must presume} negligence when a statute is breached. The defendant is free to rebut. California basically follows this rule, with a few exceptions.\footnote{Cal. Evid. C. § 669. See course reader p. 11.} 
    \item In states that do not follow negligence per se, juries are free to (but need not) \textbf{infer} that breach of statute constitutes negligence---e.g., a car doesn't slow down and hits a pedestrian in a crosswalk.
    \item Plaintiff can, and usually will, plead both common law negligence and negligence per se.
    \item Compliance with a statute is generally not proof of due care.
    \item ``Statutory purpose doctrine'': for the statute to be relevant, the harm that occurred must have been the type that the statute was intended to prevent. However, statutory purpose can sometimes be unclear, and it may change through time.
    \item ``Dual purpose doctrine'': a statute may have more than one narrow purpose.
    \item Generally (including Levy): proof of compliance with a statute is \textbf{never} proof of due care. Criminal statutes set a minimum of conduct that could be below what we'd call due care. Some cases take the opposing view.
    \item A federal statute may preempt what would otherwise be a state cause of action. Types of preemption: explicit, conflict, and field.
\end{enumerate}

\subsubsection{\emph{Wawanesa Mutual Insurance Co. v. Matlock}}

\begin{enumerate}
    \item A minor bought cigarettes for another minor, who later dropped the cigarette and caused a fire that led to significant property damage. The insurer sued the first minor's father, and the trial court found for the insurer. The appellate court overturned the ruling. It argued that the statute in question was meant to protect against the health hazards of tobacco, not the fire hazard, and therefore cannot be used to establish a standard of conduct in this case.
\end{enumerate}

\subsubsection{\emph{Stachniewicz v. Mar-Cam Corporation}}

\begin{enumerate}
    \item A patron injured in bar brawl sued the bar owner. The plaintiff relied on (1) an Oregon statute which prohibits giving alcohol to an intoxicated person and (2) an Oregon regulation that prevents bar owners from allowing disorderly conduct on their premised. The trial court found for the defendant. The appellate court overturned, reasoning that (1) the statute is inapplicable because the brawler was already drunk when he arrived, so there is no way to tell if another drink caused the brawl, but (2) the regulation was intended specifically to protect customers from injury, and therefore can be an appropriate standard for negligence in this case.
\end{enumerate}

\subsubsection{\emph{Gorris v. Scott}}

\begin{enumerate}
    \item Several sheep on a ship were swept overboard. The plaintiff sued the shipowner, arguing that the Contagious Diseases (Animals) Act required the shipowner to enclose the sheep in pens of certain dimensions, which the shipowner failed to do. The court found in favor of the shipowner, reasoning that the Act was intended to prevent the spread of contagious diseases, not to prevent sheep from falling overboard.
    \item \textbf{Statutory purpose doctrine}: for the statute to be relevant, the harm must be one of the harms the statute was meant to prevent.
\end{enumerate}

\subsection{Cause in Fact}

\begin{enumerate}
    \item Plaintiff must show that the defendant's negligence was a cause in fact of the harm.
    \item Traditionally: plaintiff must prove that the harm would not have occurred \textbf{but for} the defendant's actions.
    \item If there are multiple causes of harm, each can be but-for causes as long as the harm would not have occurred without it.
    \item If there are multiple causes of harm, but none alone is a but-for cause, courts can use the \textbf{substantial factor test}. See \emph{Northington} below.
    \item Substantial factor test in the Second Restatement:
    \begin{enumerate}
        \item % todo: add
    \end{enumerate}
    \item The biggest different between but-for test and substantial factor test is that under the substantial factor test, it's much more likely to go to a jury.
    \item \textbf{Proximate cause} removes liability when ``the connection between the plaintiff's harm and defendant's liability is unforeseeable or so attenuated that public policy precludes liability.''\footnote{Casebook p. 206.}
    \item When two people are ``acting in concert'' (i.e., trying to do the same thing), and one is the negligent actor, the court can hold both parties liable.
    \item \emph{Summers v. Tice}: CA Supreme Court adopted \textbf{alternative liability test}: when one of two negligent defendants probably caused a harm, and it has not been shown that it is more likely than not that either caused it, then each will be held jointly and severally liable for the full amount of the harm.
    \begin{enumerate}
        \item Restatement Second says the test applies when there are ``two or more'' defendants.
        \item Also think about the effect of Prop 51 on \emph{Summers v. Tice}: not clear whether it applies to these joint tortfeasor cases or not.
    \end{enumerate}
    \item \emph{Sindell}, \textbf{market share liability}: each defendant shall be held liable for the proportion of the judgment represented by its share of the market unless it can demonstrate that it did not manufacture the product that caused the plaintiffs' injuries. This is the case in CA, but not in NY.
    \item Toxic torts: what to do when there is no harm, but only enhanced risk? One approach: award damages, but to a lesser amount, based on the percent chance of the harm. Seond approach (in CA): will not award general damages, but will award damages for medical surveillance.
\end{enumerate}

\subsubsection{\emph{East Texas Theatres, Inc. v. Rutledge}}

\begin{enumerate}
    \item At the defendant's movie theater, somebody threw a bottle from a balcony which struck and injured the plaintiff. The jury found the theater liable because it negligently failed to remove ``rowdy persons'' from the balcony during the game, and the Texas appellate court affirmed. The Texas Supreme Court clarified that proximate cause has two elements: (1) cause-in-fact and (2) foreseeability. The court held that the prosecution failed to show that the injuries would have occurred but for the removal of the ``rowdy persons.'' It reversed the lower court's ruling and held for the defendants.
\end{enumerate}

\subsubsection{\emph{Anderson v. Minneapolis, St. P. \& S. S. M. Ry. Co.}}

\begin{enumerate}
    \item A spark from a railroad started a fire in a bog on one side of the defendant's property. Another unrelated fire was burning on the other side. The fire from the railroad destroyed the defendant's property, and a few days later it joined with the other fire to make one big fire. The railroad argues that it cannot be held liable because the defendant's house would have been destroyed by the other fire anyway. The trial court refused to instruct the jury to follow a rule from an earlier case, \emph{Cook}, which held that there is no liability when two fires jointly destroy property. On this basis, the trial court found for the plaintiff. The railroad requested a motion for judgment notwithstanding the verdict, which was denied. On appeal, the Supreme Court of Minnesota held that the trial court was correct in refusing to apply the \emph{Cook} rule and found for the plaintiffs.
    \item \textbf{Substantial factor test}: If two independent fires join to cause property damage, there is joint liability, even if neither alone is a but-for cause. Redundant causation is not necessary.
    \item Courts split on whether to use the substantial factor test when only one actor is liable. California courts do use it (and reject the but-for test).
\end{enumerate}

\subsubsection{\emph{Northington v. Marin}}

\begin{enumerate}
    \item The plaintiff, a prison inmate, sued the defendant, a prison guard, for circulating rumors that labeled him a snitch and caused other inmates to assault him. Other guards had spread the same rumors. The trial court found that although the defendant's action was not a but-for cause (since the harm would have occurred without his action), his contribution to the harm was nonetheless a \textbf{substantial factor}. The Tenth Circuit affirmed: ``Multiple tortfeasors who concurrently cause an indivisible injury are jointly and severally liable; each can be held liable for the entire injury.''
\end{enumerate}

\subsubsection{\emph{Herskovitz v. Group Health Cooperative of Puget Sound}}

\begin{enumerate}
    \item The plaintiff brought the action on behalf of her husband, a deceased lung cancer patient, against a doctor that negligently failed to diagnose the patient's lung cancer on his first visit, proximately causing his chance of survival to drop from 39 percent to 25 percent. Neither fact was in dispute. The defendant argued that the plaintiff must prove that the patient ``probably'' would have lived but for the negligence---that is, without the doctor's negligence, the patient's chance of survival must have been more than 50 percent. The trial court granted summary judgment for the defendant on this argument. The Supreme Court of Washington reversed, arguing that any other decision would mean a ``blanket release'' for doctors' negligence any time the patient's chance of survival was less than 50 percent. The court reasoned that if a defendant's acts have \emph{increased the risk} of harm to the plaintiff, a jury should decide whether the increased risk actually caused the harm in question.
\end{enumerate}

\subsubsection{\emph{Summers v. Tice}}

\begin{enumerate}
    \item The \emph{Summers} rule applies where there are a small number of defendants, only one of them committed the harm, and we don't know which one.
    \item The plaintiff and the two defendants were hunting quail. The two defendants shot at a quail in the direction of the plaintiff. The plaintiff suffered injuries, but it's not clear which defendant's shot was the cause. The court reasons that in this case, the burden of proof shifts to the defendants to determine which one of them caused the injury. If they cannot, ``each defendant is liable for the whole damage whether they are deemed to be acting in concert or independently.'' The lower courts found the defendants liable and the Supreme Court of California affirmed.
    \item Can you hold three defendants liable under the \emph{Summers} test?
    \item Another case with joint tortfeasors, see \emph{Drabek v. Sabley} above (kids throwing snowballs at cars).
\end{enumerate}

\subsubsection{\emph{Sindell v. Abbott Laboratories}}

\begin{enumerate}
    \item The plaintiff was harmed by DES, a prenatal drug intended to protect against miscarriages but which turned out to pose significant danger to unborn children. The plaintiff did not know which company manufactured the specific drug her mother took, but since several companies manufactured the drug according to the same formula, she sued them all. The companies won a dismissal at trial on the grounds that the plaintiff could not identify which company caused the harm.
    \item The Supreme Court of California considered four theories of liability:
    \begin{enumerate}
        \item The \emph{Summers} test: this fails because there are so many defendants (over 200) that it is highly unlikely that any one of them caused this specific injury.
        \item The ``concert of action'' theory: if the defendants had acted in concert to cause the injury, they would be equally liable. In this case, there is not sufficient evidence to show that the defendants had a common plan to cause harm (e.g., by conducting inadequate safety tests or giving insufficient safety warnings).
        \item ``Industry-wide'' or ``enterprise'' liability: if an entire industry cooperates on an element of the harm in question---e.g., by delegating safety testing to a trade association---they can be held jointly liable. Here, the fact that DES manufacturers shared testing and promotion methods does not establish industry-wide liability, because (1) there are so many manufacturers and (2) safety standards are mostly regulated by the FDA.
        \item \textbf{Market share liability}---a variation of the \emph{Summers} test: each manufacturer's liability and share of the damages are proportionate to its market share.
    \end{enumerate}
    \item Relying on the fourth theory, the Supreme Court of California reversed, allowing the plaintiff to proceed with her cause of action.
    \item Most states have not adopted market share liability.
    \item Defendants are allowed prove definitively that they did not contribute to the harm (e.g., if they can show that they did not produce the drug at the time).
    \item Some states require defendants to be joined so that a significant share of the market is represented, and that missing market share proportionally reduces the plaintiff's compensation. Usually (but not always) this must be the nationwide market.\footnote{Casebook p. 229 n. 2.}
\end{enumerate}

\subsubsection{\emph{Ayers v. Township of Jackson}}

\begin{enumerate}
    \item A town in New Jersey was found to have caused toxic exposure by its ``palpably unreasonable'' management of a landfill. Plaintiffs did not develop any illnesses, but they sought to recover (1) damages for the enhanced risk of future illness due to exposure and (2) regular medical testing for diseases from exposure. The Supreme Court of New Jersey found that he task of litigating hypothetical injuries would unreasonably strain the tort system (although it suggests that the state legislature could pass a remedy that allowed damages if toxic exposure caused a ``statistically significant incidence of disease''). On the second claim, it held for the plaintiffs.
\end{enumerate}

\subsection{Duty and Proximate Cause}

\begin{enumerate}
    \item Most courts speak about duty and proximate cause as separate elements. However, you could porbably build a torts system with just one or the other.
    \item Palsgraf: four justices deal with it as a duty question, and three in dissent view it as a proximate cause problem.
    \item Proximate cause: whether there should be liability, even though teh defendant's negligence actually caused the harm. Akin to duty, where we ask whether the defendant should be immunized from duty.
    \item Two views of proximate cause:
    \begin{enumerate}
        \item 1. Rigorous analytical meaning: scope of the risk analysis. There are fact situations where we want to limit liability because the actual harm was not one of the foreseeable harms that made us deem the act to be negligence. 
        \item (Levy's preference). 2. There are certain fact situations where even though the defendant was negligent, and it caused harm, we choose for policy reasons to have no liability. Courts can conclude that defendant was under no duty; in other cases, courts find that a defendant's conduct was not a proximate cause.
    \end{enumerate}
    \item Intervening superseding events vs. dependent/naturally occurring.
    \item ``Danger invites rescue.''
\end{enumerate}

\subsubsection{\emph{Atlantic Coast Line R. Co. v. Daniels}}

\begin{enumerate}
    \item Cause in effect are infinite. An act is the proximate cause if it's close enough. Courts and juries have to rely on reason and common sense to judge whether a cause is proximate.
    \item Some sources, like the Restatement on Torts, prefer ``legal cause.''
    \item Proximate cause is a tool for protecting defendants.
\end{enumerate}

\subsubsection{\emph{Palsgraf v. The Long Island Railroad Company}}

\begin{enumerate}
    \item A railroad employee caused a passenger's package to fall. The package turned out to be full of fireworks. It exploded, causing a scale to break and injure the plaintiff.
    \item The trial court found negligence. The Court of Appeals here reversed.
    \item Cardozo: negligence requires the defendant to have a duty to the plaintiff. There must be a point in the chain of causation where an actor is no longer liable---otherwise, anybody who jostles someone in a crowd could be liable. To be negligent, the actor must have breached a reasonable standard of care. In this case, however, the railroad employee could not have known that the package was full of fireworks.
    \item Andrews, dissenting: The actor owes a duty of care to the public at large. Ultimately, proximate cause is about expediency, not logic, and judges must rely on common sense. In this case, the defendant's actions were a but-for cause of the plaintiff's injuries. It's not possible to say that plaintiff's injuries ``were not the proximate result of the negligence.''
\end{enumerate}

\subsubsection{Directness vs. Foreseeability: \emph{Overseas Tankship (U.K.) Ltd. v. Morts Dock \& Engineering Co. (The Wagon Mound) Privy Council}}

\begin{enumerate}
    \item The plaintiffs' ship, the \emph{Corrimel}, was moored for repairs. The appellants' ship, the \emph{Wagon Mound}, was moored nearby. The crew of the \emph{Wagon Mound} accidentally spilled a large amount of oil into the bay. They left soon after without cleaning up the oil.
    \item The plaintiff checked with the manager of the wharf where the \emph{Wagon Mound} was moored to see if the oil on the water was flammable. They agreed it was not. Soon after, a small drop of molten metal from the plaintiffs' worked ignited the oil, severely damaging the \emph{Corrimal} and the wharf.
    \item \emph{In re Polemis} dealt with another scenario involving fire and negligence. Although the fire was not a foreseeable consequence of the negligence, it was clear that the defendant's action was the direct cause, and the court held for the plaintiffs.
    \item The court here replaced the direct test from \emph{Polemis} with a foreseeability test.
    \item The defendants could not have foreseen a massive fire to be the result of their negligence. Ruling for the defendants.
    \item \emph{Kinsman}: foreseeability is a weaker requirement when the consequences are direct and the damage is of the same sort that was risked.\footnote{Casebook p. 258.}
\end{enumerate}

\subsubsection{Intervening Events: \emph{Thomas v. United States Soccer Fedn.}}

\begin{enumerate}
    \item The plaintiff suffered injuries when a soccer game turned violent. He sued the soccer federation for failing to provide a properly trained referee and failing to maintain a safe playing environment. The defendants moved for a summary judgment on the grounds that the alleged negligence was not the proximate cause. The court reasoned that when an intervening act occurs, liability will turn on whether the defendant should have foreseen the act as a consequence of the negligence. It reversed the lower courts and granted the motion for dismissal.
    \item ``Superseding intervening forces are those new forces which are extraordinarily unexpected.''\footnote{Casebook p. 261.}
    \item Intervening criminal acts are generally found to be unforeseeable and therefore superseding.
    \item ``Dependent'' intervening forces are results of the defendant's action (e.g., an ambulance driver's collision while rushing to the scene of the defendant's accident). ``Independent'' intervening forces do not have a causal connection to the defendant (e.g., a lightning bolt).
    \item ``...ultimately the determinative issue is whether or not the intervening force is extraordinarily unexpected.''\footnote{Casebook p. 263.}
\end{enumerate}

\subsubsection{\emph{Bigbee v. Pacific Telephone and Telegraph Co.}}

\begin{enumerate}
    \item Plaintiff was inside a telephone booth. He saw a car approaching, and he claims he tried to get out but couldn't. He alleges the telephone booth company was negligent in (1) its manufacture of the booth, which prevented his escape, and (2) its placement in proximity to a busy street, where damage from an oncoming car was foreseeable. The lower courts upheld a motion to dismiss. Here, the Supreme Court of California held that a jury could find that the danger was reasonably foreseeable. Reversed and remanded.
    \item Unlikely intervening events are often not found to be superseding events. For instance, if an owner leaves the keys in her car in a high crime area, she may be liable for the harm the car thief causes. (But generally, car owners are not responsible for the actions of car thieves.)
\end{enumerate}

\subsubsection{The Egg-Shell Plaintiff Rule: \emph{Steinhauser v. Hertz Corporation}}

\begin{enumerate}
    \item The plaintiff was involved in a car accident. She suffered no injuries, but the accident triggered serious schizophrenia. The court held that as long as there is a causal relationship between the small accident and the catastrophic result, the defendant can be held liable for the ``precipitating cause.'' The probability that the condition would have developed is not a defense, but it can be considered in fixing damages.
    \item The large injury from the small cause need not be foreseeable.
\end{enumerate}

\subsection{Proof of Negligence: Res Ipsa Loquitur}

\begin{enumerate}
    \item Res ipsa loquitur: ``the thing speaks for itself.''
    \item It usually has three requirements (with variations among jurisdictions):
    \begin{enumerate}
        \item The accident would not have occurred without negligence.
        \item The negligent act was within the actor's control.
        \item The plaintiff was not at fault (i.e., no contributory negligence).
    \end{enumerate}
    \item It's an expansion of the common sense cookie jar rule: if a parent returns to see a child next to a broken cookie jar, it's reasonable to infer that the child broke the cookie jar.
    \item We can generally assume that a car in motion that hits a pedestrian was negligent---you don't need res ipsa loquitur to show negligence.
    \item If there is no evidence of res ipsa loquitur, whether the state is a presumption state or an inference state. If it's a presumption state, the plaintiff can receive a directed verdict; if it's merely an inference, the jury is free to draw the inference or not.
    \item If the defendant presents evidence of due care, then in all jurisdictions the question would go to a jury.
    \item If one of the three conditions is undercut, the jury is given the ``conditional res ipsa'' instruction: if you find A, B, and C, there is a presumption of negligence.
    \item Some courts have relaxed the requirement that the defendant must have had exclusive control of the accident.\footnote{Casebook p. 275 n. 4.}
    \item Some courts follow the \emph{Ybarra} rule, which expands the res ipsa loquitur doctrine to medical cases with multiple defendants, where multiple defendants did not have exclusive control of the accident and not all of them were necessarily negligent. It's an extension of the situation where a teacher punishes the entire class for breaking the goldfish bowl.
\end{enumerate}

\subsubsection{\emph{Krebs v. Corrigan}}

\begin{enumerate}
    \item The defendant inexplicably flew through the air and landed on the plaintiff's plexiglass sculpture, destroying it. The trial court granted the defendant's motion for a directed verdict.
    \item ``...human bodies do not generally go crashing into breakable personal property,'' said the appellate court.
    \item Defendant argued (1) that res ipsa loquitur does not apply when the instrumentality is a human body and (2) the doctrine does not apply because there was an eyewitness. The court rejected both of these arguments.
    \item The doctrine exists, the court reasoned, to deal with cases where only the defendant knows the details of the negligent act.
    \item The appellate court held that the evidence was sufficient to raise an inference of negligence, so it reversed the directed verdict for the defendants.
\end{enumerate}

\subsubsection{\emph{Ybarra v. Spangard}}

\begin{enumerate}
    \item (Levy: this case is better thought of as involving a causation issue.)
    \item The plaintiff underwent surgery for appendicitis. During the procedure, he suffered a shoulder injury that caused paralysis and muscle atrophy. The trial court entered a judgment of nonsuit for all defendants.
    \item The plaintiff argued that the doctrine of res ipsa loquitur should apply to the defendants, all of whom were involved at different stages of his medical care.
    \item The defendants argue that the plaintiff cannot show that any single defendant caused the injury.
    \item As in \emph{Krebs}, the court noted that the purpose of the res ipsa loquitur doctrine is to address cases where the circumstances of the negligence were unknown to the plaintiff (in this case, because he was unconscious).
    \item Classic examples where res ipsa loquitur would apply: passenger sitting awake in a train car at the time of a collision; person walking down the street and hit by a falling object.
    \item These sorts of cases ``raise the inference of negligence, and call upon the defendant to explain the unusual result.''\footnote{Casebook p. 279.}
    \item It could be found in this case that some of the defendants are liable and others are absolved. But that should not preclude the application of res ipsa loquitur. It would not be reasonable to ask the plaintiff to identify which of the individual defendants were responsible for the harm.
    \item The defendants' argument would undermine the rights of patients to recover for injuries suffered while unconscious.
    \item Judgment of nonsuit was reversed.
\end{enumerate}

\subsection{Duty and Limitations on Duty}

% \item See Ca9l. Civ. Code 1714.
% \item No duty to affirmatively act, with a few exceptions:
% \begin{enumerate}
%     \item One who causes injury may have a duty to rescue.
%     \item Relationship between P and D may create a duty: common carrier, land occupiers, innkeeper, parent...
%     \item Beginning an undertaking that places the victim in a position that makes them less likely to be rescued can lead to liability.
%     \item Good samaritan statutes protect from liability.
% \end{enumerate}
% 
%\subsubsection{Failure to Act: \emph{L. S. Ayres \& Co. v. Hicks}}
%
%\begin{enumerate}
%    \item todo
%\end{enumerate}
%
%\subsubsection{\emph{Miller v. Arnal Corp.}}
%
%\begin{enumerate}
%    \item todo
%\end{enumerate}
%
%\subsubsection{\emph{Wells v. Hickman}}
%
%\begin{enumerate}
%    \item todo
%\end{enumerate}
%
%\subsubsection{\emph{Tarasoff v. The Regents of the University of California}}
%
%\begin{enumerate}
%    \item todo
%    \item [Levy lecture: ]Relationship between defendant and third party can create a duty to a stranger.
%\end{enumerate}
%
%\subsubsection{\emph{Davidson v. City of Westminster}}
%
%\begin{enumerate}
%    \item todo
%\end{enumerate}
%
\subsubsection{Mental Distress: \emph{Thing v. La Chusa}}

\begin{enumerate}
    %%%%%%%%%%%%%%%%%%%
    \item Levy lecture 10/24: % TODO
    \begin{enumerate}
        \item Common law: no recovery unless there was physical contact. No duty to bystandars.
        \item amaya [?]: Bystander in the ``zone of danger'' can be compensated
        \item \emph{Dillon}:
        \begin{enumerate}
            \item Was the plaintiff at or near the scnee?
            \item Was the distress caused by sensory and contemporaneous observance?
            \item Does P have close relationship with V?
            \item [No physical injury or manifestation requiremenet.]
        \end{enumerate}
        \item \emph{La Chusa}: court turned dillon guidelines into requirements. a ``jurisprudence of categories.''
        \item What if ther eis only a fear of cancer? Potter: disease must be more likely than not to appear for p to recover for distress. Concedes that someone who had a 30\% chance would suffer distress, but it wanted a bright line.
        \item Asbestiosis and tobacco: move very far from what the tort system was meant to deal with. Courts develop clever solutions. Potter did allow compensation for medical monitoring.
    \end{enumerate}
    %%%%%%%%%%%%%%%%%%%%%%%%

    \item The mother of a child struck by a car sued the driver for negligent infliction of emotional distress (NIED). The trial court granted a summary judgment in favor of the defendant. The Court of Appeal reversed, holding that the mother may recover. The California Supreme Court reversed the appellate court, holding that the trial court was correct in granting summary judgment.
    \item In \emph{Amaya v. Home Ice, Fuel, \& Supply Co.}, the court held that plaintiffs must have been within the ``zone of danger'' to recover NIED damages.
    \item Five years later, the court overruled \emph{Amaya} in \emph{Dillon v. Legg}. In that case, the mother of the victim may have been endangered by the defendant's conduct, but the sister was not, leading to an incongruous result from the ``zone of danger'' test. The court held that recovery should be based on the traditional tort principles of foreseeability, proximate cause, and consequential injury. The \emph{Dillon} framework considers three factors:
    \begin{enumerate}
        \item Whether the plaintiff was \textbf{located near the scene}.
        \item Whether the plaintiff's shock resulted from the emotional impact of \textbf{``contemporaneous observance''} of the incident.
        \item Whether the plaintiff and victim are \textbf{closely related}.
    \end{enumerate}
    \item Under \emph{Dillon}, the jury will decide on a case-by-case basis ``what the ordinary man under such circumstances should reasonably have foreseen.''\footnote{Casebook p. 325.}
    \item The court here argued that \emph{Dillon} created massive uncertainty. The court replaced \emph{Dillon} with a new rule for finding NIED, which requires three factors:\footnote{Casebook p. 323.}
    \begin{enumerate}
        \item Plaintiff must be \textbf{closely related} to the victim.
        \item Plaintiff must be \textbf{present at the scene} and aware that an injury has occurred.
        \item Plaintiff must suffer \textbf{emotional distress beyond that which would be anticipated in a disinterested witness}.
    \end{enumerate}
    \item The court's motivation was to ``limit liability and establish meaningful rules for application by litigants and lower courts.''\footnote{Casebook p. 329.} Policy reasons included guarding against fraudulent claims and limiting defendants' liability.\footnote{Casebook p. 324.}
    \item The dissent argued that \emph{Dillon} was meant to be a flexible test based on the basic principles of torts The court here has replaced it with an arbitrary rule. The ``policy reasons'' for replacing the foreseeability requirement are not convincing.
    \item NIED was originally limited to cases where the plaintiff was physically impacted, i.e., for emotional distress associated with personal injuries. Virtually all courts have abandoned this rule.\footnote{Casebook p. 333.}
    \item A slight majority of courts still adhere to the ``zone of danger'' test (like that in \emph{Amaya}, above).
    \item The \emph{Dillon} approach is becoming a majority. Some jurisdictions allow non-visual perception of the accident or perception of its aftermath. The ``close relationship'' requirement generally includes only blood relatives and family memembers. California recently extended it to include domestic partners. Most \emph{Dillon} jurisdictions require a physical manifestation of the emotional distress, e.g., a heart attack or stomach pains, but not crying or insomnia.\footnote{Casebook pp. 334--336.}
    % TODO: add notes 7 and 8 on pp. 336-337
\end{enumerate}

%\subsubsection{\emph{Potter v. Firestone Tire and Rubber Co.}}
%
%\begin{enumerate}
%    \item todo
%\end{enumerate}
%
%\subsubsection{C.C.P. § 377}
%
%\begin{enumerate}
%    \item todo
%\end{enumerate}
%
%\subsubsection{Wrongful Death and Survival Actions: \emph{Gary v. Schwartz}}
%
\begin{enumerate}
%    \item todo
    \item Common law: no recovery for wrongful killing. Also, if a person had a tort action and then died, the action would end.
    \item Certain named categories of relatives can recover for wrongful death. They can recover money that was used to support them. In all but a few jurisdictions, damages are limited to pecuniary losses, not including pain and suffering. IN CA, however, survivors can collect for monetary value of loss of companionship--a way of softening the effect of not allowing pain and suffering.
    \item Survival actions: suits brought in the name of the estate of the deceased. How much did the estate lose because of the deceased? and No recovery for pain and suffering of victim before daeth
    \item Loss of consortium: loss of love, companionshiop, society, sex, services--all because of injury to another. In CA, only a spouse can bring a loss of consortium action.
    \item Wrongful birth: if a couple gets sterilized or an abortion, which is performed negligently: if teh child is healthy, some states do not allow recovery. in CA, one can recover for the cost of raising the child, but defendant can ask that it be reduced for the joy and benefits the parents received. levy has problems with that bc of question........ [??????]
    \item There can also be recoery if malpracitce failed to disclose condition fo the fetus. If child is severly unhealthy, recovery can include special needed care.
    \item wrongful life? Can a child bring its own action? yes, if it's necessary for care after the parents' death. CA does not allow recovery for pain of life, but does allow recovery for special medical care.
\end{enumerate}
%
%\subsubsection{\emph{Selders v. Armentrout}}
%
%\begin{enumerate}
%    \item todo
%\end{enumerate}
%
%\subsubsection{\emph{Compania Dominicana de Aviacion v. Knapp}}
%
%\begin{enumerate}
%    \item todo
%\end{enumerate}
%
%\subsubsection{Loss of Consortium and Society: \emph{Borer v. American Airlines, Inc.}}
%
%\begin{enumerate}
%    \item todo
%\end{enumerate}
%
%\subsubsection{Wrongful Life and Wrongful Birth: \emph{Turpin v. Sortini}}
%
%\begin{enumerate}
%    \item todo
%\end{enumerate}
%

%%% PREMISES LIABILITY


%\subsubsection{\emph{Rowland v. Christian}}
%
%\begin{enumerate}
%    \item todo
% \item All land occupiers owe a duty of care to anyone on the land, with some exceptions for felonious trespassers.
% \item Land occupiers should be held liable under general negligence principles, not jurisprudential categories.
% \item Factors when a land occupier defendant has a duty of care to plaintiff:
% \begin{enumerate}
%     \item Duty of care exists unless policy considerations dictate otherwise.
%     \item Was the harm foreseeable?
%     \item Was there a degree of certainty of the harm?
%     \item Closeness of connection betwen p's conduct and d's harm
%     \item [Morality....?]
%     \item Policy of preventing future harms
%     \item Burden of duty rule on D
%     \item Availability of insurance
% \end{enumerate}
\begin{enumerate}
    \item Duty is a question of law for the court.
    \item Foreseeability of harm was the most iportant question, and should go to a jury. --> subsequently changed to be the court's duty
\end{enumerate}
%
%\subsubsection{\emph{Ann M. v. Pacific Plaza Shopping Center}}
%
\begin{enumerate}
%    \item todo
\item Landowner rarely owes duty to provide a security guard, unless the harm occurs after a similar prior incident.
\item Returns to a jurisprudence of categorization: instead of a loose foreseeability notion, it gave a narrow rule to be followed: was there a prior similar incident?
\end{enumerate}
%
%\subsubsection{\emph{Wiener v. Southeast Childcare}}
%
%\begin{enumerate}
%    \item todo
%\end{enumerate}
%
%\subsubsection{Negligent Misrepresentation: \emph{Bily v. Arthur Young \& Co.}}
%
%\begin{enumerate}
%    \item todo
%\end{enumerate}
%
%\subsubsection{Economic Loss: \emph{J'Aire Corp. v. Gregory}}
%
%\begin{enumerate}
%    \item todo
%\end{enumerate}

\subsection{Contributory Negligence}

\begin{enumerate}
    \item (No cases.)
    \item If you're even a little bit negligent, you can recover nothing.
    \item Juries often didn't apply it. They might lower recovery based on the plaintiff's contributory negligence.
\end{enumerate}

\subsection{Comparative Negligence}

%%%%%%%%%%%%%%%%%%%%%%%%
\begin{enumerate}
    \item Contributory neg is no longer a bar to recovery. Instead, P's recovery is reduced according to P's fault.
    \item Abolished doctrine of last clear chance.
    \item \textbf{pure} comparative negligence is used.
    \item Must compare plaintiff's fault to fault of all actors, whether or not they are parties to the suit. (this is obviously in plaintiff's favor).
\end{enumerate}
%%%%%%%%%%%%%%%%%%%%%%%%

\subsubsection{\emph{Li v. Yellow Cab Co.}}

\begin{enumerate}
    % todo add facts
    \item Rule: ``pure'' comparative negligence \emph{diminishes} damages awarded in proportion to the plaintiff's negligence.
    \begin{enumerate}
        \item Court's wording: ``In all actions for negligence resulting in injury to person or property, the contributory negligence of the person injured in person or property shall not bar recovery, but the damages awarded shall be diminished in proportion to the amount of negligence attributable to the person recovering.'' %TODO add p. number -- use westlaw -- https://a.next.westlaw.com/Document/I36ea3c9ffadc11d983e7e9deff98dc6f/View/FullText.html?transitionType=UniqueDocItem&contextData=(sc.Search)
    \end{enumerate}
    \item Contributory negligence no longer applies in California.
    \item The court reasoned that contributory negligence has a ``lottery aspect.'' ``Modified'' comparative negligence, which allows plaintiffs to recover if they are less than 50\% at fault, only shifts this aspect to a different place.\footnote{Casebook p. 427} (Most cases use the modified version.\footnote{Casebook p. 428.})
    \item ``Last clear chance'' doctrine: even if the plaintiff was negligent, if the defendant had the last clear chance to avert the harm, the plaintiff could recover. The court removed the last clear chance rule in this case. % todo why?
    \item % TODO Assumption of risk: see p. 426 toward bottom. changed in knight v jewett.
    \item Plaintiff's fault if compared with \emph{all actors}---regardless of whether they are parties to the suit.
    \begin{enumerate}
        \item How to determine fault of nonparties? Sometimes: eyewitness testimony. 
    \end{enumerate}
    \item Other rules in California (not appearing in case):
    \begin{enumerate}
        \item No comparative negligence if the defendant's conduct was intentional.
        \item Comparative negligence applies if defendant's conduct is merely willful, wanton, or reckless. (Intent = desire or substantial certainty; gross negligence/willful wanton misconduct/recklessness = conscious disregard of risk; negligence = should have known the risk. Difficult to draw the line between substantial certainty and recklessness.)
        \item Split of authority when plaintiff's conduct was willful.
    \end{enumerate}
    \item Calculating liability in comparative negligence (these answers change when CA Proposition 51 is in play):
    \begin{enumerate}
        \item X suffers \$40,000 in damages. If X was 40\% at fault, \textbf{diminish} reward by \$16,000.
        % TODO: define joint and several liability; and comparative indemnity
        % joint liability: you're liable for the full amount, even if there are other defendants
        % several liability: you're only liable for your portion
    \end{enumerate}
    \item CA legislature decided not to adopt a comparative negligence statute. The court believed it had authority on its own to adopt comparative negligence judicially.
\end{enumerate}

%%%%%%%%
\begin{itemize}
    \item exculpatory clauses: plaintiff must specifically waive right to sue for certain injuries. first: determine if harm fell within the words of the clause. if they do, determine if it is an unconscionable agreement.
    \item to waive liability for negligence, the waiver must explicitly state this. in santa barbara v superior court, court ruled that you cannot waive liability for gross negligence. --> many suits brought under the term 'gross negligence.'
    \item implied assumption of risk: at common law, it was a total bar. so it didn't matter whether plaintiff's actions were contributory negligence or assumption of risk.
    \item a of r: implied and knowing waiver of liablity for damages caused by defendant. danger must have been seen with a fair amount of specifity. after Li, however, the issue is whether a of r still acts as a total bar. li tried to distinguish between reasonasble and unreasonable risk--badly chosen words. changed in knight v jewett to primary v secondary assumption of r.
    \item TODO: add primary and secondary assumption of risk.
    \item most used in sports and employment (firefighter's rule) contexts.
\end{itemize}
%%%%%%%%

\subsubsection{Assumption of Risk: \emph{Murphy v. Steeplechase Amusement Co.}}

\begin{enumerate}
    \item Plaintiff fell on ``the Flopper'' and injured himself.
    \item Trial court found for the plaintiff.
    \item Appellate court affirmed trial court.
    \item Defendant has the burden of proof in assumption of risk. To prove the assumption of risk defense, the defendant must prove that the plaintiff took a \textbf{knowing and voluntary} risk. It's a subjective test.
    \item New York Supreme Court reversed. 
    \item (If the defendant had been contributorily negligent, at the time he would not have been able to recover, because the court had not yet adopted comparative negligence.)
\end{enumerate}

\subsubsection{\emph{Rush v. Commercial Realty Co.}}

\begin{enumerate}
    \item Plaintiff fell through a trap door in an outhouse.
    \item Court found that the plaintiff had no choice but to use the outhouse. Therefore, there was no assumption of risk. Lower courts are affirmed.
    % \item Cf. McDermott p. 437. TODO add -- no choice --> no assumption risk
\end{enumerate}

\subsubsection{\emph{Emmette L. Barran, III v. Kappa Alpha Order, Inc.}}

\begin{enumerate}
    \item todo
\end{enumerate}

%\subsubsection{\emph{Knight v. Jewett}}
%
%\begin{enumerate}
%    \item todo
%\end{enumerate}
%
%\subsubsection{\emph{Priebe v. Nelson}}
%
%\begin{enumerate}
%    \item todo
%\end{enumerate}
%
%\subsubsection{\emph{Shin v. Ahn}}
%
%\begin{enumerate}
%    \item todo
%\end{enumerate}

\subsubsection{Immunity: \emph{Metcalfe v. County of San Joaquin}}

\begin{enumerate}
    \item Affirmative defense: defendant responds to a plaintiff's complaint.
    \item Immunity: defendant cannot be sued because (1) their status makes them free from liability, and (2) the relationship between plaintiff and defendant--e.g., until recently, a parent couldn't sue a child.
    \item Governments are immune by default. Governmental liability is based entirely on statute.\footnote{The main statute is the CA Government Claims Act.}
    \item There are shortened claim statutes and statutes of limitations for suits against the government. For claims against the government, you must submit the claim to a governmental agency within six months. There are then another six months to bring suit. (The general statute of limitations is two years, but not in cases of governmental liability.)
    \item Historical immunities from tort liability:
    \begin{enumerate}
        \item Charities (until the 50s), because people didn't want donations to be used to pay for suits.
        \item Intra-family: parents often couldn't sue children and spouses couldn't sue each other.
        \item Guest statutes: passengers couldn't sue for driver's liability. (Two justifications: (1) not seemly to sue your host and (2) prevents collusion to collect insurance--same justifications for intra-family immunities)
    \end{enumerate}
\end{enumerate}

\subsection{Multiple Parties: Joint and Several Liability}

%%%%%%%%%%%%%%%%%%
\begin{itemize}
    \item Two parties are injured and both are at fault. if both are insured, they both recover and there's no setoff. if one are both are ininsured, things get much more complicated.
    \item In a case of single plaintiffs: at common law, both were jointly and 
    severally liable. **prop 51 abolished j \& s liability for non-economic damages.**
    \item Common law: no contribution--one defendant could not recover against another. Then, CA statute said that if you as a def paid more than your pro rata share (50\%, 33\%, etc.). in american moto, we adopted comparative indemnity.
\end{itemize}
%%%%%%%%%%%%%%%%%%

\subsubsection{Comparative Indemnity: \emph{American Motorcycle Association v. Superior Court}}

\begin{enumerate}
    % todo: add facts
    \item Common law rule: defendants cannot bring in other parties to lawsuits, because (1) plaintiffs should be able to control their own cases, and (2) we should not take court time to shift the loss from one wrongdoer to another.
    \item 40s onward (1957 in CA): \textbf{contribution states} (``Uniform Contribution Act'') prevent defendants from bringing in other defendants, but if there were already two defendants in the suit, and plaintiff collected all damages from one defendant, a defendant could collect pro rated damages from another (e.g., if there were two defendants, they'd each pay 50\%; three defendants would pay 33\%; etc.).
    \item Appellate court here eliminated the doctrine of joint and several liability---thus the Supreme Court's care to reaffirm that the doctrine is still good law.
    \item Rules from \emph{American Motorcycle}:
    \begin{enumerate}
        \item % todo see pp. 470-471
    \end{enumerate}
    \item One of the rationales for not allowing defendants to bring in other defendants is that it would greatly complicate a plaintiff's case. After \emph{American Motorcycle}, 
    \item Plaintiffs can settle with secondary defendants that the primary defendant brings in. But if a third defendant settles with the plaintiff for a disproportionately small amount, he can still be liable to the main defendant for remaining damages (\emph{Tech-bilt v. Woodward-Clyde}). So, the defendants have a quick hearing to make sure the settlement is within a reasonable range.
    \item Plaintiffs will settle with one of multiple defendants because (1) first settlement can pay for the rest of the case, and (2) early in the case you can play defendants off each other (e.g., make settlement offers to both, and offer to accept the first).
    \item \emph{American Motorcycle} dramatically changed day-to-day lawyering. Now, around 20\% of cases involve multiple defendants. Before \emph{American Motorcyle}, a case could not have had multiple defendants.
    \item The set-off problem (\emph{Jess v. Herrmann}):
    \begin{enumerate}
        \item What if A is 75\% at fault and B is 25\% at fault, and both suffer \$100,000 injuries? Should (1) A receive \$25,000 in insurance money and B receive \$75,000, or (2) B recover \$50,000 and A nothing (the set-off problem)?
        \item (Levy: under contract law, there would not be set-off unless insurance companies had expressly stipulated it.)
    \end{enumerate}
\end{enumerate}

\subsubsection{Proposition 51: Fair Responsibility Act of 1986}

\begin{enumerate}
    \item Non-economic damages: things without a price tag, e.g., pain and suffering.
    \item Rules:
    \begin{enumerate}
        \item The full economic damages can be recovered from any defendant individually---i.e., defendants are joint and severally liable for their collective share of the blame. For instance, if the plaintiff is 10\% at fault, B is 30\% at fault, and C is 60\% at fault, then B and C are jointly and severally liable for 90\% of the damages.
        \item Non-economic damages can only be recovered proportionally to each defendant's degree of blame. For instance, if total non-economic damages are \$200,000, and defendant B is 30\% liable, plaintiff can recover \$60,000 in non-economic damages from plaintiff B.
    \end{enumerate}
    \item Example: hotel fails to put a proper lock on a door; someone breaks in and rapes a tenant. The economic losses can be small, but the non-economic consequences can lead to disastrous consequences for the hotel owner.
    \item 
\end{enumerate}


% todo:
% History of the law when multiple tortfeasors are involved
% ---------------------------------------------------------
% 
% [see Levy PPT.]
% 
% Hypo: A and B both cause injury to C.
% 
% Common law: Plaintiff can sue both. If plaintiff only sued A, A could not implead B.
% 
% Contribution statute: plaintiff sues both. Collects all from A. A can recover the pro rata contribution from B, so they each end up paying half.
% --> if plaintiff only sued A, A could *not* implead B under the contribution statute.
% 
% After _American Motorcycle_: defendant can implead another tortfeasor. (Argument against this model: plaintiff should be able to control his own suit. E.g., a simple auto accident could become a complicated products liability case. To avoid this scenario, the plaintiff could settle with the new defendant (or the old)--and it must be a good faith settlement--see _tech-bilt v. woodward-clyde_ which fleshes out the good faith settlement question. there must be a short good faith hearing. after settling, defendant is no longer liable for damages or indemnity. notice of the settlement must be given to the other defendants.)

% once a defendant settles, the defendant is no longer liable for damages or indemnity.

\subsection{Insurer's Failure to Settle within Policy Limits}

\subsubsection{\emph{Crisci v. Security Ins. Co.}}

\begin{enumerate}
    \item On appeal, the real party in interest is DiMares.
    \item A prudent insurer without policy limits would have accepted the settlement.
    \item When an insurance company provides a lawyer for an insured, the lawyer's duty is to the insured, not the insurance company. This is the rule, though it's not always followed.
    \item Barratry
    \item Champerty
\end{enumerate}
