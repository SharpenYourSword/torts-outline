\subsection{Proof of Negligence: Res Ipsa Loquitur}

\begin{enumerate}
    \item Res ipsa loquitur: ``the thing speaks for itself.''
    \item It usually has three requirements (with variations among 
    jurisdictions):
    \begin{enumerate}
        \item The accident would not have occurred without negligence.
        \item The negligent act was within the actor's control.
        \item The plaintiff was not at fault (i.e., no contributory 
        negligence).
    \end{enumerate}
    \item It's an expansion of the common sense cookie jar rule: if a parent 
    returns to see a child next to a broken cookie jar, it's reasonable to 
    infer that the child broke the cookie jar.
    \item We can generally assume that a car in motion that hits a pedestrian 
    was negligent---you don't need res ipsa loquitur to show negligence.
    \item If there is no evidence of res ipsa loquitur, whether the state is a 
    presumption state or an inference state. If it's a presumption state, the 
    plaintiff can receive a directed verdict; if it's merely an inference, the 
    jury is free to draw the inference or not.
    \item If the defendant presents evidence of due care, then in all 
    jurisdictions the question would go to a jury.
    \item If one of the three conditions is undercut, the jury is given the 
    ``conditional res ipsa'' instruction: if you find A, B, and C, there is a 
    presumption of negligence.
    \item Some courts have relaxed the requirement that the defendant must 
    have had exclusive control of the accident.\footnote{Casebook p. 275 n. 
    4.}
    \item Some courts follow the \emph{Ybarra} rule, which expands the res 
    ipsa loquitur doctrine to medical cases with multiple defendants, where 
    multiple defendants did not have exclusive control of the accident and not 
    all of them were necessarily negligent. It's an extension of the situation 
    where a teacher punishes the entire class for breaking the goldfish bowl.
\end{enumerate}

\subsubsection{\emph{Krebs v. Corrigan}}

\begin{enumerate}
    \item The defendant inexplicably flew through the air and landed on the 
    plaintiff's plexiglass sculpture, destroying it. The trial court granted 
    the defendant's motion for a directed verdict.
    \item ``...human bodies do not generally go crashing into breakable 
    personal property,'' said the appellate court.
    \item Defendant argued (1) that res ipsa loquitur does not apply when the 
    instrumentality is a human body and (2) the doctrine does not apply 
    because there was an eyewitness. The court rejected both of these 
    arguments.
    \item The doctrine exists, the court reasoned, to deal with cases where 
    only the defendant knows the details of the negligent act.
    \item The appellate court held that the evidence was sufficient to raise 
    an inference of negligence, so it reversed the directed verdict for the 
    defendants.
\end{enumerate}

\subsubsection{\emph{Ybarra v. Spangard}}

\begin{enumerate}
    \item (Levy: this case is better thought of as involving a causation 
    issue.)
    \item The plaintiff underwent surgery for appendicitis. During the 
    procedure, he suffered a shoulder injury that caused paralysis and muscle 
    atrophy. The trial court entered a judgment of nonsuit for all defendants.
    \item The plaintiff argued that the doctrine of res ipsa loquitur should 
    apply to the defendants, all of whom were involved at different stages of 
    his medical care.
    \item The defendants argue that the plaintiff cannot show that any single 
    defendant caused the injury.
    \item As in \emph{Krebs}, the court noted that the purpose of the res ipsa 
    loquitur doctrine is to address cases where the circumstances of the 
    negligence were unknown to the plaintiff (in this case, because he was 
    unconscious).
    \item Classic examples where res ipsa loquitur would apply: passenger 
    sitting awake in a train car at the time of a collision; person walking 
    down the street and hit by a falling object.
    \item These sorts of cases ``raise the inference of negligence, and call 
    upon the defendant to explain the unusual result.''\footnote{Casebook p. 
    279.}
    \item It could be found in this case that some of the defendants are 
    liable and others are absolved. But that should not preclude the 
    application of res ipsa loquitur. It would not be reasonable to ask the 
    plaintiff to identify which of the individual defendants were responsible 
    for the harm.
    \item The defendants' argument would undermine the rights of patients to 
    recover for injuries suffered while unconscious.
    \item Judgment of nonsuit was reversed.
\end{enumerate}
